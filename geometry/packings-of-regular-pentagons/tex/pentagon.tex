
% The Optimal Packing of Regular Pentagons in the Plane, I
% Tex file started June 23, 2015.
% Notation: c = rho = cos pi /5.
% 

%Title: Packings of Regular Pentagons in the Plane,
%Part I, preliminaries.
% draft Nov, 2015.

%Authors: Thomas Hales, Woden Kusner.

\def\area{\op{area}}
\def\areta{\op{areta}}
\def\c{{\mathbf c}}
\def\K{\op{K}}
\def\aK{a_K}
\def\rat{\Rightarrow}
\def\ra#1{\Rightarrow_{#1}}
\def\na#1{{\not\Rightarrow}_{#1}}

\def\epso{\epsilon_0}
\def\C{\mathcal C}
\def\S{\mathcal S}
\def\N{\mathcal N}
\def\cong{\equiv}
\def\r{{\mathbf r}}
\def\bl{\bar\ell}

% tikz.
\def\smalldot#1{\draw[fill=black] (#1) node [inner sep=0.8pt,shape=circle,fill=black] {}}
\def\graydot#1{\draw[fill=gray] (#1) node [inner sep=1.3pt,shape=circle,fill=gray] {}}
\def\whitedot#1{\draw[fill=gray] (#1) node [inner sep=1.3pt,shape=circle,fill=white,draw=black] {}}
\tikzset{dartstyle/.style={fill=black,rotate=-90,inner sep=0.7pt,dart,shape border uses incircle}}
\tikzset{grayfatpath/.style={line width=1ex,line cap=round,line join=round,draw=gray}}

% parameters (x,y) center coords, theta vertex angle, rho radius in cm.
\def\pent#1#2#3#4{%
\draw[red] (#1,#2) + (#3:#4cm) -- + (#3+72:#4cm) -- +(#3+144:#4cm) -- +(#3+216:#4cm) -- + (#3+288:#4cm) -- cycle
}

% parameters (x,y) center coords, theta vertex angle.
\def\pen#1#2#3{%
\draw[red] (#1,#2) + (#3:1cm) -- + (#3+72:1cm) -- +(#3+144:1cm) -- +(#3+216:1cm) -- + (#3+288:1cm) -- cycle
}

\def\threepent#1#2#3#4#5#6#7#8#9{%
\pen{#1}{#2}{#3};
\pen{#4}{#5}{#6};
\pen{#7}{#8}{#9};
\draw[blue] (#1,#2) -- (#4,#5) -- (#7,#8) -- cycle
}

\def\threepentnoD#1#2#3#4#5#6#7#8#9{%
\pen{#1}{#2}{#3};
\pen{#4}{#5}{#6};
\pen{#7}{#8}{#9}
}


\centerline{\it We dedicate this article to W. Kuperberg.}

\section{Introduction}

Kuperberg and Kuperberg
 have conjectured that the
densest packing of congruent regular pentagons in the plane is
achieved by a double-lattice arrangement: verticals column of aligned
pentagons pointing upward, alternating with vertical columns of
aligned pentagons pointing downward (Figure~\ref{fig:double-lattice}) \cite{Kup}.  We call this the {\it Kuperberg
  conjecture,}  and we call this packing of pentagons the {\it Kuperberg packing}.
This packing has density
\[
\frac{5 - \sqrt{5}}3 \approx 0.921311.
\] % Checked 2016/2/18. In Vallentin, p.3.  calcs.ml.
Until now, the best known bound on the density of packings of regular
pentagons is $0.98103$ \cite{Val}.
Our research is a continuation of W. Kusner's thesis \cite{Kus}, which proves
the local optimality of the Kuperberg packing.
% 0.98103 page 27 of Vallentin. Checked 2016/2/18.


\tikzfig{double-lattice}{Kuperbergs' double-lattice packing of regular pentagons.  All figures show
pentagons in red and Delaunay triangles in blue.}
{
\pent{0.0}{0.0}{-90}{0.4};  % s = 0.4
\pent{0.0}{0.724}{-90}{0.4};  % (1+kappa)*s
\pent{0.0}{2*0.724}{-90}{0.4};  
\pent{0.57}{-0.3854}{90}{0.4};  % 3kappa*sigma*s,(3sigma^2-2)*s
\pent{0.57}{-0.3854+0.724}{90}{0.4}; 
\pent{0.57}{-0.3854+2*0.724}{90}{0.4};  
\pent{0.57}{-0.3854+3*0.724}{90}{0.4};  
\pent{-0.57}{-0.3854}{90}{0.4};  % 3kappa*sigma*s,(3sigma^2-2)*s
\pent{-0.57}{-0.3854+0.724}{90}{0.4}; 
\pent{-0.57}{-0.3854+2*0.724}{90}{0.4};  
\pent{-0.57}{-0.3854+3*0.724}{90}{0.4};  
\pent{2*0.57}{0.0}{-90}{0.4};  % s = 0.4
\pent{2*0.57}{0.724}{-90}{0.4};  % (1+kappa)*s
\pent{2*0.57}{2*0.724}{-90}{0.4}; 
\pent{3*0.57}{-0.3854}{90}{0.4};  % 3kappa*sigma*s,(3sigma^2-2)*s
\pent{3*0.57}{-0.3854+0.724}{90}{0.4}; 
\pent{3*0.57}{-0.3854+2*0.724}{90}{0.4};  
\pent{3*0.57}{-0.3854+3*0.724}{90}{0.4};   
}


%\section{Introduction}

This paper gives a computer-assisted proof of the upper
bound $0.9611$ on the density of a packing of
regular pentagon
(Corollary~\ref{lemma:961}).{\let\thefootnote\relax\footnote{The
    results of this paper were announced in 
Veszpr\'em, Hungary, July 2015.}}
More significantly, we set up the structures aimed
towards an eventual computer-assisted proof of the Kuperberg
conjecture.  Our strategy reduces the Kuperberg conjecture to 
area minimization problems that involve at most four acute Delaunay triangles.

\begin{remark}\label{rem:interval}
  The code for the computer-assisted proofs is written in Objective
  Caml.  There are several hundred lines of code, available for
  download from github \cite{Git}.  The computer calculations for this
  paper take less than one minute in total to run.  To control
  rounding errors on the computer, we use an interval arithmetic
  package for OCaml by Alliot and Gotteland, which runs on the Linux
  operating system and Intel processors \cite{All}.
\end{remark}


In this paper, we consider packings of congruent regular pentagons in
the Euclidean plane.  All pentagons are regular pentagons of fixed
circumradius $1$.  The inradius of each pentagon is $\kappa:= \cos
(\pi/5) = (1+\sqrt{5})/4 \approx 0.809$. We set $\sigma := \sin(\pi/5)
\approx 0.5878$.  The length of each pentagon edge is $2\sigma$.
% checked 2016/2/18 in calcs.ml.

All pentagon packings will be assumed to be saturated; that is, no
further regular pentagons can be added to the packing without overlap.
The assumption of saturation can be made without loss of generality,
because our ultimate aim is to give upper bounds on the density of
pentagon packings, and the saturation of a packing cannot decrease its
density.

We form the Delaunay triangulations of the pentagon packings.  The
vertices of the triangles are alway taken to be the centers of the
pentagons.  The saturation hypothesis implies that no Delaunay
triangle has circumradius greater than $2$.  This property of Delaunay
triangles is crucial.  It also follows that every edge of a Delaunay
triangle has length at most $4$.

For most of the paper, we consider a fixed saturated packing and its
Delaunay triangulation.  Generally, unless otherwise stated,
every triangle is a  Delaunay triangle.
Statements of lemmas and theorems implicitly
assume this fixed context.

Initially, pentagons in a packing play two roles: they constrain the
shapes of the Delaunay triangles and they carry mass for the density.
We prefer to we change our model slightly so pentagons are only used
to constrain the shapes of triangles.  We replace the mass of each
pentagon by a small, massive, circular disk (each of the same small
radius) centered at the center of the pentagon, and of uniform density
and the same total mass as the pentagon.  In this model, each Delaunay
triangle contains exactly one-half the mass of a pentagon.  By
distributing the pentagon mass uniformly among the Delaunay triangles,
we may replace density maximization with Delaunay triangle area
minimization.

We write 
\[
\aK := \frac{3}{2}{ \sigma \kappa(1+\kappa)} \approx 1.29036
\] % checked 2016/2/18 in calcs.ml
for the common area of every Delaunay triangle in the 
Kuperberg packing.  

\section{Clusters of Delaunay triangles}

The area of a Delaunay triangle in a saturated pentagon packing can be
smaller than $\aK$.  Our strategy for proving the Kuperberg conjecture
is to collect triangles into finite clusters such that the average
area over each cluster is at least $\aK$.  The clusters are defined by
various equivalence relations.  These equivalence relations, in turn,
are defined as the reflexive, symmetric, transitive closure of further
relations on the set of Delaunay triangles in a pentagon packing.

In more detail, below, we define various relations $(\ra{x})$, viewing
relations in the usual way as sets of ordered pairs.  For each
relation $(\ra{x})$, we write ${(\equiv_{x})}$ for the equivalence
relation obtained as the reflexive, symmetric, transitive closure of
$(\ra{x})$.  We call a corresponding equivalence classes $\C$ an
$x$-cluster, or simply a {\it cluster}, when the relation $(\ra{x})$
has been fixed.  In other words, each relation defines a directed
graph whose nodes are the Delaunay triangles of a pentagon packing,
with directed edges given as arrows $T_1\ra{x} T_2$.  An $x$-cluster
is the set of nodes in a connected component of the underlying
undirected graph.

We say that a Delaunay triangle (in any pentagon packing) is {\it
  subcritical} if its area is at most $\aK$.  We will obtain a lower
bound $a_0 := 1.237$ on the area of a nonobtuse Delaunay triangle
(Lemma~\ref{lemma:a0}).  This is a very good bound.  It is very close
to the numerically smallest achievable area, which is approximately
$1.23719$.\footnote{In the notation of the appendix, the numerical 
minimum is achieved by a pinwheel with parameters $\alpha=\beta=0$
and $x_\gamma\approx 0.16246$.}  
We write $\epso := \aK - a_0 \approx 0.05336$, for the
difference between the desired bound $\aK$ on averages of triangles
and the bound $a_0$ for a single nonobtuse triangle.
% Numerical page 8. of nonobtuse notes.
% numerical min checked 2016/2/18 in Mathematica.

We say that Delaunay triangle $T_1$ {\it attaches to Delaunay
  triangle} $T_2$, provided that the following condition holds: $T_2$
is the adjacent triangle to $T_1$ along the longest edge of $T_1$.
(If the triangle $T_1$ has more than one equally longest edge, fix
once and for all a choice among them, and use this choice to determine
the triangle that $T_1$ attaches to.  Thus, $T_1$ always attaches to
exactly one triangle $T_2$.  We can assume that the tie-breaking
choices are made according to a translation invariant rule.)  We write
$T_1 \Rightarrow T_2$ for the attachment relation.


We define the relation $(\ra{a})$ by $T_1 \ra{a} T_2$ iff
$T_1\Rightarrow T_2$ and $T_1$ is subcritical.

Let $\N$ be the set of pairs $(T_1,T_2)$ such that $T_1$ is nonobtuse
subcritical, $T_2$ is obtuse, and $T_1\Rightarrow T_2$.  Write $n_1(T)
= \card \{T_2\mid (T,T_2)\in \N\}$ and $n_2(T) = \card \{T_1\mid
(T_1,T)\in \N\}$.  Define
\[
b(T) := \op{area}(T) + \epso (n_1(T) - n_2(T)).
\]
We say that $T$ is {\it $b$-subcritical} if $b(T) \le \aK$.  We write
$T_1\ra{b} T_2$, if $T_1\Rightarrow T_2$ and $T_1$ is $b$-subcritical.
The equivalence classes of triangles under the corresponding
equivalence relation $(\equiv_b)$ are called {\it $b$-clusters}.

\begin{remark}
  Note that a triangle is not both nonobtuse and obtuse, so that at
  least one of the two terms $n_1(T), n_2(T)$ is zero for each $T$.
  Note also that $n_1(T)\le 1$, because each triangle attaches to
  exactly one other triangle.  Also, $n_2(T)\le 3$, because each
  attachment forms along an edge of the triangle $T$.
\end{remark}

We write $T_1\ra{c} T_2$ iff ($T_1 \ra{b} T_2$ and ($T_1$ is obtuse or
$T_2$ is not $b$-subcritical)).  A $c$-cluster $\C$ is an equivalence
class of Delaunay triangles under the relation $(\equiv_c)$.


\begin{lemma}\label{lemma:main}  
  Let $(\ra{x})$ be a subset of $(\ra{b})$.  Assume that $(\ra{x})$ is
  given by a translation-invariant rule.  Form $x$-clusters of
  Delaunay triangles under the equivalence relation $(\equiv_x)$. If
  every $x$-cluster $\C$ in every saturated packing of regular
  pentagons is finite, and if for some $a$ every $x$-cluster average
  satisfies
\begin{equation}\label{eqn:main}
\frac{\sum_{T\in \C} b(T)}{\card(\C)} \ge a,
\end{equation}
then the density of a packing of regular pentagons never exceeds 
\[
\frac{\area_P} {2 a},
\]
where $\area_P = 5\kappa\sigma$ is the area of a regular pentagon of
circumradius $1$.  In particular, if the inequality holds for $a=\aK$,
then the density never exceeds
\[
\frac{\area_P}{2 \aK} = \frac{5 - \sqrt{5}}{3},
\] % checked 2016/2/18 in calcs.ml
the density of the Kuperberg packing.
\end{lemma}

For any finite set $\C$ of triangles, we will call the inequality
(\ref{eqn:main}) with the constant $a=\aK$ the {\it main inequality}
(for $\C$).  We call the strict inequality,
\begin{equation}\label{eqn:strict-main}
\frac{\sum_{T\in \C} b(T)}{\card(\C)} > \aK,
\end{equation}
the {\it strict main inequality} (for $\C$).


\begin{proof} The maximum density can be obtained as the limit of the
  densities of a sequence of saturated periodic packings.  Thus, it is
  enough to consider the case when the packing is periodic.  A
  periodic packing descends to a packing on a flat torus
  $\ring{R}^2/\Lambda$, for some lattice $\Lambda$.  The rule defining
  $\N$ is translation invariant, and $\N$ descends to the torus.  On
  the torus, the set of pentagons, the set of triangles, and the set
  $\N$ are finite.  The equivalence relation $(\equiv_x)$ defining
  clusters is translation invariant, and each cluster is finite, so
  that no cluster contains both a triangle and a translate of the
  triangle under a nonzero element of $\Lambda$.  Thus, each cluster
  $\C$ in $\ring{R}^2$ maps bijectively to a cluster $\C$ in the flat
  torus.  Let $m$ be the number of pentagons in the torus.  By the
  Euler formula for a torus triangulation, the number of Delaunay
  triangles is $2m$.  We have
\[
\sum_{T} n_2(T) =  \sum_{T} n_1(T) = 
\card(\N).
\]
Thus, the terms in $b(T)$ involving $n_1(T)$ and $n_2(T)$ cancel:
\[
\area(\ring{R}^2/\Lambda) = \sum_T \area(T) = \sum_T b(T).
\]    
Let $\area_P$ be the area of a regular pentagon.  Making use of the
hypothesis of the lemma, we see that the density is
\[
\frac{m\, \area_P}{\sum_T \area(T)} 
=\frac{m\, \area_P}{\sum_T b(T)} \le\frac{m\, \area_P}{2 m\, a} 
= \frac{\area_P}{2\, a}.
\]
When $a=\aK$, the term on the right is the density of the Kuperberg
packing, as desired.
\end{proof}


\begin{conjecture}\label{conj:main}
  Let $\C$ be a $b$-cluster of Delaunay triangles in a saturated
  packing of regular pentagons.  Then $\C$ is finite and the average
  of $b(T)$ over the cluster is at least $\aK$.  That is, $\C$
  satisfies the main inequality.  Equality holds exactly when $\C$
  consists of two adjacent Delaunay triangles from the Kuperberg
  packing, attached along their common longest edge.
\end{conjecture}

Analyzing the proof of Lemma~\ref{lemma:main}, we see that for a
periodic packing, the maximum density is achieved exactly when each
 cluster in the packing gives exact equality in the main inequality.
Thus, the conjecture implies that the Kuperberg packing is the
unique periodic packing that achieves maximal density.


\section{Pentagons in contact}


\subsection{notation}

By way of general notation, we use uppercase $A,B,C,\ldots$ for
pentagons; $\v_A,\v_B,\ldots$ for the vertices of pentagons;
$\c_A,\c_B,\ldots$ for centers of pentagons; $\p,\q,\ldots$ for
general points in the plane; $\normo{\p}$ for the Euclidean norm;
$\norm{\c_A}{\c_B}$ for center-to-center distances;
$\alpha,\beta,\gamma,\phi,\psi,\ldots$ for angles; and $T_1,T_2,\ldots$
for Delaunay triangles.

We let $\eta(T) = \eta(d_1,d_2,d_3)$ be the circumradius of a triangle
$T$ with edge lengths $d_1,d_2$, and $d_3$.

Let $\angle(\p,\q,\r)$ be the angle at $\p$ of the triangle with
vertices $\p$, $\q$, and $\r$.  Let $\arc(d_1,d_2,d_3)$ be the angle
of a triangle (when it exists) with edge lengths $d_1$, $d_2$, and
$d_3$, where $d_3$ is the edge length of the edge opposite the
calculated angle.



% page 7 pdf obtuse.

We write $\area(T)= \area(d_1,d_2,d_3)$ for the area of triangle $T$
with edge lengths $d_1,d_2,d_3$.  We write
$\areta(d_1,d_2,\eta)$ for the area of a triangle with two edges
$d_1,d_2$ and circumradius $\eta$.  There are in general two
noncongruent triangles with data $d_1,d_2,\eta$.  We choose
$\areta(d_1,d_2,\eta)$ to give the area of the triangle such that its
third edge $d_3$ is as long as possible.

A Delaunay triangle has edge lengths at least $2\kappa$.  This minimum
Delaunay edge length is attained exactly when the the two
corresponding pentagons have an edge in common.  The following lemma
shows that under quite general conditions, we are justified in our
decision to choose $d_3$ as long as possible in the definition of the
function $\areta(d_1,d_2,\eta)$.  It is justified in the sense that
the other choice does not usually give a Delaunay triangle of a pentagon
packing, according to the following simple test.

\begin{lemma}\label{lemma:areta}  Let $d_1$, $d_2$
and  $\eta$ be positive real numbers.
Assume that $T$ and $T'$ are triangles with edge lengths
$d_1,d_2,d_3$ and $d_1,d_2,d_3'$,  and with the same circumradius
$\eta$. Assume $2\kappa\le d_1\le d_2$.  Set
 $\alpha = \arc(\eta,\eta,d_1)+\arc(\eta,\eta,2\kappa)$.
If $2\kappa \le d_3' < d_3$, then $\alpha < \pi$ and $2\eta\sin(\alpha/2) \le d_2$.
\end{lemma}

In contraposition, if $\alpha\ge\pi$ or if $d_2 < 2\eta\sin(\alpha/2)$,
then the triangle $T'\ne T$, with $d_3' < d_3$,
cannot satisfy the constraint $2\kappa\le d_3'$ of a  Delaunay triangle.


\begin{proof} See Figure \ref{fig:areta}.
Let $\p$, $\q$, and $\r$ (resp. $\p$, $\q$, and $\r'$) be the vertices of $T$
(resp. $T'$) on a common  circle, with
\[
\norm{\p}{\q} =d_1, \quad\norm{\p}{\r}=\norm{\p}{\r'}=d_2, \quad\text{and } 
\norm{\q}{\r'} = d_3' \le d_3=\norm{\q}{\r}.
\]
If $\alpha\ge\pi$, a point $\r'\ne \r$ satisfying the constraints does not exist.
Assume $\alpha < \pi$. 
As the figure indicates, $\norm{\p}{\r'}$ is minimized (as a function of $d_2$)
when $d_3' = 2\kappa$, the lower constraint.  Then
$d_2 = \norm{\p}{\r'}\ge 2\eta\sin(\alpha/2)$.
\end{proof}

\tikzfig{areta}{There can be two positions $\r,\r'$ on the circumcircle for
the third vertex of the triangle.}{
[scale=1]
\draw (0,0) circle (1cm);
\draw (1,0) node[anchor=west] {$\p$} --  (70:1cm) node[anchor=south] {$\q$};
\draw (1,0) -- node[above=1pt] {$d_2$} (130:1cm) node[anchor=south] {$\r'$};
\draw (1,0) -- node[below=1pt] {$d_2$} (-130:1cm) node[anchor=north] {$\r$};
\draw (130:1cm) -- (-130:1cm);
\draw[thin,gray] (1,0) -- (-1,0);
}



\subsection{triple contact}

In this subsection, we describe possible contacts between pentagons.

When two pentagons touch each other, some vertex of one touches an
edge of the other.  We call the pentagon with the vertex contact the
{\it pointer} pentagon, and the pentagon with the edge contact the
{\it receptor} pentagon (Figure~\ref{fig:receptor}).  We also call the
vertex in contact the {\it pointer vertex} of the pointer pentagon. There are
degenerate cases, when the contact set between two pentagons contains
of a vertex of both pentagons.  In these degenerate cases, the
designation of one pentagon as a pointer and the other as a receptor
is ambiguous.

\tikzfig{receptor}{Pointer and receptor pairs of pentagons.  In each pair, the pentagon on the
left can be considered a pointer pentagon, with receptor on the right.  The first pair is nondegenerate,
and the other two pairs are degenerate.}{
[scale=0.5]
\pent{0}{0}{0}{1};
\pent{1.8}{0}{0}{1};
\pent{5}{0}{0}{1};
\pent{7}{0}{180}{1};
\pent{10+0.176}{0.243}{0}{1};  % 0.3 {Cos [54^o],Sin[54^o]}
\pent{12-0.172}{-0.243}{180}{1};
}


% nonobtuse.
We consider a single Delaunay triangle and the three nonoverlapping
pentagons centered at the triangle's vertices
(Figure~\ref{fig:3C-type}).  We call such a configuration a {\it
  $P$-triangle}.  A $P$-triangle is determined up to congruence by six
parameters: the lengths of the edges of the Delaunay triangle and the
rotation angles of the regular pentagons.  When we refer to the area
or edges of a $P$-triangle, we mean the area or edges of the
underlying Delaunay triangle.  More generally, we allow $P$-triangles
to inherit properties from Delaunay triangles, such as obtuseness or
nonobtuseness, the relations $(\ra{x})$, clusters, and so forth. In
clusters of $P$-triangles it is to be understood that the pentagons
coincide at coincident vertices of the triangles.  When there is a
fixed backdrop of a Delaunay triangulation of a pentagon packing, it
is not necessary to make a careful distinction between a
$P$-triangle and its underlying Delaunay triangle.

We say that a $P$-triangle is $3C$ (triple contact),
if each of the three pentagons contacts the other two.

We may direct the edges of a $3C$ triangle by drawing an arrow from
the pointer pentagon to the receptor pentagon.  We may classify $3C$
triangles according to the types of triangles with directed edges.
There are two possibilities for the directed graph.
\begin{enumerate}
\item Some vertex is a source of two directed edges and another vertex
is the target of two directed edges  ($L$-junction, $T$-junction or
  $\Delta$-junction).
\item Every vertex is both a source and a target (pinwheel, pin-$T$).
\end{enumerate}

As indicated in parentheses, we have named each of the various contact
types.  An example of each of the contact types is shown in
Figure~\ref{fig:3C-type}.  An exact description of these contact types
appears later.  The name $L$-junction is suggested by the $L$-shaped
region bounded by the three pentagons.  Similarly, the name
$T$-junction is suggested by the $T$-shaped region bounded by the
three pentagons.  Similarly, for $\Delta$-junctions.  We will see
below that the types in the figure exhaust the geometric types of
$3C$ contact.


\tikzfig{3C-type}{Types of $3C$-contact from left-to-right:
a pinwheel, a pin-$T$ junction, a $\Delta$-junction, an $L$-junction, and a $T$-junction.}
{
[scale=0.6]
\threepentnoD{0.00}{0.00}{46.69}{0.82}{1.53}{218.09}{1.73}{0.00}{163.28};
\begin{scope}[xshift=4.5cm,yshift=1.5cm]
\threepentnoD{0.00}{0.00}{90}{1.40}{-1.377}{0}{-0.35}{-1.628}{-18.89};
\end{scope}
\begin{scope}[xshift=8cm]
\threepentnoD{0.00}{0.00}{-5.16}{0.99}{1.70}{235.38}{1.98}{0.00}{183.43};
\end{scope}
\begin{scope}[xshift=12cm]
\threepentnoD{0.00}{0.00}{80.86}{0.97}{1.58}{232.21}{1.82}{0.00}{223.24};
\end{scope}
\begin{scope}[xshift=16cm]
\threepentnoD{0.00}{0.00}{114.48}{0.90}{1.59}{237.18}{1.66}{0.00}{219.24};
\end{scope}
}

% formatpent3deg (delToPent3 (delAf pinwheeldelA (0.15) (0.15) (0.5)));;  
% formatpent3deg (delToPent3 (delAf pinwheeldelA (0.15) (0.15) (0.0)));;  

% formatpent3deg (delToPent3 (delAf deltajdelA (0.2) (0.15) (0.05)));;  
% formatpent3deg (delToPent3 (delAf ljdelA (0.5) (0.6) (0.90)));;  
% formatpent3deg (delToPent3 (delAf tjdelA (1.0) (1.2) (1.3*. ee)));;  

% graphics for pin-T brute forced in Mathematica.

%Note that the figures give two different geometric types
%associated to the combinatorial structure of a vertex that
%is the target of two edges.  

A {\it cloverleaf} arrangement is a $3C$ triangle that has a point at
which vertices from all three pentagons meet
(Figure~\ref{fig:clover}).  This is degenerate because this shared
vertex can be considered as a pointer or receptor.

\tikzfig{clover}{A cloverleaf (degenerate pinwheel)}
{
[scale=0.6]
\begin{scope}[xshift=4cm]
\threepentnoD{0.00}{0.00}{31.70}{0.76}{1.52}{203.11}{1.70}{0.00}{148.30};
\end{scope}
}


In general, in this paper, a {\it non-anomaly} lemma refers to a
geometrical lemma that shows that certain geometric configurations are
impossible.  Generally, it is obvious from the informal pictures that
various configurations cannot exist.  The non-anomaly lemmas then
translate the intuitive impossibilities into mathematically precise
statements.  We give a few non-anomaly lemmas as follows.
They are expressed as separation results, asserting that
two pentagons $A$ and $C$ do not touch.

\begin{lemma}\label{lemma:sep1} 
Let $T$ be a $3C$-triangle with pentagons $A$, $B$, and $C$ such that
$B$ is a pointer to both of the other pentagons $A$ and $C$.  
Assume that $T$ is not a cloverleaf.
Then the two pointer vertices $\v_B$ and $\v_B'$ are adjacent
vertices of $B$.
\end{lemma}

\begin{lemma}\label{lemma:sep2}  
Let $T$ be a $3C$-triangle with pentagons $A$, $B$, and $C$.
Suppose that pentagon
$A$ is a pointer to $B$ at $\v_A$ and that $B$ is a pointer to $C$
at $\v_B$.  Then on $B$, the vertex
$\v_B$ is not opposite to the edge of $B$ containing
$\v_A$.
\end{lemma}

\begin{lemma}\label{lemma:sep3} 
Let $T$ be a $3C$-triangle with pentagons $A$, $B$, and $C$ 
such that $B$ is a receptor of both of
  the other pentagons.  Then the two pointer vertices
  $\v_A$ and $\v_C$ lie on the same edge or adjacent pentagon edges of
  $B$. 
\end{lemma}



\tikzfig{sep}{A line through the center 
of the middle pentagon $B$
through one of its vertices 
separates the two extremal 
pentagons $A$ and $C$.}{
[scale=0.5]
\pen{0}{0}{90};
\pen{1.72}{0.56}{90};  % (1+kappa) {Cos pi/10 , sin pi/10}
\pen{-1.72}{0.56}{90};  
\draw (0,-1.5) -- (0,1.5);
\begin{scope}[xshift=6cm]
\pen{0}{0}{90};
\pen{1.72}{0.56}{90};  
\pen{-1.72}{-0.56}{90};  
\draw (0,-1.5) -- (0,1.5);
\end{scope}
\begin{scope}[xshift=12cm]
\pen{0}{0}{90};
\pen{1.72}{-0.56}{90};  
\pen{-1.72}{-0.56}{90};  
\draw (0,-1.5) -- (0,1.5);
\end{scope}
}


\begin{proof} The Lemmas~\ref{lemma:sep1}, \ref{lemma:sep2}, and
  \ref{lemma:sep3} can be proved in the same way.  In each case, we
  prove the contrapositive, assuming the negation of the geometric
  conclusion, and proving that the configuration is not $3C$.  We show
  that the configuration is not $3C$ by constructing a separating
  hyperplane between the pentagons $A$ and $C$.  In each case, the
  separating hyperplane is a line through the center of the middle
  pentagon $B$ and passing through a vertex $\v$ of that pentagon.  See
  Figure~\ref{fig:sep}.  In the case of Lemma~\ref{lemma:sep1}, there
  is a degenerate case of a cloverleaf, where all three pentagons meet
  at the vertex $\v$ on the separating line.
\end{proof}

\begin{definition}[$\Delta$]
  We say that a $3C$-triangle has type $\Delta$ if we are in the first
  case of Lemma~\ref{lemma:sep3} (the two pointer vertices $\v_A$ and
  $\v_C$ of $A$ and $C$ lie on the same edge of $B$) provided the line $\lambda$
  through that edge of $B$ separates $B$ from $A$ and $C$. (See
  Figure~\ref{fig:delta}.)
\end{definition}

\tikzfig{delta}{In type $\Delta$, a line separates pentagon 
$B$ from
the other two pentagons. The second figure (which is degenerate of type $L$) does not have
type $\Delta$.}{
[scale=0.5]
\pen{0}{0}{90};
\draw (0,0) node {$B$};
\draw (-2.0,-0.809) node[above] {$\lambda$} --  (2.0,-0.809);
\pen{-1.05}{-2*0.809}{-90};
\pen{1.9 - 1.05}{-2*0.809}{-90};
\begin{scope}[xshift=6cm]
\pen{0}{0}{90};
\draw (0,0) node {$B$};
\draw (-2.0,-0.809) node[above] {$\lambda$} --  (2.0,-0.809);
\pen{-0.428}{-2*0.809}{-90};
\pen{1.36}{ -1.44}{ 212.704};
\end{scope}
}


In type $\Delta$, say $A$ is a pointer into $C$ at $\v$.  Then $\v_A$
and $\v$ are the two endpoints of some edge of $A$.  Also, $\v_C$ and
$\v$ lie on the same edge of $C$.  If the line $\lambda$ does not
separate $B$ from $A$ and $C$, then $\v_C$ is a shared vertex of $B$
and $C$, and we have a degeneracy that can also be viewed as $\v_A$
and $\v_C$ on adjacent pentagon edges of $B$.  This case will be
classified as a degenerate $L$-junction below.

\begin{definition} Let $T$ be a $P$-triangle with pentagons $A$, $B$,
  and $C$.  Suppose that $A$ is in contact with $B$ and $C$ at points
  $\v_B$, $\v_C$ of $A$ that lie on adjacent edges of $A$.  Then we
  call the common endpoint $\v$ of those adjacent edges an {\it inner
    vertex} of $A$.  (The definition of inner vertex includes the
  degenerate case when a vertex of $A$ is $\v_B$ or $\v_C$.)
\end{definition}

\begin{definition} We say that a $3C$-triangle $T$ is type {\it
    pin}-$k$, for $k\in \{0,1,2,3\}$ if $A$ points into $B$, $B$
  points into $C$, $C$ points into $A$, and if there are exactly $k$
  pentagons (in $\{A,B,C\}$) that have an inner vertex.  We use {\it
    pinwheel} as a synonym for pin-$0$ and {\it pin-T} as a synonym
  for pin-$2$.
\end{definition}

%\begin{definition}
%  A {\it pinwheel} is a $3C$-triangle in which all three pentagons are
%  both pointers and receptors.  The following lemma identifies the
%  geometric structure of such a $P$-triangle.  The two contact ponts
%  on each pentagon are a vertex and another point on the same pentagon
%  edge.
%\end{definition}

% page 23-nonobtuse.
\begin{lemma} Let $T$ be a $3C$-triangle of type pin-$1$ with
  pentagons $A$, $B$, and $C$.  Then $T$ is degenerate of type
  $\Delta$.
% or type pin-$2$.
%  Assume that $C$ points to $A$ at $\v_{CA}$, that $A$ points to $B$
%  at $\v_{AB}$, and that $B$ points to $C$ at $\v_{BC}$.  Assume that
%  $T$ is not (degenerately) type $\Delta$ (Figure~\ref{fig:delta}).
%  Then $\v_{CA}$ and $\v_{BC}$ lie on the same edge of $C$.
\end{lemma}

\tikzfig{pinwheel}{A pin-$1$ configuration.}{
\begin{scope}[scale=1.0]
\pen{0}{0}{-90};
\pen{-0.974}{ 1.543}{ 32.70422};
\draw[red] ++ (-0.974,1.543) 
  ++ (0.842,0.532) node[black,anchor=west] {$\v_{CA}$} -- 
  ++ (115:0.2) -- ++ (-65:1.6) node[black,anchor=south west] {$\v_{AB}$}
   -- ++ (7:1.17);
\draw (0,0) node {$B$};
\draw (0,0) + (3*72-90:1) node[anchor=north west] {$\v_{BC}$};
\draw (-0.974,1.543) node {$C$};
\draw (1.1,1.543) node {$A$};
%\draw (0.57,0.809) -- (2.1,0.809) node[anchor=south] {$\gamma$};
\end{scope}
}

\begin{proof}  We draw a (distorted) picture of a
  pinwheel that violates the conclusion (Figure~\ref{fig:pinwheel}).
 %(There are
 % other configurations (and associated figure) that must be ruled out,
 % when additionally $\v_{AB}$ and $\v_{BC}$ lie on adjacent edge of
 % pentagon $B$ or when $\v_{CA}$ and $\v_{AB}$ lie on adjacent edges
 % of pentagon $A$.  We leave these cases to the reader's imagination.)
%XX
  We let $\p$ be the inner vertex of $C$; that is, the vertex that is
  interior to the triangle $(\v_{AB},\v_{BC},\v_{CA})$.  It is an
  endpoint of the edge of the pentagon $C$ containing $\v_{BC}$.  We have
\[
\angle (\p,\v_{BC},\v_{AB}) \le \pi,\quad \angle(\p,\v_{CA},v_{BC}) = 3\pi/5,\quad
\angle(\v_{AB},\v_{CA},\v_{BC})\le 2\pi/5.
\]
(The last inequality uses the fact that $T$ is not pin-$2$, so
that $\v_{AB}$ is not a vertex of $B$.)
We also have
\[
\angle(\p,\v_{CA},\v_{AB}) \ge 2\pi /5 \ge \angle(\v_{AB},\v_{CA},\v_{BC}) \ge \angle(\v_{AB},\v_{CA},\p).
\]
The law of sines applied to the triangle $(\p,\v_{CA},\v_{AB})$ then gives
\[
2\sigma =\norm{\v_{CA}}{\p}\le\norm{\v_{CA}}{\v_{AB}}\le 2\sigma.
\]
%\[
%2e\le \norm{\p}{\v_{CA}} \le \norm{\v_{CA}}{\v_{AB}} = 2e.
%\]
Thus, we have equality everywhere.  In particular, $\angle(\p,\v_{BC},\v_{AB})=\pi$.  This
has type $\Delta$.
\end{proof}

\begin{lemma}  The type pin-$3$ does not exist.  
\end{lemma}

\begin{proof} Suppose for a contradiction that a $P$-triangle $T$ of
  type pin-$3$ exists.  The region $X$ bounded by the three pentagons
  is a nonconvex star-shaped hexagon, with interior angles $\alpha'$,
  $7\pi/5$, $\beta'$, $7\pi/5$, $\gamma'$, and $7\pi/5$.  The vertices
  of $X$ with angles $7\pi/5$ are the inner vertices of the three
  pentagons of $T$.  The sum of the interior angles in a hexagon is $4\pi$:
\[
4\pi = \alpha'+\beta'+\gamma' + 3 (7\pi/5),
\]
which implies that $\alpha'+\beta'+\gamma' = -\pi/5$, which is
impossible.
\end{proof}

\begin{definition}[$T$ and $L$-junction]
  We say that a $3C$-triangle is a type $T$- or $L$-junction if it is
  not type $\Delta$ and if we are in the second case of
  Lemma~\ref{lemma:sep3} (the two pointer vertices $\v_A$ and $\v_C$
  lie on adjacent edges of $B$).  Say $A$ is a pointer into $C$ at
  $\v$.  We say that it has type {\it $L$-junction} if $\v_C$ and $\v$
  lie on the same pentagon edge of $C$, and otherwise we say it has
  type {\it $T$-junction}.
\end{definition}

We can be more precise about the structure of a $T$-junction.  In the
context of the definition, Lemma~\ref{lemma:sep2} implies that $\v$
and $\v_C$ lie on adjacent pentagon edges of $C$.

This completes the classification of $3C$-triangles.  Useful
coordinate systems for the various types can be found in the appendix
(Section~\ref{sec:appendix}).

\section{Delaunay triangle areas}

As an application of the classification from the previous section,
this section makes a computer calculation of a lower bound on the
longest edge length of a subcritical triangle.  We also obtain a lower
bound on the area of a nonobtuse Delaunay triangle.


%The next lemma gives a slight improvement on the edge length bound
%$\kappa\sqrt{8}\approx 2.288$.

\begin{lemma}\label{lemma:21} A nonobtuse subcritical triangle has edge lengths
  at most $2.1$.
\end{lemma}

\begin{proof} By the monotonicity of area as a function of edge length
  for nonobtuse triangles, a triangle with an edge length at least
  $2.1$ has area at least
\[
\area(2.1,2\kappa,2\kappa) > \aK,
\] % checked 2016/2/18 in Mathematica.
which is not subcritical.
\end{proof}

\begin{lemma}\label{lemma:right} 
  A nonobtuse subcritical triangle has edge length less than
  $\kappa\sqrt8$.  In particular, a right-angled Delaunay triangle is
  not subcritical.
\end{lemma}

\begin{proof}  
This is a corollary of the previous lemma, because $\kappa\sqrt8 >
2.1$.
%
%  The area of a nonobtuse triangle is monotonic increasing in its edge
%  lengths.  A nonobtuse Delaunay triangle with an edge of length at
%  least $\kappa\sqrt{8}$ has area at least that of the isosceles right
%  triangle
%\begin{equation}\label{eqn:right}
%\area(T) \ge \area(2 \kappa,2 \kappa,\kappa\sqrt{8}) = 2 \kappa ^2 \approx 1.30902 > \aK.
%\end{equation} % checked 2016/2/18 in calcs.ml
%This is not subcritical.  
\end{proof}


\begin{definition}
In a $P$-triangle, we say that a pentagon $A$ has {\it primary
  contact} if some of the following three conditions hold:
\begin{enumerate}
\item (slider contact) The pentagon $A$ and one $B$ of the other two pentagons
share a positive length edge segment;
\item (midpointer contact)  A vertex of one of the other two pentagons
is  the midpoint of one of the edges of the pentagon $A$; or
\item (double contact) The pentagon $A$ is in contact
with both of the other pentagons.
\end{enumerate}
\end{definition}

The next lemma is used to give area
estimates when an edge has length at most $\kappa\sqrt{8}$.

\begin{lemma}\label{lemma:primary-contact} 
  Let $A$ be a pentagon in a nonobtuse $P$-triangle.  Assume that the
  triangle edge opposite $\c_A$ has length at most $\kappa\sqrt{8}$.
  Then the $P$-triangle can be continuously deformed until $A$ is in
  primary contact, while preserving the following constraints: the
  deformation (1) maintains nonobtuseness, (2) is non-increasing in
  the edge lengths, and (3) keeps fixed the other two pentagons $B$
  and $C$.
\end{lemma}

\begin{proof} Fixing $B$ and $C$, we move $A$ to contract the two
  edges of the triangle at $\c_A$.  For a contradiction, assume that
  none of the primary contact conditions occur throughout the
  deformation.  Continue the contractions, until $A$ contacts another
  pentagon, then continue by rotating $A$ about its center $\c_A$ to
  break the contact and continue.  Eventually, the assumption of
  nonobtuseness must be violated.  However, this triangle cannot be
  obtuse at $\c_A$, because the triangle edge lengths are at least $2
  \kappa$, $2 \kappa$ with opposite edge at most
  $\kappa\sqrt{8}$. This is a contradiction.
\end{proof}

% page 30.
\begin{lemma} A subcritical $3C$-triangle does not have type $\Delta$.
  In fact, such a $P$-triangle $T$ has area greater than $1.5$.
\end{lemma}

\begin{proof} The proof is computer-assisted.  The $3C$-triangles of
  type $\Delta$ form a three-dimensional configuration space.  The
  appendix (Section~\ref{sec:appendix}) introduces good coordinate
  systems for each of the various $3C$-triangle types.  We make a
  computer calculation of the area of the Delaunay triangle as a
  function of these coordinates.  We use interval arithmetic to
  control the computer error.  The lemma follows from these computer
  calculations.
\end{proof}



\begin{lemma}\label{lemma:mid-172}  
  If a pentagon $A$ has midpointer contact with a pentagon $B$, then
  $\norm{\c_A}{\c_B} > 1.72$.
\end{lemma}

\begin{proof} Suppose a pointer vertex $\v_A$ of $A$ is the midpoint of an
  edge of pentagon $B$.  Rotating $A$ about the vertex $\v_A$, keeping
  $B$ fixed, we may decrease $\norm{\c_A}{\c_B}$ until $A$ and $B$
  have slider contact.  This determines the configuration of $A$ and
  $B$ up to rigid motion.  By the Pythagorean theorem, the
  distance between pentagon centers is
\[
\norm{\c_A}{\c_B} = \sqrt{(2\kappa)^2 + \sigma^2} \approx 1.72149 > 1.72.
\] % checked 2016/2/18 in calcs.ml
\end{proof}

\begin{lemma}\label{lemma:172}
  If every edge of a $P$-triangle $T$ is at most $1.72$, then the
  triangle is not subcritical.
\end{lemma}

\begin{proof} This is a computer-assisted proof.\footnote{The constant
    $1.72$ is nearly optimal.  For example, in the notation of the
    appendix, the pinwheel with parameters $\alpha=\beta=\pi/15$,
    $x_\gamma = 0.18$ is subcritical equilateral with edge lengths
    approximately $1.72256$.} Such a triangle is nonobtuse.  By
  Lemma~\ref{lemma:primary-contact}, we may deform $T$, decreasing its
  edge lengths and area, until each pentagon is in primary contact
  with the other two.  By the previous lemma, we may assume that the
  contact is not midpointer contact.  Thus, each pentagon has double
  contact or slider contact with the other pentagons.

  If the $P$-triangle does not have $3C$ contact, then obvious
  geometry forces one pentagon to have double contact and the other
  two pentagons to have slider contact (Figure~\ref{fig:172-slider}).
  The nonoverlapping of the pentagons forces one of slider contacts to
  be such that a sliding motion along the edges of contact decreases
  area and edge lengths of $T$.  Thus, the $P$-contact can be deformed
  until $3C$ contact results.

\tikzfig{172-slider}{We can slide pentagons to decrease
lengths and the area of the Delaunay triangle}{
[scale=0.5]
\threepent{0}{0}{-90}{0.63}{-1.54}{90}{-1.27}{ -1.07}{90};
}


Now assume that the $P$-triangle has $3C$ contact.
We have classified all $3C$ triangles.  We obtain the proof
by expressing each family of triangles in terms of explicit
coordinates from the appendix (Section~\ref{sec:appendix}) and
computing bounds on the areas and edge lengths of the triangles using
interval arithmetic.  The result follows.
\end{proof}

\begin{lemma}\label{lemma:2C} 
  Let $T$ be a subcritical nonobtuse $P$-triangle with pentagons $A$,
  $B$, and $C$.  Then fixing $B$ and $C$, we may deform $T$ by moving
  the third pentagon $A$, without increasing the area of $T$, until
  $A$ has double contact (with $B$ and $C$).
\end{lemma}

\begin{proof}
  By Lemma~\ref{lemma:right}, the edge lengths of $T$ are at most
  $\kappa\sqrt8$.  By Lemma~\ref{lemma:primary-contact}, we may assume
  that the pentagon $A$ has primary contact.  If the primary contact
  of a pentagon $A$ is slider contact, we may slide $A$ along the edge
  of contact in the direction to decrease the area of $T$ until it has
  double contact.  If the contact of $A$ is midpointer contact, then
  we may rotate $A$ about the point of contact with a second pentagon
  $B$, in the direction to decrease the area of $T$ until it has
  double contact.  These area-decreasing deformations never transform
  the nonobtuse subcritical triangle into a right triangle
  (Lemma~\ref{lemma:right}).
\end{proof}


\begin{lemma}\label{lemma:a0}  
A nonobtuse $P$-triangle $T$ has area at least $a_0$.
\end{lemma}

\begin{proof} This is a computer-assisted proof.  We may assume for a
  contradition that $T$ has area less than $a_0$.  In particular, it
  is subcritical.  By Lemma~\ref{lemma:2C}, we may assume that each
  pentagon has double contact with the other two, and that $T$ is
  $3C$.  We have classified all $3C$ triangles.  We obtain the proof
by expressing each family of triangles in terms of explicit
coordinates from the appendix (Section~\ref{sec:appendix}) and
computing bounds on the areas and edge lengths of the triangles using
interval arithmetic.  The result follows.
\end{proof}

\section{Obtuse Clusters}\label{sec:obtuse}


% Fig 1. 1-obtuse.pdf page 13b.

\begin{remark}\label{rem:delaunay}
  Recall that the Delaunay property implies that two adjacent Delaunay
  triangles $T_1$ and $T_2$ have the property that the angle
  $\alpha_1$ of $T_1$ and $\alpha_2$ of $T_2$ satisfy $\alpha_1 +
  \alpha_2\le \pi$, where $\alpha_i$ is the angle of $T_i$ that is not
  on the shared edge of $T_1$ and $T_2$.  In particular, {\it two
    obtuse Delaunay triangles cannot be joined along an edge that is
    the longest on both triangles.}  The extreme case
  $\alpha_1+\alpha_2=\pi$ corresponds to the degenerate situation
  where $T_1$ and $T_2$ form a cocircular quadrilateral. When
  cocircular, either diagonal of the quadrilateral gives an acceptable
  Delaunay triangulation.
\end{remark}

\subsection{structure of obtuse clusters}

We continue with our analysis of the clusters in a fixed saturated
packing of regular pentagons.  
%By an {\it obtuse cluster}, we mean any
%$b$-cluster of $P$-triangles that contains at least one obtuse
%triangle.

\begin{lemma}\label{lemma:notlong}  
Let $T_1$ and $T_2$ be $P$-triangles, such that $T_1\Rightarrow T_2$, where $T_1$ is nonobtuse subcritical and
  $T_2$ is obtuse.  Then the edge of attachment is not the longest
  edge of $T_2$.
\end{lemma}

\begin{proof} We have seen that each edge of a nonobtuse subcritical
  triangle has length less than $\kappa\sqrt8$ and that the longest edge of an
  obtuse Delaunay triangle has length at least $\kappa\sqrt{8}$.  These are
  incompatible conditions on a shared edge.
\end{proof}

\begin{lemma}  
  Let $(T_1,T_2)\in\N$.  Then the edge shared between the triangles is
  not the longest edge of $T_2$.  In particular, $n_2(T_2)\le 2$.
\end{lemma}

\begin{proof} If $(T_1,T_2)\in \N$, then $T_1$ and $T_2$ satisfy the
  assumptions of Lemma~\ref{lemma:notlong}.
\end{proof}

\begin{lemma}\label{lemma:no-ao} 
  There is no arrow $T_1 \ra{b} T_2$ with $T_1$ nonobtuse and $T_2$
  obtuse.
\end{lemma}

\begin{proof}  
  Assume for a contradiction that such a pair $(T_1,T_2)$ exists.

  We claim that $(T_1,T_2)\not\in \N$.  Otherwise, by
  Lemma~\ref{lemma:a0} we have
\[
b(T_1) = \area(T_1) + \epso > a_0 + \epso =  \aK.
\]
This implies that $T_1$ is not $b$-subcritical, contrary to the
definition of $b$-attachment $(\ra{b})$.

Since $(T_1,T_2)\not\in \N$, we have $b(T_1) = \area(T_1)$.  So $T_1$
is subcritical and nonobtuse, $T_2$ is obtuse, and $T_1\Rightarrow
T_2$.  By the definition of $\N$, this gives $(T_1,T_2)\in\N$, a
contradiction.
\end{proof}


\begin{lemma}\label{lemma:t2b}
  If $T_1$ is obtuse, and $T_1 \ra{b} T_2$, then $T_2$ is not
  $b$-subcritical.
\end{lemma}

\begin{proof} 
  If $T_1$ is obtuse, then its longest edge, which is shared with
  $T_2$, has length at least $\kappa\sqrt8$.

  Assume first that $T_2$ is nonobtuse.  Then $\area(T_2) > a_K$ by
  Lemma~\ref{lemma:right}.  Thus, $T_2$ is not $b$-subcritical:
\[
b(T_2) = \area(T_2) + \epso n_1(T_2) > \aK.
\]

Now assume that $T_2$ is obtuse.  By basic properties of Delaunay
triangles (Remark~\ref{rem:delaunay}), Delaunay triangles never join
along an edge that is the longest on both triangles.  Thus, $T_1$
attaches to $T_2$ along an edge adjacent to the obtuse angle of $T_2$.
To bound the area of $T_2$, we deform $T_2$ decreasing its area and
increasing its longest edge and its circumradius, until we obtain a
triangle of circumradius $\eta=2$, and shortest edges $2\kappa$ and
$\kappa\sqrt{8}$.  Then a numerical calculation gives
\[
b(T_2) = \area(T_2) - \epso n_2(T_2) \ge 
\areta(2\kappa,\kappa\sqrt{8},2) - 2\epso > \aK.
\] % checked 2016/2/18 in Mathematica. It holds by a margin 0.1856....
The use of the function $\areta$ is justified by
Lemma~\ref{lemma:areta} and the numerical estimate
\[
d_2 = \kappa\sqrt8 <  4 \sin(\arc(2,2,2\kappa)) = 2\eta\sin(\alpha/2).
\] % checked 2016/2/18 in Mathematica.
\end{proof}

\subsection{obtuse theorem}

Recall from above that we refine the $b$-clusters into $c$-clusters by
defining a relation $(\ra{c})$ and corresponding equivalence relation
$(\equiv_c)$, where $T_1 \ra{c} T_2$ iff ($T_1 \ra{b} T_2$ and ($T_1$
is obtuse or $T_2$ is not $b$-subcritical)).  Recall that a
$c$-cluster $\C$ is an equivalence class of Delaunay triangles under
the corresponding equivalence relation $(\equiv_c)$.

\begin{lemma}\label{lemma:c-weak}
  Let $\C$ be a $c$-cluster that does not contain an obtuse triangle.
  Then $\C$ is finite of cardinality at most $4$,
  and we have the following weak form of the main
  inequality on $\C$:
\[
\frac{\sum_{T\in \C} b(T)}{\card(\C)} > a_0,
\]
\end{lemma}

\begin{proof}  
  We claim that every $c$-cluster $\C$ that does not contain
 an obtuse triangle is finite.
  In fact, on such a cluster $T_1\ra{c} T_2$ implies that $T_1$ is
  $b$-subcritical nonobtuse and $T_2$ is nonobtuse but not
  $b$-subcritical.  The $b$-criticality conditions on $T_1$ and $T_2$
  are incompatible, so that there are no paths of length greater than
  $1$ in the directed graph of $(\ra{c})$.  Also, the out-degree is at
  most $1$, and in-degree at most $3$ at each node of the directed
  graph.   The bound on cardinality follows.

  If $T$ is any nonobtuse Delaunay triangle, then by the definition of
  $b(T)$ and by Lemma~\ref{lemma:a0}, we have
\[
b(T) = \area(T)+\epso n_1(T) \ge \area(T) > a_0.
\]
Thus, the average is also greater than $a_0$.
\end{proof}

The following theorem and its corollary are the main results of this
section.  We will prove the theorem after presenting the corollary.

\begin{theorem}\label{lemma:obtuse}  
  Let $\C$ be any $c$-cluster that contains an obtuse triangle.  Then
  $\C$ is finite, and the strict main inequality
  (\ref{eqn:strict-main}) holds for $\C$.
\end{theorem}

\begin{corollary}\label{lemma:961}  
  The density of a packing of regular pentagons is at most
\[
\frac{\area_P}{2a_0} < 0.9611,
\] % checked calcs.ml 2016/2/18.
where $\area_P$ is the area of a regular pentagon of circumradius $1$.
\end{corollary}

\begin{proof}[Proof (corollary).] 
  Combining the Theorem~\ref{lemma:obtuse} with
  Lemma~\ref{lemma:c-weak}, 
  each $c$-cluster $\C$ satisfies
  the main inequality with constant $a=a_0$.  By
  Lemma~\ref{lemma:main}, this gives the density bound.
\end{proof}

We remark that $b$-clusters will be used later more directly in our
strategy to prove the Kuperberg conjecture.  We now turn to the proof
of the main result in this section.

\begin{proof}[Proof (theorem).] 
  The proof involves several relatively simple cases.  
  We recall that each Delaunay triangle has edge lengths at
  least $2\kappa$ and circumradius at most $2$.

  If the $c$-cluster is a singleton $\{T_1\}$, where $T_1$ is obtuse,
  then there does not exist an arrow 
$T_1\ra{c} T_2$.  This implies that the triangle $T_1$ is
  not $b$-subcritical, and $b(T_1) > \aK$.  This completes this case.
  We now assume that $\C$ is not a singleton.

  Let $\C$ be a $c$-cluster with an obtuse triangle.  Then there
  exists an arrow $T_1 \ra{c} T_0$ in $\C$ with $T_1$ or $T_0$ obtuse.
  By Lemma~\ref{lemma:no-ao}, $T_1$ is obtuse.  By the definition of
  $(\ra{c})$, the triangle $T_1$ is $b$-subcritical.  By
  Lemma~\ref{lemma:t2b}, $T_0$ is not $b$-subcritical, so there is no
  further arrow $T_0\ra{c} T$.

  Let $\C_0 = \{ T \mid T\ra{c} T_0\}$.  We claim that for all $T\in
  \C_0$, there does not exist an arrow $T' \ra{c} T$.  Otherwise, by
  the definition of $(\ra{c})$ applied to the arrows $T'\ra{c} T\ra{c}
  T_0$, we have that $T$ is $b$-subcritical and $T'$ is obtuse.  Then
  by Lemma~\ref{lemma:t2b} applied to the arrow $T'\ra{c} T$, the
  triangle $T$ is not $b$-subcritical, which is a contradiction.

  Hence, all arrows in the $c$-cluster have been determined, and $\C =
  \C_0 \cup \{T_0\}$.  The cluster is finite, consisting of a single
  triangle $T_0$, together with all the arrows $T\ra{c} T_0$ into it.
  Before continuing with the proof of the theorem, we need the
  following lemma.

\begin{lemma} Let $\C$ be any $c$-cluster that contains an obtuse
  triangle.  Write $\C = \C_0 \cup \{T_0\}$ as above.  Let
  $\C_0'\subset \C_0$ be the obtuse triangles in $\C_0$ and let $\C' =
  \C_0' \cup \{T_0\}$.  Set
\[
\S = \{T \mid T \text{ is nonobtuse subcritical such that } \exists
T'\in \C' \text{ with } T\rat T'\}.
\]
If
\begin{equation}\label{eqn:S}
\sum_{T\in \C'} \area(T) > a_K \card(\C') + \epso \card(\S),
\end{equation}
then the strict main inequality holds for $\C$.
\end{lemma}

\begin{proof}
  We note that $\C_0' \cap \S$ is empty, because triangles in $\C_0'$
  are obtuse and triangles in $\S$ are not.
  We break the proof into two cases depending on whether $T_0$ is
  obtuse.

  Assume first that $T_0$ is obtuse.  If $T\in \C_0\setminus \C_0'$,
  then $T$ is nonobtuse.  By the construction of $\C_0$, we have
  $T\ra{c} T_0$, so $T\ra{b} T_0$.  By Lemma~\ref{lemma:no-ao}, $T$
  does not exist.  Hence $\C_0' = \C_0$ and $\C'=\C$.  That is, every
  triangle in $\C$ is obtuse.  Note that
\[
\sum_{T\in \C} n_2(T) = \card(S).
\]
Then by the assumption (\ref{eqn:S}), we have
\begin{align*}
\sum_{T\in \C} b(T) &=\sum_{T\in \C} (\area(T) - \epso n_2(T)) \\
&= \sum_{T\in \C} \area(T) - \epso \card(S) \\
&> a_K \card(\C).
\end{align*}
This is the strict main inequality.

Finally, assume that $T_0$ is nonobtuse.  We have
\[
b(T_0) = \area(T_0) + \epso n_1(T_0) \ge \area(T_0).
\]

We claim that 
\[
\card(\S) \ge \card(\C_0 \setminus \C_0') + \card(S\setminus \C_0).
\]
In fact, it is enough to show that if $T\in \C_0\setminus \C_0'$, then
$T\in \S$.  Any such $T$ is nonobtuse and $T\ra{c}T_0$.  Expanding
the definition of $(\ra{c})$, we get $T\rat T_0$, and $T$ is
$b$-subcritical, $(T,T_0)\not\in\N$, so $b(T) = \area(T)$, and $T$ is
subcritical.   This gives $T\in \S$.

We have
\[
\sum_{\C_0'} n_2(T) = \card\{T'\in\S \mid T' \text{ nonobtuse
  subcritical, and } T'\rat T\in \C_0' \} \le \card(\S\setminus \C_0 ),
\]

Using (\ref{eqn:S}) and Lemma~\ref{lemma:a0}, we have
\begin{align*}
\sum_{T\in \C} b(T) &= \sum_{\C_0\setminus \C_0'} b(T) + b(T_0) +
\sum_{\C_0'} b(T) \\
&\ge \sum_{\C_0\setminus \C_0'} \area(T) + \area(T_0) +
\sum_{\C_0'} (\area(T) - \epso n_2(T)) \\
&\ge \card({\C_0\setminus \C_0'}) a_0 + (\area(T_0) +
\sum_{\C_0'} \area(T)) - \epso \card(\S\setminus \C_0)\\
&> \card({\C_0\setminus \C_0'}) (\aK  - \epso) + (\aK \card(\C') +
\epso \card(\S)) - \epso \card(\S\setminus \C_0)\\
&\ge a_K \card(\C).
\end{align*}
This is the strict main inequality for $\C$.
\end{proof}

\begin{remark} If $T'\in\S$, then $T'$ is nonobtuse subcritical.
  Also, $T'\rat T$, for some $T\in \C'$.  By construction, $T'$
  attaches to $T$ along the longest edge of $T'$.  By
  Lemmas~\ref{lemma:21} and \ref{lemma:172}, the longest edge has
  length in $[1.72,2.1]$.  Thus, we can get an upper bound on
  $\card(\S)$ by counting the number $n'$ of edges of $\C'$ in this
  range.  Note that the longest edge of each $T\in \C_0'$ is at least
  $\kappa\sqrt8 > 2.1$, hence is not in this range.  Thus, to prove the
  strict main inequality, it is enough to show that
\begin{equation}\label{eqn:n'}
\sum_{T\in \C'} \area(T) > a_K \card(\C') + n'\epso.
\end{equation}
\end{remark}

We now continue with the proof of Theorem~\ref{lemma:obtuse}.  
Without loss of generality, we may assume that
$\C'$ is not empty.  Otherwise, the unique obtuse
triangle in $\C$ is $T_0$.  The proof of the preceeding lemma shows
that when $T_0$ is obtuse, then $\C_0 \setminus \C_0' = \emptyset$.
Thus, $\C = \{T_0\}$, which is a case that has already been treated.

We break the proof into six cases depending on whether $T_0$ is
nonobtuse, and depending on $\card(\C')\in \{1,2,3\}$.
In each case we prove inequality (\ref{eqn:n'}).


{\it Case 1. The triangle $T_0$ is a nonobtuse triangle, and
  $\C'=\{T_0,T_1\}$.}  The triangle $T_0$ has a vertex $\v$ that is
not shared with $T_1$.  By the Delaunay property, $\v$ lies outside
the circumcircle of $T_1$.  The triangles $T_0$ and $T_1$ form a
quadrilateral $Q$ whose diagonal is the shared edge of $T_0$ and
$T_1$.  We deform the quadrilateral $Q$ to decrease its area while
maintaining the following constraints:
\begin{enumerate}
\item The vertex $\v$ lies on or outside the circumcircle of
  $T_1$. The circumradius of $T_1$ is at most $2$.
\item The edge length of the $i$th edge of $Q$ is at least
  $d_i\in\{2\kappa,1.72\}$,
where $n'$ is the number of $d_i$ that equal $1.72$; and
\item $T_1$ is not acute.
\end{enumerate}
We drop all other constraints as we deform. (In particular, we do not
enforce the nonoverlapping of pentagons in the $P$-triangles.)
We continue to deform $Q$ until one of the following two subcases hold:
\begin{enumerate}
\item $Q$ is cocircular; or
\item For all $i=1,2,3,4$, the $i$th edge of $Q$ has reached its lower
  bound $d_i$.
\end{enumerate}

In the first subcase (cocircularity), we drop the third constraint
(non-acuteness) and continue area decreasing deformations for $Q$
under the constraint of a fixed circumcircle.  We note that the area of
a cocircular quadrilateral $Q$ depends only on the lengths of the
edges and not on their cyclic order on $Q$.  We may thus rearrange the
edge order as we deform.  For a given circumcircle, the area is
minimized when three of the edges attain their lower bound $d_i$.  By
suitable reordering of the edges, we may assume that $Q$ is an
isosceles trapezoid and that the ``free'' edge is parallel to and
longer than its opposite edge on $Q$.  For such $Q$, the area as a
function of the circumradius is concave, so that the minimum occurs
when the circumradius is as small (that is, all edges attain the
minimum $d_i$) or as large (that is, $\eta(Q)=2$) as possible.
When $\eta(Q)=2$, we relax the edge lengths constraints further
to allow three edges to have length $2\kappa$.
Explicit numerical calculations in these two extremal configurations
show that the inequality (\ref{eqn:n'}) is satisfied (for each $n'$).

In the second subcase (every edge attains its minimal length $d_i$), the
four edge lengths are fixed.  The area of $Q$ is a concave function of
the length of the diagonal.  We thus minimize the area of $Q$ when the
diagonal is as small as possible (that is, $T_1$ is a right triangle
-- when this satisfies the other constraints) or as large as possible
(that is, $Q$ is cocircular).  The cocircular case has already been
considered.  Explicit numerical calculations of $Q$ when $T_1$ is
right gives the inequality (\ref{eqn:n'}) in each case.
% checked 2016/2/18 in Mathematica.




{\it Case 2. The triangle $T_0$ is a nonobtuse triangle, and $\C'=\{T_0,T_1,T_1'\}$.}
The long edges of the obtuse triangles $T_1$ and $T_1'$ have length at least
$\kappa\sqrt{8}$.

We consider a subcase where $\eta(T_1)\le 1.7$ and $\eta(T_1') \le
1.7$.  Then calculations based on the monotonicity of the area
functions give
\[
\area(T_0) \ge \area(d,\kappa\sqrt{8},\kappa\sqrt{8}) > 
   \begin{cases}1.73,&\text{if } d \ge 2\kappa\\ 
     1.73+\epso, & \text{if } d \ge 1.72 \end{cases}.
\]
The areas of $T=T_1,T_1'$ are at least
\begin{equation}\label{eqn:173}
\area(T) \ge \areta(d,2\kappa,1.7) >
   \begin{cases}1.08,&\text{if } d \ge 2\kappa\\
     1.08+2\epso, & \text{if } d \ge 1.72 \end{cases}.
\end{equation}
These bounds give
\[
\area(T_1) + \area(T_1') + \area(T_0) > 1.73 + 2(1.08) + n'\epso > 
3 \aK + n' \epso.
\]
% checked 2016/2/18 in Mathematica.

By symmetry, we may now assume that $\eta(T_1) \ge 1.7$.  By the
Delaunay condition, since $T_1$ is obtuse and $T_0$ is nonobtuse, this
forces $\eta(T_0)\ge 1.7$.  We minimize the area of $T_0$ subject to
the constraints that its circumradius is at least $1.7$, that it is
nonobtuse, and its edge lengths are at least $\kappa\sqrt{8}$,
$\kappa\sqrt{8}$, and $2\kappa$.  If two edges are $2\kappa$,
$\kappa\sqrt{8}$, then $T_0$ is obtuse, so the binding constraints for
the optimization become $\eta(T_0)=1.7$, $2\kappa$ edge length, and a
right triangle.  Such a triangle has area at least 
\[
2\kappa\sqrt{\eta^2 - \kappa^2} \ge 2.41.
\]  
The areas of $T_1$ and $T_1'$ are at least
\begin{equation}\label{eqn:968}
\areta(2\kappa,2\kappa,2) > 0.968.
\end{equation}
This completes this case:
\[
\area(T_1) + \area(T_1') + \area(T_0) 
>
2(0.968) + 2.41 > 3\aK + 5 \epso \ge 3\aK + n'\epso.
\] %checked 2016/2/18 in Mathematica.

{\it Case 3. The triangle $T_0$ is a nonobtuse triangle, and $C'=\{T_0,T_1,T_1',T_1''\}$.}

This case is almost identical to case 2.  We use the same bounds
(Equations (\ref{eqn:173}) and (\ref{eqn:968})) on $\area(T)$ as
before, for $T = T_1, T_1', T_1''$.  We can improve the bound on the
area of $T_0$:
\[
\area(T_0) \ge \area(\kappa\sqrt{8},\kappa\sqrt{8},\kappa\sqrt{8}) > 2.2668.
\]
Moreover, in the subcase where $\eta(T_0)\ge 1.7$, we have
(even after dropping the nonobtuseness constraint):
\[
\area(T_0) \ge \areta(\kappa\sqrt{8},\kappa\sqrt{8},1.7) > 2.6.
\]
In this case, $n'\le 6$.  Proceeding as before, we get
\[
\area(T_0)  + \area(T_1) + \area(T_1') + \area(T_1'') > 
\begin{cases}
2.2668 + 3 (1.08) \\
2.6 + 3(0.968)
\end{cases}
> 4\aK + n' \epso.
\] %checked 2016/2/18 in Mathematica.


This completes the proof for cases involving a nonobtuse triangle
$T_0$.  In the remaining cases, we assume that $T_0$ is obtuse.  In
fact, every triangle in $\C'$ is obtuse.

{\it Case 4. The triangle $T_0$ is an obtuse triangle, and $\C'=\{T_0,T_1\}$.}  

In this case, $n'\le 3$.  It will not be necessary to create subcases
according to whether short edges are at least $2\kappa$ or $1.72$.  We
will show that we can relax the lower bound on the edge to $2\kappa$
and still obtain the bound (\ref{eqn:n'}).

We minimize area by flattening $T_0$ by stretching its long edge until
$\eta(T_0)=2$.  We further decrease area, keeping the circumradius
fixed, by contracting the shorter edge not shared with $T_1$, until
the edge has length $2\kappa$.

Next continue to minimize area by contracting an edge of $T_1$,
keeping its circumradius fixed, until an edge has length $2\kappa$.
Then, allowing the circumradius of $T_1$ to increase, we contine until
both shorter edges have length $2\kappa$ or until the circumradius
reaches $2$.

First assume that both shorter edges of $T_1$ have length $2\kappa$.
We have reduced to a one-parameter family of quadrilaterals.  We can
choose the parameter to be the length $x$ of the diagonal, the common
edge of $T_1$ and $T_0$.  The parameter $x$ ranges between
$\kappa\sqrt{8}$ and $x_{\max}\approx 2.9594$, determined by the
condition $\eta(2\kappa,2\kappa,x_{\max}) = 2$.  We check numerically
that
\[
\area(T_1) + \area(T_0) \ge \area(2\kappa,2\kappa,x) +
\areta(x,2\kappa,2) > 2\aK + 3 \epso \ge 2\aK + n'\epso.
\] % checked 2016/2/18 in Mathematica.

Next, assume the circumradius of $\eta(T_1)$ reaches $2$, then we
have a cocircular quadrilateral that can be treated as in Case 1.  In
particular, the minimizing cocircular quadrilateral has three edges of
length $2\kappa$ and circumradius $2$.  This is precisely the limiting
case of a quadrilateral with diagonal $x_{\max}$ considered above.

This completes the argument in this case.

{\it Case 5. The triangle $T_0$ is an obtuse triangle, and $\C'=\{T_0,T_1,T_1'\}$.}  

By Remark~\ref{rem:delaunay}, there is no arrow $T \rat T_0$ in $\C'$
such that the shared edge is the long edge of $T_0$.  In particular,
there cannot exist (Case 6) with $\C'=\{T_0,T_1,T_1',T_1''\}$ with
every triangle obtuse.  Thus, Case 5 is the last case to be
considered.

We have $n'\le 4$.
The area of $T_0$ is at least
$\areta(\kappa\sqrt{8},\kappa\sqrt{8},2) > 2.45$.  
Using our earlier estimates (\ref{eqn:968})
for $\area(T)$, for $T=T_1,T_1'$, we
have
\[
\area(T_1) + \area(T_1') + \area(T_0) > 2 (0.968) + 2.45 > 3\aK + n'\epso.
\] % checked 2016/2/18 in Mathematica.
%If some shorter edge of $T_1$ or $T_1'$ is at least $1.72$, then we
%again use the estimates from earlier cases to
%obtain % specify earlier cases XX
%\[
%\area(T_1) + \area(T_1') + \area(T_0) >  3\aK  + n'\epso.
%\]
This completes the proof of the theorem.
\end{proof}

\section{Conjectures}

In this section, we give refinements of Conjecture~\ref{conj:main}.  A
proof of Conjecture \ref{conj:dimer} would represent a strengthening
of the local optimality results in~\cite{Kus}.

\begin{definition}
  We define a {\it dimer} to be a pair $\{T_1,T_2\}$ such that $T_1$
  and $T_2$ are both nonobtuse subcritical $P$-triangles, and $T_1\rat
  T_2$ or $T_2\rat T_1$.
\end{definition}

\begin{conjecture}\label{conj:dimer}[Dimer conjecture]
  If $\{T_1,T_2\}$ is a dimer, then $T_1$ and $T_2$ are joined along
  their common (uniquely) longest edge. Furthermore, the
$P$-triangles, $T_1$ and $T_2$, are
  congruent to those in the Kuperberg packing.
\end{conjecture}

The dimer conjecture implies in particular that $\area(T_1)=\area(T_2)
= \aK$.  Also, $T_1\rat T_2$ and $T_2\rat T_1$.  We note that dimer
conjecture would allow us to simplify a number of the arguments in
earlier sections.  The following lemmas give a number of other
consequences of the dimer conjecture.

\begin{lemma}\label{lemma:dimer}
  The dimer conjecture implies that there does not exist a pair of
  arrows $T_1 \ra{b} T_2 \ra{b} T_3$, with $T_1$ nonobtuse, unless
  $T_1= T_3$ and $\{T_1,T_2\}$ is a dimer.
\end{lemma}

\begin{proof} Assume for a contradiction that the arrows exist.  By
  Lemma~\ref{lemma:no-ao}, $T_2$ and $T_3$ are nonobtuse.  By the
  definition of $b$ and $\N$, we have $b(T_1) = \area(T_1)$ and $b(T_2) =
  \area(T_2)$.  Hence $T_1$ and $T_2$ are nonobtuse subcritical.  By
  the dimer conjecture, $T_2\rat T_1$.  Thus, an arrow $T_2\rat T_3\ne
  T_1$ does not exist.
\end{proof}

\begin{lemma}\label{lemma:dimer-obtuse} 
Assume the dimer conjecture.  There does not exist an
arrow $T\rat T_1$, where $T$ is obtuse, and $T_1$ is part of a dimer.
\end{lemma}


\begin{proof}  
  If $T$ is obtuse, its longest edge has length greater than
  $\kappa\sqrt8$, but the edges of $T_1$ in a dimer are shorter than
  this, so we do not have the given arrow $T\rat T_1$.
\end{proof}

\begin{lemma}
  Assume the dimer conjecture.  There does not exist an arrow $T\rat
  T_1$, where $T$ is nonobtuse subcritical, and $T_1$ is part of a
  dimer, unless the dimer pair is $\{T,T_1\}$.
\end{lemma}


\begin{proof}
  With $T$ nonobtuse subcritical, the dimer conjecture implies that
  $\{T,T_1\}$ is a dimer.
\end{proof}

\begin{lemma}\label{lemma:dimer-cluster}
Assume the dimer conjecture.  A dimer $\{T_1,T_2\}$ is an $a$-cluster and
a $b$-cluster.
\end{lemma}

\begin{proof}  
We have $T_1 \ra{a} T_2$ and $T_1 \ra{b} T_2$.

Assume $T\rat T_1$, where $T\ne T_2$.  Then by
Lemma~\ref{lemma:dimer-obtuse}, we see that $T$ is nonobtuse.  By the
dimer conjecture, $T$ is not subcritical.  Also $n_1(T) = 0$ and
$b(T)=\area(T)$, so $T$ is not $b$-subcritical.  So
\[
T\na{a} T_1,\quad\text{and}\quad T\na{b} T_1.
\]
The result follows.
\end{proof}



\begin{lemma}\label{lemma:b=c}  
The dimer conjecture implies that if $\C$ is a $b$-cluster
that contains an obtuse triangle $T$, then $(\ra{c})$ restricted to $\C$ is
equal to $(\ra{b})$ restricted to $\C$.
\end{lemma}

\begin{proof} By definition, we have $(\ra{c})\subseteq (\ra{b})$.
  Suppose that $T_1\ra{b} T_2$ but not $T_1 \ra{c} T_2$, for some
  $T_1,T_2\in\C$.  By the definition of $(\ra{c})$, the triangle $T_1$
  is nonobtuse and $T_2$ is $b$-subcritical.

  Let $T_3$ be such that $T_2\rat T_3$.  Then $T_1\ra{b} T_2\ra{b}
  T_3$.  By Lemma~\ref{lemma:dimer}, $T_1=T_3$ and $\{T_1,T_2\}$ is a
  dimer.  By Lemma~\ref{lemma:dimer-cluster}, $\{T_1,T_2\}$ is a full
  $b$-cluster and contains no obtuse triangles, which is contrary to
  assumption.
\end{proof}

\begin{lemma}\label{lemma:b-structure} 
  Assume the dimer conjecture.  Then every $b$-cluster $\C$ is finite.
  Specifically, one of the following holds:
\begin{enumerate}
\item $\C$ is a singleton $\{T_0\}$, where $T_0$ is not
  $b$-subcritical; or
\item $\card(\C)>1$ and there exists $T_0\in \C$ such that
\[
\C = \{T \mid T\ra{b} T_0\} \cup \{T_0\}.
\]
\end{enumerate}
\end{lemma}

\begin{proof} If $\C$ contains an obtuse triangle, then the
  $b$-cluster $\C$ is also a $c$-cluster by Lemma~\ref{lemma:b=c}.
  These structure results for $c$-clusters were obtained in
  Section~\ref{sec:obtuse} in the proof of Theorem~\ref{lemma:obtuse}.

  Now assume that every triangle in $\C$ is nonobtuse.  We consider
  three cases depending on whether $\C$ is a singleton, a dimer, or
  general.

  Assume first that $\C = \{T_0\}$, a singleton.  Then $T_0\rat T$ for
  some triangle $T$.  Because $T$ is not in the $b$-cluster of $C$, we
  do not have $T_0\ra{b} T$, which implies that $T_0$ is not
  $b$-subcritical.

  Next, assume that $\C = \{T_0,T_1\}$ a dimer.  Then it is easy to
  see that the conclusions of the lemma hold.

  Finally assume that $\C$ is not a singleton nor a dimer.  We have
  $T'\ra{b} T_0$, for some $T',T_0\in \C$.  Set
\[
\C_1 = \{T \mid T\ra{b} T_0\} \cup \{T_0\}.
\]
By the definition of a $b$-cluster, we have $\C\supseteq \C_1$.  By
Lemma~\ref{lemma:dimer} there are no other arrows $(\ra{b})$ between
triangles in $\C_1$, so $\C_1 = \C$.
\end{proof}

\begin{lemma}\label{lemma:obtuse-main} 
  The dimer conjecture implies that every $b$-cluster with an obtuse
  triangle satisfies the strict main inequality.
\end{lemma}

\begin{proof} By Theorem~\ref{lemma:obtuse}, the strict main
  inequality holds for a $c$-cluster with an obtuse triangle.  By the
  lemma~\ref{lemma:b=c}, this $c$-cluster coincides with the
  $b$-cluster.
\end{proof}

Our strategy of proof for the Kuperberg conjecture is to prove the
main inequality for every $b$-cluster.  The next lemma places strong
restrictions on the structure of a possible counterexample to the
Kuperberg conjecture.

\begin{lemma}\label{lemma:props}  
  Assume the dimer conjecture.  Let $\C$ be a $b$-cluster that
  violates the strict main inequality.  Then
\begin{enumerate}
\item All triangles in $\C$ are nonobtuse.
%\item $T\ra{b} T'$ iff $T \ra{a} T'$ if $T\in C$ or if $T'\in\C$.
%\item $\C$ is an $a$-cluster.
\item $b(T) \ge \area(T)$ for all $T\in \C$.  
\item There is $T_0\in \C$ such that $\C = \{T_0\}\cup \{T \mid
  T\ra{a} T_0\}$.
\item The cardinality of $\C$ is $2$, $3$, or $4$.
\item All edges of $\C$ have length at most $2.1$.
\end{enumerate}
\end{lemma}

In particular, the Kuperberg conjecture is reduced to calculations of
areas involving clusters of at most four acute triangles, each with
edge lengths at most $2.1$.

\begin{proof}  By Lemma~\ref{lemma:obtuse-main},  every triangle in
  the counterexample $\C$ is nonobtuse.  Hence (1).


Next we prove (2).  If $T\in \C$, then $T$ is nonobtuse, so that $b(T)
= \area(T) + \epso n_1(T)\ge \area(T)$.  

For part (3), we use Lemma~\ref{lemma:b-structure}.  We can eliminate
the case when $\C$ is a singleton, because in that case the single
triangle is not $b$-subcritical, and it satisfies the strict main
inequality.  The lemma now gives $T_0\in \C$ such that
\[\C = \{T_0\}\cup \{T \mid
  T\ra{b} T_0\}.
\]
We show that we can replace the arrows $(\ra{b})$ with $(\ra{a})$.
Suppose $T\rat T_0$.  We claim that $T\ra{a} T_0$ iff $T\ra{b} T_0$.
In fact, $T_0$ is nonobtuse so $(T,T_0)\not\in \N$ and
\[
b(T) = \area(T) - \epso n_2(T) \le \area(T).
\]
If $T\ra{a} T_0$ then $b(T)\le \area(T)\le \aK$, so $T\ra{b} T_0$.
Conversely, if $T\ra{b} T_0$, then $T\in \C$, so $T$ is nonobtuse and
$n_2(T) = 0$.  Thus, $\area(T) = b(T)\le \aK$, and $T\ra{a}T_0$.  This
completes the proof of the claim and the proof of (3).

There are at most three triangles in addition to $T_0$ in $\C$,
corresponding to the three edges of $T_0$.  This gives (4).

We have now completed the proof of all but the last property of $\C$
(concerning edge lengths $2.1$).  All the edges of a subcritical
triangle have lengths at most $2.1$ by Lemma~\ref{lemma:21}.  All the
triangles, except possibly $T_0$ are subcritical.  The edges of $T_0$
shared with $T\in \C\setminus\{T_0\}$ have length at least $1.72$ by
Lemma~\ref{lemma:172}.

Let $c = \card(\C\setminus \{T_0\})$. Assume $T_0$ has an edge length
at least $2.1$.  Then $c\le 2$, because that edge is not on a
subcritical triangle.  We show that $\C$ satisfies the strict main
inequality by showing that $\area(T_0) > \aK + c \epso$ so that
\[
\sum_{\C} b(T) \ge \sum_{\C}\area(T) 
   > (\aK + c\epso) + \sum_{\C\setminus\{T_0\}}\area(T) > (\aK + c\epso) + c a_0
 = \card(\C) \aK.
\]
Explicitly,  the following area estimates complete the proof:
\[
\area(T_0)\ge \begin{cases}
\area(1.72,2\kappa,2.1) > \aK+\epso,& \text{if } c = 1,\\
\area(1.72,1.72,2.1) > \aK+2\epso,& \text{if } c = 2.
\end{cases}
\] %checked 2016/2/18 in Mathematica.
\end{proof}

\section{Appendix on Explicit Coordinates}
\label{sec:appendix}

Let $A$ and $B$ be pentagons in contact, with $B$ the pointer at
vertex $\v_B$ to the receptor pentagon $A$.  Label vertices
$(\u_A,\w_A,\u_B,\v_B,\w_B)$ of $A$ and $B$ as in
Figure~\ref{fig:theta}.  Let $x=\norm{\v_B}{\w_A}$ and $\beta =
\angle(\v_B,\u_B,\u_A)$.  We have $0\le x\le 2\sigma$ and $0\le
\beta\le 2\pi/5$.

\tikzfig{theta}{coordinates for a pair of pentagons in contact}{
[scale=1.2]
\pen{0}{0}{0};
\pen{1.616}{0.598}{168.54};
\draw[blue] (0,0) node[anchor=south,black] {$B$}
 -- (1.616,0.598) node[anchor=south,black] {$A$};
\draw (0,0) node[anchor=north,black] {$\c_B$};
\smalldot{0,0};
\draw (1.616,0.598) node[anchor=north,black] {$\c_A$};
\smalldot{1.616,0.598};
\draw (72:1) node[anchor=south] {$\w_B$};
\draw (0:1) node[anchor=west] {$\v_B$};
\draw (-72:1) node[anchor=north] {$\u_B$};
\smalldot{72:1};
\smalldot{0:1};
\smalldot{-72:1};
\smalldot{0.636,0.797};
\draw (0.636,0.797) node[anchor=west] {$\w_A$}
node[anchor=north west] {$x$};
\smalldot{1.124,-0.2727};
\draw (1.124,-0.2727) node[anchor=north west] {$\u_A$}
node[anchor=north east] {$\beta$};
}


Let $\bl = \bl(x,\beta) = \norm{\c_A}{\c_B}$, viewed as a function of
$x$ and $\beta$.  We omit the explicit formula for $\bl$, but it is
obtained by simple trigonometry.  Under the symmetry
$\u_A\leftrightarrow \w_A$, $\u_B\leftrightarrow\w_B$, we have the
symmetries $x\leftrightarrow 2\sigma-x$, $\beta\leftrightarrow
2\pi/5-\beta$ and
\[
\bl(x,\beta) = \bl(2\sigma-x,2\pi/5 - \beta).
\]

If we have coordinates on a $3C$-triangle that determine the variables
$(x,\beta)$ for each of the pairs $\{A,B\}$, $\{B,C\}$, and $\{A,C\}$
of pentagons, then we may use the function $\bl$ to calculate the edge
lengths and area of the $3C$-triangle.

The space of all $P$-triangles is six-dimensional, described by the
three edge lengths of the Delaunay triangle and the three rotation
angles of the pentagons at its vertices.  The space of $3C$-triangles
is three dimensional, obtained by imposing three contact constraints
between pairs of pentagons.

\subsection{$\Delta$-junction}

We describe a coordinate system on $3C$-triangles of $\Delta$-junction
type.  As indicated in Figure~\ref{fig:cord-delta}, we use coordinates
$(x_\alpha,\alpha,\beta)$, where $x_\alpha$ is a length and $\alpha$
and $\beta$ are each angles between lines through edges of pentagons
in contact.  We assume that $B$ points into $A$ and into $C$ and that
$A$ points into $C$. The length $x_\alpha$ is the (small) distance
between the nearly coincident vertices of pentagons $B$ and $C$.  The
coordinates satisfy the conditions $0\le\beta\le\alpha\le\pi/5$,
$\alpha+\beta\le \pi/5$, and $x_\alpha\in[0, 2\sigma -
\sigma/\kappa]$.  Starting from these coordinates, we define $\gamma$
by $\alpha+\beta+\gamma=\pi/5$, and angles of the triangle $\Delta$ by
\begin{align}\label{eqn:abc}
\alpha+\alpha' &= 2\pi/5,\\
\beta+\beta' &= 2\pi/5,\nonumber\\
\gamma+\gamma' &= 2\pi/5.\nonumber
\end{align}
The edges $y_\alpha$, $y_\beta = 2\sigma$, and $y_\gamma$ of the
triangle $\Delta$ opposite the angles $\alpha'$, $\beta'$, $\gamma'$,
respectively are easily computed by the law of sines.  Define
$x_\beta$ by $x_\alpha+y_\gamma+x_\beta=2\sigma$, and $x_\gamma$ by
$y_\alpha+x_\gamma=2\sigma$.  The value $x_\beta$ is the distance
between the nearly coincident vertices of pentagons $A$ and $C$, and
$x_\gamma$ is the distance between the nearly coincident vertices of
pentagons $A$ and $B$.  The edges of the $3C$ Delaunay triangle have
lengths
\[
\bl(x_\alpha,\alpha'),\quad \bl(x_\beta,\beta'),\quad \bl(x_\gamma,\gamma').
\]

\tikzfig{cord-delta}{Coordinates for $\Delta$-types}
{
[scale=1.0]
\threepentnoD
{0.00}{0.00}{-5.16}%C
{0.99}{1.70}{235.38}%B
{1.98}{0.00}{183.43}; %A
\draw (0,0) node {$C$};
\draw (0.99,1.70) node {$B$};
\draw (1.98,0.0) node {$A$};
\draw (1.0,0.6) node {$\Delta$};
%\draw[blue] (0,0) -- (1.98,0);
\draw (-5.16:1) -- ++ (126 - 5.16 :2.5) node[anchor=south] {$\alpha$};
\draw (-5.16:1) -- ++ (- 180 + 126 - 5.16 :1.5) node[anchor=west] {$\beta$};
\draw ++ (1.98,0) ++ (183.43 - 72:1) -- ++ (54 + 3.43:1) node[anchor=south] {$\gamma$};
\draw (72-5.16:1) -- ++ (126 + 90 - 5.16:0.5);
\draw (0.99,1.70) ++ (235.38:1) node[anchor = south west] {$x_\alpha$} -- ++ (126+90 - 5.16:0.5);
}



\subsection{Pinwheel type}

We describe a coordinate system on $3C$-triangles of pinwheel type.
We assume that $C$ points into $B$, that $B$ points into $A$, and that
$A$ points into $C$.  As indicated in Figure~\ref{fig:cord-pinwheel},
we use coordinates $(\alpha,\beta,x_\gamma)$.  The angles $\alpha$ and
$\beta$ are angles between pentagon edges on touching pentagons.  The
value $x_\gamma$ is the distance between the pointer vertex of
pentagon $A$ and the pointer vertex of pentagon $B$.  The coordinates
satisfy constraints: $0\le\alpha$, $0\le\beta$, $\alpha+\beta\le
\pi/5$, and $0\le x_\gamma\le 2\sigma$.  Define $\gamma$ by
$\alpha+\beta+\gamma=\pi/5$.  The angles $\alpha'$, $\beta'$, and
$\gamma'$ of the inner background triangle $P$ of the pinwheel are
given by Equation~\ref{eqn:abc}.  The edge lengths $x_\alpha$,
$x_\beta$, $x_\gamma$ of the inner triangle $P$ are easily computed by
the law of sines.  The edges of the $3C$-triangle have lengths
\[
\bl(x_\alpha,\alpha),\quad \bl(x_\beta,\beta),\quad \bl(x_\gamma,\gamma).
\]

\tikzfig{cord-pinwheel}{Coordinates for pinwheel type}{
\begin{scope}[xshift=4cm,scale=1.2]
\threepentnoD{0.00}{0.00}{46.69}%A
{0.82}{1.53}{218.09}%C
{1.73}{0.00}{163.28};%B
\draw(0,0) node {$A$};
\draw(0.82,1.53) node {$C$};
\draw(1.73,0) node {$B$};
\draw(1.8,1.2) node {$\alpha$};
\draw(-0.2,1.0) node {$\beta$};
\draw(0.85,-0.7) node {$\gamma$};
\draw(0.9,0.5) node {$P$};
\draw(0.5,0.5) node {$x_\gamma$};
\smalldot{0.772,0.288}; %B pointer.
\smalldot{0.686,0.728}; %A pointer.
\end{scope}
}


\subsection{$L$-junction type}

We describe a coordinate system on $3C$-triangles of $L$-junction
type.  As indicated in Figure~\ref{fig:cord-L}, we use coordinates
$(\alpha,\beta,x_\alpha)$.  The angles $\alpha$ and $\beta$ are each
formed by edges of two pentagons in contact.  Let $x_\alpha$ be the
distance between the pointer vertex of $C$ to $A$ and the pointer
vertex of $B$ to $C$.  The coordinates satisfy relations:
$\alpha,\beta\in [0,2\pi/5]$, $\pi/5\le\alpha+\beta\le 3\pi/5$, and
$0\le x_\gamma\le 2\sigma$.  Define $\gamma$ by
$\alpha+\beta+\gamma=3\pi/5$.  The angles $\alpha'$, $\beta'$, and
$\gamma'$ of the inner $L$-shaped quadrilateral are given by
Equation~\ref{eqn:abc}.  The edge lengths of the $L$-shaped
quadrilateral are easily computed by triangulating the quadrilateral
into two triangles and applying the law of sines.  (Triangulate
$L$ by extending the line through the edge of $A$ containing the
pointer vertex of $C$ into $A$.)  This gives
$x_\beta$, the distance between the pointer vertex of $C$ to $B$ and
the inner vertex of $A$.  This gives $x_\gamma$, the distance between
the pointer vertex of $B$ and the inner vertex of pentagon $A$.
As before, the edges of the $3C$-triangle have lengths
\[
\bl(x_\alpha,\alpha),\quad \bl(x_\beta,\beta),\quad \bl(x_\gamma,\gamma).
\]

\tikzfig{cord-L}{Coordinates for $L$-junction type}{
\begin{scope}[xshift=8cm,scale=1.2]
\threepentnoD{0.00}{0.00}{80.86}
{0.97}{1.58}{232.21}
{1.82}{0.00}{223.24};
\draw (0,0) node {$C$};
\draw (0.97,1.58) node {$B$};
\draw (1.82,0) node {$A$};
\draw (2.0,1.2) node {$\gamma$};
\draw (-0.1,1.2) node {$\alpha$};
\draw (0.85,-0.8) node {$\beta$};
\draw (0.85,0.75) node {$L$};
\draw (0.55,0.4) node {$x_\alpha$};
%\smalldot{0.94,0.48};
%\smalldot{1.532,0.753};
\end{scope}
}


\subsection{$T$-junction type}

We describe a coordinate system on $3C$-triangles of $T$-junction
type.  As indicated in Figure~\ref{fig:cord-T}, we use coordinates
$(\alpha,\beta,x_\gamma)$.  The angles $\alpha$ and $\beta$ are each
formed by edges of two pentagons in contact.  The length $x_\gamma$ is
the distance between the pointer vertex of $B$ and the inner vertex of
$A$.  The coordinates satisfy: $\alpha,\beta\in[\pi/5,2\pi/5]$,
$3\pi/5\le \alpha+\beta\le 4\pi/5$, $0\le x_\gamma\le 2\sigma$.
Define $\gamma$ by $\alpha+\beta+\gamma=\pi$.  The angles $\alpha'$,
$\beta'$, and $\gamma'$ of the inner irregular $T$-shaped pentagon $P$
are given by Equation~\ref{eqn:abc}.  The edge lengths of the
$T$-shaped pentagon are easily computed by triangulating $P$ into
three triangles and applying the law of sines.  (Triangulate by
extending the edge of $P$ shared with $A$ that ends at the pointer
vertex of $A$ into $C$ and by extending the edge of $P$ shared with
$C$ that contains pointer vertex of $B$ into $C$.)  This gives
$x_\alpha$, the distance between the pointer vertex of $B$ to $C$ and
the inner vertex of $C$.  This gives $x_\beta$, the distance between
the pointer vertex of $A$ to $C$ and the inner vertex of $C$.  As
before, the edges of the $3C$-triangle have lengths
\[
\bl(x_\alpha,\alpha),\quad \bl(x_\beta,\beta),\quad \bl(x_\gamma,\gamma).
\]



\tikzfig{cord-T}{Coordinates for $T$-junction types}{
\begin{scope}[xshift=12cm,scale=1.2]
\threepentnoD{0.00}{0.00}{114.48} %A
{0.90}{1.59}{237.18} %B
{1.66}{0.00}{219.24}; %C
\draw (0,0) node {$A$};
\draw (0.9,1.59) node {$B$};
\draw (1.66,0) node {$C$};
\draw (0.9,0.9) node {$P$};
\draw (0,1.1) node {$\gamma$};
\draw (0.75,-0.9) node {$\beta$};
\draw (1.9,1.2) node {$\alpha$};
\draw (0.45,0.5) node {$x_\gamma$};
\smalldot{0.737,0.675};
\smalldot{0.358,0.75};
\end{scope}
}

\subsection{pin-$T$ junction type} We describe a coordinate system on
$3C$-triangles of pin-$T$ junction type.  As indicated in
Figure~\ref{fig:cord-pint}, we use coordinates $\alpha$, $\beta$, and
$x_\alpha$.  The angles $\alpha$ and $\beta$ are each formed by edges
of two pentagons in contact.  The length $x_\alpha$ is the distance
between the nearly coincident vertices of $B$ and $C$.
The coordinates satisfy $\pi/5 \le \alpha \le 2\pi/5$, $\pi/5\le
\beta \le 2\pi/5$, and $3\pi/5 \le \alpha+\beta$.
Lemma~\ref{lemma:0605} shows that $0\le x_\alpha\le 0.0605$.  Define
$\gamma$ by $\alpha+\beta+\gamma = \pi$.  The angles $\alpha'$,
$\beta'$, and $\gamma'$ of the inner irregular $T$-shaped pentagon are
given by Equation~\ref{eqn:abc}.  The edge lengths of the $T$-shaped
pentagon $P$ are easily computed by triangulating $P$ into three
triangles and applying the law of sines.  (Triangulate by extending
the two edges of $P$ that meet at the pointer vertex of $C$ into $A$.)
This gives $x_\beta$, the distance between the pointer vertex of $C$
to $A$ and the inner vertex of $A$.  This gives $x_\gamma$, the
distance between the pointer vertex of $A$ to $B$ and the pointer
vertex of $B$ into $C$.  As before, the edges of the $3C$-triangle
have lengths
\[
\bl(x_\alpha,\alpha),\quad \bl(x_\beta,\beta),\quad \bl(x_\gamma,\gamma).
\]

\tikzfig{cord-pint}{Coordinates for pin-$T$ junction types. Although
  it is difficult to tell from the figure, $A$ points into $B$, $B$
  into $C$, and $C$ into $A$.  The nearly horizontal edges of $A$ and $C$
  need not be parallel. The parameter $\beta'$ measures their
  incidence angle.  The region bounded by the three pentagons is a
  $T$-shaped pentagon, with stem between the nearly parallel
  edges of $A$ and $C$ and two arms along $B$.  The arm between $B$
  and $C$ is imperceptibly small.}{
\begin{scope}[xshift=4.5cm,yshift=1.5cm]
\threepentnoD{0.00}{0.00}{90} %C
{1.40}{-1.377}{0} % B
{-0.35}{-1.628}{-18.89}; % A
\draw (0,0) node {$C$};
\draw (1.4,-1.37) node {$B$};
\draw (-0.35,-1.62) node {$A$};
\draw (0.36,-2.4) node {$\gamma$};
\draw (-1.1,-0.5) node {$\beta$};
\draw (1.0,-0.3) node {$\alpha$};
\draw (0.4,-0.49) node {$x_\alpha$};
\draw (1.4 - 0.809,-1.377 - 0.5878) -- +(0,-0.75);
\end{scope}
}



\begin{lemma}\label{lemma:0605}
  Let $T$ be a $3C$ triangle of type pin-$T$.  The coordinates
  $\alpha$, $\beta$, and $x_\alpha$ satisfy the relation
\[
x_\alpha \sin(2\pi/5) \le 2\sigma (\sin(\alpha+\pi/5) - \sin(\beta+\pi/5)).
\]
In particular, $x_\alpha \le 0.0605$.
\end{lemma}

\begin{proof} Let $\v_{AB}$ be the pointer vertex of $A$ to $B$, and
  let $\v_{BC}$ be the pointer vertex of $B$ to $C$.  Let $\v$ and
  $\v_{BC}$ be the endpoints of the edge of $B$ containing $\v_{AB}$.
  We represent $T$ as in Figure~\ref{fig:cord-pint}, with the lower edge of
  $C$ along the $x$-axis.  Since $\v_{AB}$ lies on the segment between
  $\v$ and $\v_{BC}$, The $y$-coordinate $y(\v_{AB})$ of $\v_{AB}$ is
  nonpositive and lies between the $y$-coordinates $y(\v)$ and
  $y(\v_{BC})$.  We have
\begin{align*}
y(\v) &= x_\alpha \sin(2\pi/5) - 2\sigma\sin(\alpha+\pi/5)\\
y(\v_{AB}) & = -x_\beta \sin(\beta') - 2\sigma\sin(\beta+\pi/5).
\end{align*}
Using $x_\beta \sin(\beta')\ge 0$ and $y(\v) \le y(\v_{AB})$, we
obtain the claimed inequality.

Recall that $\pi/5\le \beta \le 2\pi/5$.  In particular, we have
$\sin(\alpha+\pi/5) \le 1$ and $\sin(\beta+\pi/5)\ge \sin(2\pi/5)$.
This gives
\[
x_\alpha \le 2\sigma(1/\sin(2\pi/5) - 1) < 0.0605.
\]
\end{proof}


This completes our discussion of coordinates used for computations.

