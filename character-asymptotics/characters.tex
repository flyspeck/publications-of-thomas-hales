\documentclass[12pt]{amsart}
\usepackage{amssymb,amscd,amsmath,amsthm,color}
\usepackage{calc}
\usepackage{amsrefs}
\usepackage{color}
\usepackage[T1]{fontenc}
\usepackage{mathtools}
\usepackage[normalem]{ulem}
\usepackage{tikz}
\usetikzlibrary{matrix,arrows,decorations.pathmorphing}
\usepackage{setspace}
\usepackage{verbatim}
\usepackage{mathrsfs}
%\usepackage[notcite, notref]{showkeys}
%\usepackage[left=2cm,top=2cm,right=2cm,nohead]{geometry}

\textwidth 6.5 in
\oddsidemargin 0 in
\evensidemargin 0 in
\topmargin -.125 in
\textheight 8.75 in


\definecolor{bettergreen}{rgb}{0,.7,0}
%\newcommand\blue[1]{{\color{blue}{#1}}}
%\newcommand\green[1]{{\color{bettergreen}{#1}}}
%\newcommand\red[1]{{\color{red}{#1}}}

%\long\def\comment#1{\marginpar{{\footnotesize\color{red} #1\par}}}
%\long\def\commentimmi#1{\marginpar{{\footnotesize\color{bettergreen} #1\par}}}
%\long\def\change#1{{\color{blue} #1}}
%newcommand\green[1]{{\color{green} #1}}
%\newcommand   [1]{{\color{red}\small #1}}

%\let\immi=\green
%\let\raf=\blue

\newcommand\todo[1]{\ \vspace{5mm}\par \noindent\framebox{\begin{minipage}[c]{0.95 \textwidth} \tt #1\end{minipage}} \vspace{5mm} \par}


\DeclarePairedDelimiter\ceil{\lceil}{\rceil}
\DeclarePairedDelimiter\floor{\lfloor}{\rfloor}


%%%%  Definitions  re: typeface  %%%%%%%%%%

\newcommand{\A}{\mathbb{A}}
\newcommand{\Q}{{\mathbb Q}}
\newcommand{\C}{{\mathbb C}}
\newcommand{\R}{{\mathbb R}}
\newcommand{\Z}{{\mathbb Z}}
\newcommand{\N}{{\mathbb N}}
\newcommand{\LL}{{\mathbb L}}
\newcommand{\TT}{{\mathbb T}}
\newcommand{\CC}{{\mathbb C}}
\newcommand{\ZZ}{{\mathbb Z}}


\newcommand{\bF}{\mathbf{F}}
\newcommand{\Gl}{\mathbf {GL}}
\newcommand{\SL}{\mathbf {SL}}
\newcommand{\Sp}{\mathbf {Sp}}
\newcommand{\aut}{\mathbf{Aut}}
\newcommand{\bG}{\mathbf{G}}
\newcommand{\bT}{\mathbf {T}}
\newcommand{\bM}{\mathbf {M}}


\newcommand{\cF}{\mathcal{F}}
\newcommand{\cO}{\mathcal{O}}
\newcommand{\ri}{\mathcal{O}}
\newcommand{\cM}{\mathcal {M}}
\newcommand{\cV}{\mathcal{V}}
\newcommand{\cG}{\mathcal{G}}
\newcommand{\cB}{\mathcal{B}}


\newcommand{\gl}{\mathfrak{gl}}
\newcommand{\fg}{\mathfrak{g}}
\newcommand{\ft}{\mathfrak{t}}
\newcommand{\fz}{\mathfrak{z}}
\newcommand{\fsp}{\mathfrak{sp}}
\newcommand{\fsl}{\mathfrak{sl}}
\newcommand{\fO}{\mathfrak{O}}
\newcommand{\fP}{\mathfrak{P}}
\newcommand{\ff}{\mathfrak{f}}
\newcommand{\fh}{\mathfrak{h}}
\newcommand{\fF}{\mathfrak{F}}


\newcommand{\reg}{\mathrm{rss}}
\newcommand{\ur}{\mathrm{ur}}
\newcommand{\nil}{\mathrm{Nil}}
\newcommand{\val}{\mathrm{val}}
\newcommand{\rs}{\mathrm{rs}}
\newcommand{\ram}{\mathrm{Ram}}
\newcommand{\can}{\mathrm{can}}
\newcommand{\diag}{\mathrm{diag}}
\newcommand{\oi}{{\bf \mathrm{O}}}


\newcommand{\Ad}{\operatorname{Ad}}
\newcommand{\jac}{\operatorname{Jac}}
\newcommand{\gal}{\operatorname{Gal}}
\newcommand{\res}{\operatorname{Res}}
\newcommand{\vol}{\operatorname{vol}}
\newcommand{\supp}{\operatorname{supp}}
\newcommand{\ep}{\operatorname{EP}}
\newcommand{\cI}{{\mathcal I}}


\newcommand{\rf}{k}


%%%%  Motivic  definitions  %%%%%%%

\newcommand{\ldp}{{\mathcal L}_{\mathrm {DP}}}
\newcommand\ldpo[1][\ri]{{\mathcal L}_{#1}}
\newcommand\cmf[1]{{\mathcal C}(#1)}
\newcommand\cA{{\mathcal A}}
%\newcommand\co{{\mathcal O}}
\newcommand\ord{\mathrm{ord}}
\newcommand\ac{\overline{\mathrm{ac}}}
\newcommand\lef{\mathbb L}
\newcommand\cP{{\mathcal P}}
\newcommand\cC{{\mathcal C}}
\newcommand\cH{{\mathcal H}}
\newcommand\cX{{\mathcal X}}
\newcommand\mot{\mathrm{mot}}
\newcommand{\scD}{\mathscr{D}}
\newcommand{\rss}{\mathrm{rss}}
\newcommand{\bGam}{{\mathbf \Gamma}}
\newcommand{\Loc}{\mathrm{Loc}}


\newcommand{\de}{{\text{Def}}}
\newcommand{\rde}{{\text{RDef}}}
\newcommand{\K}{F}
\newcommand{\mexp}{\mathbf{e}}
\newcommand{\tf}{{C_c^\infty}}


\def\llp{\mathopen{(\!(}}
\def\llb{\mathopen{[\![}}
\def\rrp{\mathopen{)\!)}}
\def\rrb{\mathopen{]\!]}}


%%%%%%%%Definitions from Lance %%%%%%%%%%%%%%%%

\newcommand{\Hasse}{\operatorname{Hasse}}
 %\newcommand{\diag}{\operatorname{diag}}
 \newcommand{\spa}{\operatorname{span}}
 \newcommand{\spec}{\operatorname{spec}}
 \newcommand{\Stab}{\operatorname{Stab}}
\newcommand{\m}{\operatorname{mod}}
 \newcommand{\tr}{\operatorname{tr}}
 \newcommand{\disc}{\operatorname{disc}}
 \newcommand{\meas}{\operatorname{meas}}
 \def\ve{\varepsilon}
 \def\an{_{\text{aniso}}}
 \def\gerg{\mathfrak{g}}
\def\gerh{\mathfrak{h}}
\def\gert{\mathfrak{t}}
\def\gergbar{\overline{\mathfrak{g}}}
\def\dgdot{d\dot{g}}
\def\dmu{d\mu}
\def\Qtilde{\widetilde{Q}}
%\def\cA{\mathcal{A}}
\def\C{\mathbb C}
\def\R{\mathbb R}
\def\N{\mathbb N}
\def\Q{\mathbb Q}
\def\Z{\mathbb Z}
\def\A{\mathbb A}
\def\Fp{\mathbb{F}_p}
\def\Fq{\mathbb{F}_q}
\def\Qp{\mathbb{Q}_p}
\def\GL{\mathrm{GL}}
\def\GLN{\mathrm{GL}_N}
\def\GLn{\mathrm{GL}_n}
\def\Lie{\mathrm{Lie}}
%\def\SL2{\mathrm{SL}_2}
%\def\Sp{\mathrm{Sp}}
\def\sln{\mathfrak{sl}_n}
\def\gln{\mathfrak{gl}_n}
\def\sl{\mathfrak{sl}}
\def\Gm{{\mathbb{G}_m}}
\def\Grs{G_{\text{rss}}}
\def\Gbar{\overline{G}}
%\def\bG{\mathbb{G}}
\def\bc{\boldsymbol{c}}
%\def\bT{\mathbb{T}}
\def\FpT{\mathbb{F}_p((T))}
\def\Fbar{\overline{F}}
\def\Aff{\mathbb{A}}
\def\cci{C_c^\infty}
\def\cO{\mathcal{O}}
\def\cQbar{\overline{\mathcal{Q}}}
\def\cQtilde{\widetilde{\mathcal{Q}}}
\def\spn{\mathfrak{sp}_{2n}}
\def\sp{\mathfrak{sp}}
\def\cT{\mathcal{T}}
\def\cL{\mathcal{L}}
\def\cP{\mathcal{P}}
\def\cN{\mathcal{N}}
\def\cE{\mathcal{E}}
\def\cF{\mathcal{F}}
\def\cH{\mathcal{H}}
\def\cR{\mathcal{R}}
\def\cS{\mathcal{S}}
\def\cQ{\mathcal{Q}}
\def\cX{\mathcal{X}}
\def\gO{\mathfrak{O}}
\def\gp{\mathfrak{p}}
\def\gP{\mathfrak{P}}
\def\gq{\mathfrak{q}}
\def\inv{^{-1}}

% Defs from Hales

\newcommand{\op}[1]{\operatorname{#1}}
\newcommand{\ring}[1]{{\mathbb #1}}
\def\rtie{\times}

% More defs 

\newcommand{\fG}{\mathfrak G}
\newcommand{\fT}{\mathfrak T}

%%%%%%%%%%%%% Theorem declarations %%%%%%%%%%%%%

\theoremstyle{plain}
\newtheorem{thm}{Theorem}
\newtheorem{theorem}[thm]{Theorem}
\newtheorem{lem}[thm]{Lemma}
\newtheorem{cor}[thm]{Corollary}
%\newtheorem{defn}[thm]{Definition}
%\newtheorem{rem}[thm]{Remark}
\newtheorem{prop}[thm]{Proposition}

\theoremstyle{definition}
\newtheorem{rem}[thm]{Remark}
\newtheorem{defn}[thm]{Definition}
\newtheorem{example}[thm]{Example}

\title{Coefficients of Harish-Chandra's local character expansion are motivic}

\author{Julia Gordon, Thomas Hales and Loren Spice}



\begin{document}

%\begin{abstract}  
%\end{abstract}

\maketitle

\rightline{17 hours 54 minutes since the start}
\section{Introduction}
The present version of article is inspired by \cite{unsuccessful self-treatment of writer's block}. 
 
We try to prove that the coefficients of Harish-Chandra's local character expansion are motivic functions of the parameters defining the group and the representation. 

\section{Reductive groups}
Define field extensions, fixed data, the cocycle space $Z$, and a family of reductive groups $G$ over $Z$ as in \cite{transfer transfer}. Define Haar measure on $G_z$ the same way as well.
Will also use the term ``fixed choices'' in the same sense. 

\section{Representations}
We assume that $p$ is large. 
Supercuspidal representations $\pi$ of $G_z$ are parametrized by J.-K. Yu's data 
$(\bar G, \bar\phi, \pi_0, x)$, where $\bar G=(G^1, \dots G^d)$ is a sequence of twisted Levi subgroups with $G^d=G$, $\phi_i$ is a character of $G^i$, $\pi_0$ is a depth-zero supercuspidal representation of $G^0$, and $x$ is a point in the reduced building of $G$.
Let us fix the depth of $\pi$. 

The groups $G^i$ are encoded by appropriate  cocycle spaces $Z^i$.

The representation $\pi_0$ is determined by its inducing datum: a parahoric subgroup $K^0$ and a representation $\rho_0$ of its reductive quotient $\fG^0$. 

\subsection{Parahorics} Suppose the depth of the representation $\pi$ is fixed. Then there are finitely many points in the alcove that can arise in J.-K. Yu's construction. This list will become part of fixed choices. 

  
\subsection{Deligne-Lusztig representations} 
For now, consider the case when $\rho_0$ is a Deligne-Luszting representation. 
Then it is determined by an elliptic torus $\fT$ in $\fG$, which is determined by an element $w$ in the Weyl group of $\fG$, and a character $\chi$ of $\fT$.  
The element $w$ becomes  part of the fixed choices.
  
The multiplicative characters $\phi_i$, as well as $\chi$, will play no role, so we will group representations into 
families with the same $(\bar G, x, w)$. 

Thus, fixed choices will determine: the root system of $G$, root systems of $G_i$, {\bf{hopefully}} the point $x$ and the data that determines the parahoric $K_0$, and an element $w$ that determines 
$\fT$.  
Every fixed choice gives a family of representations of each of the groups $G_z$, with $z\in Z$ -- the cocycle space determined by this fixed data.
For a given fixed choice, we have the cocycle spaces $Z$ (determining $G$), and $Z^i$ (determining $G^i$), and from this we can reconstruct the representations. 
Let $\cF_d$ be the set of fixed choices for fixed $d$.
The elements of $\cF_d$ are tuples 
$(X^\ast, X_\ast, \Phi, \Phi^\vee), \Sigma, (\Phi^i, {\Phi^\vee}^i)_{i=0}^d, \Sigma^i, x_0, \bar x, w)$, where 
\begin{enumerate}
\item  $(X^\ast, X_\ast, \Phi, \Phi^\vee)$  determines the absolute root datum of $G$, 
\item $\Sigma$ is an enumerated Galois group (with a choice of inertia subgroup and a 
generator of its cyclic quotient that gives the Galois group of the maximal unramified 
subextension) of the Galois extension that splits $G$,
\item  $(\Phi^i, {\Phi^\vee}^i, \Sigma^i)$ does the same for $G^i$. 
\item  The point $x$ (along with some fixed $r$) determines the parahoric $K^0$.   
\item $w$ determines  $\fT$. 
\end{enumerate}

We want to prove: 
\begin{theorem} (Deligne-Lusztig case) Fix an element of $\cF_d$. 
Then there is an associated cocycle space $Z\times Z^1\times \dots Z^d$, and a family of representations $\pi_{z_i}$ of $G_z$. 
For $F\in \Loc_m$, every connected reductive group $G$ over $F$ appears as a member  of such  family for some fixed choice; 
every supercuspidal representation of $G$ that has a Deligne-Luzstig depth-zero component 
appears in the family of representations of $G_z$, and 
the coefficients of Harish-Chandra's local character expansion are motivic functions of 
$z, z_i$. 
\end{theorem}

General case: should include the decomposition of a general character near the identity into Deligne-Lusztig ones, and the parameters that account for this -- will probably be part of fixed choices. 

\subsection{field-independent choices for the depth-zero data}


\section{Asymptotic character formulae}
\subsection{mock exponential maps}

\subsection{asymptotic character expansion}
Loren says:
$$\Theta_\pi(\mexp(X))= \sum_{\cO\in \cO(X)} c_\cO\widehat\mu_{\cO},$$


We will prove by induction
\begin{theorem} The coefficients $c_{\cO}$ of this expansion are motivic functions of the parameters defining $G$, $\pi$ (as above) and $\mexp$. 
\end{theorem}

\begin{cor}
Coefficients of Harish-Chandra local character expansion are motivic functions of the same thing.
\end{cor}

\section{Done!}

\end{document}





