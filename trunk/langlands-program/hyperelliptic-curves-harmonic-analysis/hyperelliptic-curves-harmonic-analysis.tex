% First line of file hyperelliptic-curves-harmonic-analysis.tex
% paper on Hyperelliptic curves and harmonic analysis
% Written and typed by Thomas C. Hales
% Format AMS-TeX
% Created April 26, 1993
% Version 1.2
% Most recent corrections made June 1, 1994
% 
% proc. of the Rutgers conference in memory
%     of Larry Corwin, edited by R. Goodman
% 
% Reformatted March 2015.
%

\documentclass{amsart}

\usepackage{amssymb}
\usepackage{amsthm}
\usepackage{enumitem}
\usepackage{amsrefs}

%\input amstex
%\documentstyle{conm-p}
%\NoBlackBoxes
%
%\leftheadtext{THOMAS C. HALES}
%\rightheadtext{HYPERELLIPTIC CURVES AND HARMONIC ANALYSIS}

%\magnification=\magstep1
%\parindent=0pt
%\parskip=\baselineskip
%\baselineskip=1.5\baselineskip

\begin{document}

\title[Hyperelliptic Curves]{Hyperelliptic Curves and Harmonic Analysis\\
      {\it (Why harmonic analysis on reductive 
                       p-adic groups is not elementary) }}
\author{Thomas C. Hales}
\address{Department of Mathematics, University of Michigan, Ann Arbor, Michigan}


%

\subjclass[2000]{22E50; Secondary 22E35, 14H52, 11G20}
\begin{abstract}
This paper constructs several families of hyperelliptic
curves over finite fields and shows how basic objects
of $p$-adic harmonic analysis are described by the
number of points on these curves.  This leads to proofs
that there are no general elementary formulas for
characters and other basic objects of harmonic analysis
on reductive $p$-adic groups.  Basic objects covered
by this theory include characters, Shalika germs,
Fourier transforms of invariant measures, and the
orbital integrals of the unit element of the Hecke algebra.
Various examples are given for unitary, orthogonal, and
symplectic groups.
\end{abstract}

\thanks{Research supported by an NSF postdoctoral fellowship}
\thanks{Contemporary Mathematics 177 (1994): 137--137.\quad Copyright 1994, AMS}
\thanks{This paper is in final form, and no version of it
will be submitted for publication elsewhere.}


%\UseAMSsymbols
%\loadmsbm
%\loadeufm
%\pretolerance=10000
%\raggedbottom

\newcommand\Fq{{\Bbb F}_q}
\newcommand\PSp{{P\! Sp}}
\newcommand\calO{{\mathcal O}}
\newcommand\B{{\frak B}}
\newcommand\h{{\frak h}}
\newcommand\g{{\frak g}}
\newcommand\p{{\frak p}}   %maximal ideal
\newcommand\gsp{{\frak gsp}}
\newcommand\so{{\frak so}}
\newcommand\bP{{\Bbb P}}
\newcommand\G{{\Bbb G}}
\newcommand\Q{{\Bbb Q}}
\newcommand\Z{{\Bbb Z}}
\newcommand\C{{\Bbb C}}
\newcommand\Ima{{\mathcal I\hskip-.2em m}}
\newcommand\bGamma{\bar\Gamma}
\newcommand\diag{\hbox{diag}}
\newcommand\bk{\bar k}
\newcommand\Gal{\operatorname{Gal}}
\newcommand\bF{{\bar F}}
\newcommand\vvol{\operatorname{vol}}

\font\smc=cmcsc10
\def\proclaim#1{\medbreak\medskip\noindent 
  {\smc#1.\enspace}\sl} 
\def\endproclaim{\par\rm 
  \ifdim\lastskip<\medskipamount\removelastskip 
  \penalty55\medskip\fi} 

%\newcommand\star{\hbox to 0pt{\hskip -10pt $\bullet$ \hss}}  %compare Knuth, Exercise 14.28
%\newcommand\leftmargin{$\bullet$
%    \vadjust{\vbox to 0pt{\vskip-0.8\baselineskip\hbox to \hsize{\star\hfil}\vss}}}

% small square box:
%\newcommand\x{%
%        \hskip .05em
%        \vbox{\hrule height .4pt%
%          \hbox{\vrule width .4pt%
%                      \hskip .25em%
%                      \vbox{\vskip .61em}%
%                      \hskip .25em%   was 15 on both sides
%                      \vrule width .4pt%
%                }
%          \hrule height .4pt%
%          }}
%\newcommand\qed{\hfill\hbox{}\nobreak\hfill\x}



%Cover Page
%\hbox{}
%\centerline{\bf Hyperelliptic Curves and 
%                 Harmonic Analysis}

\maketitle

%\document

% This is the body of paper on hyperelliptic curves and
%   harmonic analysis by Thomas C. Hales
% Small changes made June 1, 1994

% Accepted for publication in the proc. of the Rutgers conference
% in memory of L. Corwin, edited by R. Goodman.


% Body

\section{Characters and Curves} %1

{\it What is the form of a character of an admissible representation
of a reductive $p$-adic group, and more generally what 
is the form of 
the basic objects of invariant harmonic analysis on these groups?}
Kazhdan, Lusztig, and Bernstein were the first to realize that the answer
to this
question is not elementary, and will have, when the full story
is eventually told, a gratifying answer.
  This is the implication
of a 
1988 paper of Kazhdan and Lusztig \cite{KLB}.
The stated purpose of
their paper is to study fixed-point varieties on complex affine
flag manifolds,
but the final section and an appendix by Kazhdan
and Bernstein turn to the rank-three symplectic group
over the local field $F=\Fq((t))$ of sufficiently
large characteristic.
They construct a 
cuspidal representation
of the group $Sp(6,F)$ and a collection of regular elliptic 
elements $\{u_\lambda\}$ parameterized by elements $\lambda$ in
the finite field $\Fq$.  The character $\chi_w$
of the representation
evaluated at $u_\lambda$ has the form $\chi_w(u_\lambda) = A(q) + B(q)
\# E_\lambda(\Fq)$, where $A$ and $B$ are 
nonzero polynomials in $q$,
and $E_\lambda$ is an elliptic curve over the finite field $\Fq$
with $j$-invariant $\lambda$.  Based on the form of this character,
they conclude the paper with the 
claim, ``In particular, we see that there is
no `elementary' formula for $\chi_w(u)$.'' 

One of the most compelling problems in nonarchimedean harmonic analysis
is to understand the implications of this
claim for arbitrary reductive groups.   In this 
paper, we adopt the same notion of elementariness: an object is
not {\it elementary\/} if its description depends essentially
on the number of points on a family of nonrational
varieties over a finite field.  In our examples, these varieties will be
elliptic or hyperelliptic curves.  (A {\it hyperelliptic curve} is a complete
nonsingular curve of genus $g\ge 2$ that admits a finite morphism
of degree two to the projective line $\bP^1$.)
We will also say, with the same
meaning, that 
the given object or group is {\it nonrational}.
This paper makes significant progress on (without completing)
the task of delineating which reductive groups are not elementary.
Examples support the expectation
that the harmonic analysis on most reductive groups will not be elementary.
Perhaps only $GL(n)$ 
and its relatives (possibly including unramified unitary groups) 
will possess an elementary theory.

The Kazhdan-Lusztig-Bernstein example for $Sp(6)$ stands in sharp contrast to the
experience with $GL(n)$.   
Substantial evidence
suggests a rational theory, especially calculations of characters,
Fourier transforms of invariant measures supported
on nilpotent orbits, and Shalika germs.  
The work of Corwin and Howe \cite{CH} gives characters on tamely
ramified division algebras, which, by the abstract matching
theorem \cite{BDKV}, gives the characters of the discrete series 
of $GL(n)$ on the
elliptic set.  Corwin and Sally 
have made this even more concrete \cite{CS}.
Harish-Chandra established a local expansion of characters as linear
combinations of
Fourier transforms of nilpotent orbits, and 
formulas for these Fourier transforms on
$GL(n)$ appear in \cite{Ho}.  
Waldspurger gives an algorithm for computing the Shalika germs
on the group $GL(n)$ \cite{W1}.  In none of this work on the general linear
group is there any indication
of nonrational behavior.

For a general reductive group very little explicit work
has been done along
these lines.  But there is mounting evidence to show
that the harmonic analysis on a general reductive group behaves
more like the $Sp(6)$ example than $GL(n)$.  Specifically, 
examples show that the following objects of local harmonic
analysis admit no elementary description.
In the following theorem, let $F$ be a $p$-adic
field of characteristic zero, and assume that its residual
characteristic is odd. We say that there is no {\it general elementary
formula\/} for an object, if there exists a reductive $p$-adic group $G$
over $F$, and a given object on the group $G$ whose description
is not elementary (in the sense given above).

\proclaim{Theorem 1.1}
\begin{enumerate}[label=(\arabic*)]
\item
There is no general elementary formula for characters.
\item
There is no general elementary formula for the Fourier transform of 
invariant measures supported on nilpotent
orbits.
\item
There is no general elementary formula for Shalika germs.
\item
There is no general elementary formula for the orbital integrals of the
unit element of Hecke algebras.
\item
The Langlands principle of functoriality
 is not generally elementary. 
\end{enumerate}
\endproclaim

Statement 5 has the
following interpretation.  Kottwitz and Shelstad have developed
a theory of twisted endoscopy (\cite{KS1},\cite{KS2}).  In it,
 they make precise conjectures
about the relationship between orbital integrals on two different groups.
In special cases, we express orbital integrals in terms of points on varieties
over finite fields.  The conjectures of Kottwitz and Shelstad may
then be viewed as conjectural relationships (e.g., correspondences) between
the underlying varieties.   Statement 5 asserts that examples may be
produced in which the underlying
varieties on the two groups are not rational and, moreover, are not
birationally equivalent to each other.  What we actually show is more
precise: the varieties in our example are elliptic curves, and the
matching or transfer is expressed as an isogeny between the two
curves.

It seems likely that Murnaghan's work on Kirillov-type formulas for
local character expansions will also imply that there is no general
elementary formula for the coefficients $c_{\calO}(\pi)$ arising
in Harish-Chandra's local character expansion, but I have not
checked this (see \cite{M2}).

We would also like to delineate which groups exhibit 
nonrational behavior (say (1), (2), (3), or (4) of Theorem 1.1).  
The following
groups display at least one of these four types of
nonrational behavior.  In the next theorem, 
suppose that $F$ is a $p$-adic field
of characteristic zero, with sufficiently large residual characteristic.


\proclaim{Theorem 1.2}
The invariant harmonic analysis
on the following groups is {\it not} elementary:
\begin{enumerate}[label=(\arabic*)]
\item
the group ${}^2\! A_n$, for $n\ge 2$ 
     (quasi-split or an inner form),
    assuming that
the group splits over a {\it ramified} quadratic extension,
\item
   the group $B_n$, for $n\ge 2$
    (split or an inner form),
\item
   the split group $C_n$, for $n=2,3$, 
\item
  the split group $D_4$. 
\end{enumerate}
\endproclaim

To prove the theorem, we show that the Shalika germs associated with
certain subregular unipotent classes, when evaluated at particular
elliptic semisimple elements,  are expressed by the number
of points on families of elliptic or
hyperelliptic curves over finite fields.  In the list of Theorem 1.2, 
the center
of the group is irrelevant, and one can build  further
examples by taking products, isogenies, and so forth.
The groups $U(3)$ and $GSp(4)$ come as
pleasant surprises because of the
already considerable literature
on these groups. 

Assuming the principle of
functoriality, we can add more groups to this list.  Again, we assume
that $F$ is a $p$-adic field of characteristic zero and odd
residual characteristic.

\proclaim{Theorem 1.3}
Assume the conjectures of Kottwitz
and Shelstad on the stabilization of twisted orbital integrals.
Then the invariant harmonic analysis on the following additional groups
is {\it not} elementary:
\begin{enumerate}[label=(\arabic*)]
\item the split group $C_n$, for $n\ge 2$, 
\item the group $D_n$, for $n\ge 5$ (any quasi-split form).
\end{enumerate}
\endproclaim

In all of our examples, we produce nonrational behavior
on the elliptic set.  Trivial extensions of these
theorems could be obtained by including the full set
of regular semisimple elements.

I have not determined whether the exceptional groups $(G_2,F_4,E_6,E_7,E_8)$
display nonrational behavior on the elliptic set.  
I suspect that 
none of the exceptional groups will be elementary.  Also,
the status of unramified unitary groups remains 
unresolved.
The Shalika germs that I have considered here are elementary,
and it would not be overly surprising for the entire theory to
be rational in this case.

Other results along these lines arise in 
the twisted harmonic analysis on reductive groups
with an automorphism (see \cite{KS1},\cite{KS2}).
For instance, consider the group $GL(2n+1)$
and the outer automorphism of the group fixing a splitting.  
Then for $n\ge 2$, its
twisted theory is not elementary.  Assuming the conjectures
of Kottwitz and Shelstad, one can add 
several other twisted cases to the list, including
$GL(2n)$ twisted by the
outer automorphism that fixes a splitting, for $n\ge 2$.

Sections 4 through 7 will give complete proofs of Theorems 1.1, 1.2, and 1.3
except for the proofs of the nonrationality of $C_3$ and $D_4$.
Although we do not give the proof that $C_3$ and $D_4$ are not
elementary, this is at least made plausible by Examples 2.3 and 2.6
of Section 2.  These examples show how elliptic curves are associated
with $Sp(6)$ and $SO(8)$ (without relating these examples back to
the harmonic analysis of the group).  
  Section 2
gives short arguments establishing the existence of elliptic
or hyperelliptic
curves in several reductive groups.

In Section 3 we give a brief account of a general framework that
ties harmonic analysis to varieties over finite fields.  This section
contains no proofs (and no theorems).  Section 4 gives a series
of concrete examples that establish that certain Shalika
germs are not elementary for many reductive groups.  The stable orbital integrals
of the unit element of the spherical 
Hecke algebra of $SO(2n+1)$ are
treated in Section 5.  Section 6 gives a couple of examples of the
Fourier transforms of nilpotent orbits in low rank.  Section 7 shows
how the examples of Sections 4, 5, and 6 establish Theorems 1.1, 1.2,
and 1.3 
(with the exception, noted above, of the nonrationality of $C_3$ and $D_4$).
The proof that the harmonic analysis of
$D_n$ is not elementary relies on the result for $D_4$.

At a 1992 conference at Luminy, J.-L. Waldspurger announced
results concerning the homogeneity of certain invariant distributions
(\cite{W2},\cite{W3}).  By giving general explicit estimates on the domain of
validity of Harish-Chandra's local character expansion, Waldspurger's
work related the Kazhdan-Lusztig-Bernstein example on $Sp(6)$ to the
Fourier transforms of nilpotent orbits in positive characteristic.
This gave reason to believe that the Fourier transforms of nilpotent
orbits would not be elementary in characteristic zero. 
At the same conference, G. Laumon suggested to me that 
it was hardly possible for the fundamental lemma to be elementary,
in light of the
Kazhdan-Lusztig-Bernstein example.  Their comments provided the
impetus behind nonrationality results described in this paper, for
it was not until then that I clearly understood that 
nonrationality cannot make any entrance into harmonic
analysis without being all-pervasive.

I dedicate this paper to the memory of L. Corwin, 
whose work with Howe on characters of
tamely ramified division algebras gave
some of the first general
insights into the characters of reductive
$p$-adic groups.

\section{The Construction of Hyperelliptic Curves} %2.

The argument that certain Shalika germs are described
by hyperelliptic curves
is a long one.  The first step is made in \cite{H3}.  This
section presents a quick overview of the argument.  We show how
hyperelliptic curves are associated with some of the
classical groups.  In Section 3, we 
give a brief indication of the
connection between the curves constructed in this section and
the $p$-adic harmonic analysis on reductive groups.

We begin with a split reductive group $G$ defined over a
field $k$ whose characteristic is not two, and 
with a short simple root $\alpha$ of $G$.
In practice, $\alpha$ will usually be taken to be adjacent to a long root.
From this data we will construct a variety that turns
out to be an elliptic
or  hyperelliptic curve in a number of interesting situations.

We let $\B$ denote the variety of Borel subalgebras of the
Lie algebra of $G$.  Let $W$ denote
the Weyl group of $G$, which we identify with the $G$-orbits on
$(\B\times \B)$.   For $w\in W$, we write $(\B\times \B)_w$ for the
orbit corresponding to $w$.  
For every Borel subalgebra $B$ and Cartan subgroup $T$ whose Lie algebra
lies in $B$,
the Weyl group $W(T)$ of $T$ is identified with $W$ by
associating $(B,{}^w\! B)$, for $w\in W(T)$, with the 
corresponding $G$-orbit $(\B\times\B)_w$, for $w\in W$.
   The simple reflections correspond to the
orbits of dimension one more than the dimension of $\B$.

Let $\g$ denote the Lie algebra of $G$, and let $\tilde\g$ denote
the Springer-Grothendieck variety.  It consists of the subvariety
of $\g\times \B$ formed by pairs $(X,B)$, where $X$ belongs to the
subalgebra $B$.  We let $\h$ denote the Cartan subalgebra of $\g$.
By definition, $\h$ is the Lie algebra $B/[B,B]$, where $B$ is a Borel
subalgebra, and $[B,B]$ is the commutator ideal.  
A different choice of Borel subalgebra in the definition of $\h$ leads
to a canonically isomorphic Lie algebra. 
We have a canonical morphism $\tilde \g\to \h$ that sends a pair $(X,B)$ in $\g$
to the image of $X\in B$ in $\h=B/[B,B]$.

We let $\tilde\g^{rs}$ denote the elements $(X,B)$ of $\tilde\g$ whose
first coordinate $X$ is a regular semisimple element of $\g$.
There is an action of the Weyl group on $\tilde\g^{rs}$.  The element
$w$ sends a pair $(X,B)$ to a pair $(X,B')$, where $B'$ is the unique
Borel subalgebra containing $X$ such that $(B',B)$ belongs to the
orbit $(\B\times\B)_w$.  To distinguish
this action from various other actions to be defined below we let
$\phi_w(X,B)$ denote the resulting pair.  The group $W$ acts naturally
on $\h$, and the morphism $\tilde\g^{rs}\to\h^{rs}$ is equivariant.  
We also
denote the action on $\h$ by $\phi_w$.

We introduce a formal parameter $t$ in the Cartan subalgebra as follows.
Let $k_1=k(\h)$ denote the field of rational functions on $\h$.  
The field contains the linear dual $\h^*$ of $\h$.
The Weyl group $W$ acts on $\h^*$ by $\langle\sigma_w\cdot X^*,\phi_w\cdot X\rangle
=\langle X^*,X\rangle$, where $X^*\in\h^*$, $X\in\h$, and $\langle\cdot,\cdot
\rangle$ is the canonical pairing.  This action extends to field
automorphisms $\sigma_w$ on $k_1$.  The desired formal parameter $t$
is then the canonical element of $\h\otimes_k\h^* \subseteq \h\otimes_k k_1
= \h(k_1)$
(the algebra of $k_1$ points of $\h$).  It equals
the element $\sum e_i\otimes e_i^*$, where $\{e_i\}$ is a basis of $\h$
and $\{e_i^*\}$ is the dual basis of $\h^*$.  Equivalently, it is
the $k_1$-point in $\h(k_1) = \hbox{Hom}_{k-\hbox{alg}}(k[\h],k(\h))$,
obtained by the canonical inclusion of the algebra $k[\h]$ of regular
functions on $\h$ into its quotient field of rational functions $k(\h)$.
The canonical element $t$ is invariant under the automorphism
$\phi_w\sigma_w$ of $\h(k_1)$.

The construction of the curve depends on a choice of a particular
nilpotent element $N$.  It is well known that if $N$ is a nilpotent
element of the Lie algebra, then 
$r=(\dim C_G(N) - \hbox{rank}(G))/2$ is a nonnegative integer, and we say that
$N$ is $r$-regular if this number equals $r$.  When $r=0$, we also
say that $N$ is {\it regular}, and
when $r=1$, the nilpotent element is also
called {\it subregular}.  
Over the algebraic closure $\bar k$ of $k$, there
is exactly one conjugacy class of regular nilpotent elements.  
Over $\bar k$, 
there is exactly one conjugacy class of subregular nilpotent
elements for each connected component of the Dynkin diagram of the
group.  We fixed, at the beginning of this section, a particular
simple root $\alpha$, which lies in
a particular component of the Dynkin diagram.  Hence the
given simple root $\alpha$ determines a subregular nilpotent element $N$ up to
stable
conjugacy.  We fix such an element $N$.  It is a $k$-point of the
Lie algebra $\g$.

The variety $\B_N$ of Borel subalgebras containing a subregular nilpotent element
$N$ is a union of projective lines.  Each projective line is associated
with a simple root.  There is precisely one line in $\B_N$ associated
with the short root $\alpha$.  Thus, $\alpha$ and $N$ determine a
projective line in $\B$, which we denote $\bP(\alpha,N)$.
We let $S$
be the surface $\bP(\alpha,N)\times \bP(\alpha,N)$
in $\B\times \B$.

For certain groups $G$ and simple roots $\alpha$,  we will
construct a rational function on (a twisted
form of) the surface $S$.
An irreducible component of the zero set of the rational
function
will be the hyperelliptic
curve.  
The rational function
is (more or less) of the form $\sigma(y)-y$, for
some involution $\sigma$ of the surface.
If we pick a local coordinate on $\bP(\alpha,N)$, 
then a point on $S$
is described by a pair $(x_1,x_2)$, and the coordinate
$y$ of the construction is $x_2-x_1$.
The key is to show
that the involution arises naturally.

We pause to clarify the notation.  The elements
$s_\beta$ represent the simple reflections of the Weyl group, and $w$
denotes a general element of $W$.  The corresponding
automorphisms of the field $k_1$
associated with $W$ are denoted $\sigma_\beta$ and $\sigma_w$.  The
automorphisms of the regular semisimple part $\tilde \g^{rs}$
of the Springer-Grothendieck
variety are denoted $\phi_{s_\beta}$ and $\phi_w$.  The same notation
is used for the action of $W$ on $\h$.
The automorphisms $\phi_w$ are defined over
$k$, so $\sigma_w$ and $\phi_w$ commute on the $k_1$-points of 
$\tilde \g^{rs}$.
 Finally, $\sigma^*_\beta$ and $\sigma^*_w$,
to be constructed in Lemma 2.2 below,
denote the birational action (composed with field automorphisms) of
$W$ on the surface $S$.  


Let $Z_\alpha$ denote the variety of regular triples:
$$Z_\alpha = \{(Y,B_1,B_2)\in \g^{rs}\times \B\times\B :
   Y\in B_1\cap B_2,\ \phi_{s_\alpha}(Y,B_1) = (Y,B_2)\}.
$$
Let $(B_1,B_2)$ be an element of $S(k_1)$ with $B_1\ne B_2$.
We select a 
deformation $(N(\lambda),B_1(\lambda),B_2(\lambda))$, 
for $\lambda\in k^\times$, of $(N,B_1,B_2)$ satisfying
the following properties:
\begin{enumerate}[label=(\arabic*)]
\item $(N(\lambda),B_1(\lambda),B_2(\lambda))$, for $\lambda\ne0$,
belongs to $Z_\alpha(k_1)$.
\item The center $(N(0),B_1(0),B_2(0))$ of the deformation is $(N,B_1,B_2)$.
\item The image of $(N(\lambda),B_1(\lambda))$ in $\h(k_1)$, under the
canonical map $\tilde \g^{rs}\to\h$ is $\lambda t$, where $t$ is the
canonical element of $\h(k_1)$.
\end{enumerate}
\par
\noindent
Such deformations exist.

Now fix an element $w$ in the Weyl group.  Define $B_1'(\lambda)$ and
$B_2'(\lambda)$ by the conditions $\phi_w(N(\lambda),B_1(\lambda)) =
(N(\lambda),B_1'(\lambda))$ and $\phi_{s_\alpha}(N(\lambda),B_1'(\lambda))=
(N(\lambda),B_2'(\lambda))$. Let $(N,B_1',B_2')$ denote the center
$(N(0),B_1'(0),B_2'(0))$ of the resulting deformation 
$(N(\lambda),B_1'(\lambda),B_2'(\lambda))$.

\proclaim{Lemma 2.1}  On a Zariski open set of $S$, the pair $(B_1',B_2')$ is
independent of the choice of deformation.
\endproclaim

\proclaim{Notation}  We define $\sigma^*_w (B_1,B_2) = (\sigma_w B_1',\sigma_w B_2')$,
for $(B_1,B_2)$ in a Zariski open set of $S/k_1$.
\endproclaim

\proclaim{Lemma 2.2}  $(B_1,B_2) \mapsto \sigma^*_w (B_1,B_2)$ defines an action
of the Weyl group (by birational maps composed with field automorphisms of $k_1$).
\endproclaim

\begin{proof}[Proof (Lemmas 2.1 and 2.2)] 
The definitions show directly that
$\sigma^*_1 (B_1,B_2)$ does not depend on the choice of deformation and
that $\sigma^*_1 (B_1,B_2) = (B_1,B_2)$.

We will show below that Lemma 2.1 holds when $w$ is a simple reflection.
Assuming this result for the moment, we prove Lemma 2.1 by induction on
the length $\ell(w)$ of $w$.  Assume $\ell(w)\ge 2$.  Pick a simple
reflection $s_\beta$ such that $w' = s_\beta w$ has length less than $\ell(w)$.
Let $(N(\lambda),B_1'(\lambda),B_2'(\lambda))$ be 
associated as above 
with a deformation $(N(\lambda),B_1(\lambda),B_2(\lambda))$
and the Weyl group element $w$.  Let $(N(\lambda),B''_1(\lambda),B''_2(\lambda))$
be associated with the same deformation and the Weyl group element $w'$.
Set $(B_1'',B_2'') = \sigma^*_{w'}(B_1,B_2)$.  By our induction hypothesis,
it is independent of the choice of deformation, and is defined on a Zariski
open set.  We show that the center of $(N(\lambda),B_1'(\lambda),B_2'(\lambda))$
is $\sigma_w^{-1}(\sigma^*_\beta(B_1'',B_2''))$, which, by the
induction hypothesis, is indeed
independent of the choice of deformation.   To calculate
$\sigma^*_\beta(B_1'',B_2'')$ we use the deformation
$$(\sigma_{w'}N(\lambda),\sigma_{w'}B_1''(\lambda),\sigma_{w'}B_2''(\lambda)).$$
This satisfies Conditions 1, 2, and 3 given above.
Then
\begin{align*}
\phi_{s_\beta}(\sigma_{w'}N(\lambda),\sigma_{w'}B_1''(\lambda))&=
      \sigma_{w'}B_1'(\lambda),\\
\phi_{s_\alpha}(\sigma_{w'}N(\lambda),\sigma_{w'}B_1'(\lambda))&=
      \sigma_{w'}B_2'(\lambda).
\end{align*}
So $\sigma_\beta^* (\sigma^*_{w'} (B_1,B_2))$ is the center of 
$\sigma_\beta\sigma_{w'}(N(\lambda),B_1'(\lambda),B_2'(\lambda))$ as desired.

It is now a formality that we have a group action.  Take $w'$ and $w$.
Write $\sigma^*_{w'} (B_1,B_2) = (B_1'',B_2'')$, obtained by a deformation
$(N(\lambda),B_1(\lambda),B_2(\lambda))$.  Use the same deformation to
compute $\sigma^*_{ww'} (B_1,B_2)$, and use the deformation 
$$(\sigma_{w'}N(\lambda),\sigma_{w'}B_1''(\lambda),\sigma_{w'}B_2''(\lambda))$$
to compute $\sigma^*_w (B_1'',B_2'')$, where
$$\phi_{w'}(N(\lambda),B_1(\lambda)) = (N(\lambda),B_1''(\lambda))
\hbox{ and } \phi_{s_\alpha}(N(\lambda),B''_1(\lambda)) = (N(\lambda),B_2''(\lambda)).$$
With these choices, composition holds.

We have thus reduced to Lemma 2.1 for simple reflections.
Suppose that the simple root is $\alpha$.  Then from the definitions it follows
easily that $\sigma^*_\alpha (B_1,B_2) = (\sigma_\alpha B_2,
\sigma_\alpha B_1)$.  For the other
simple reflections $s_\beta$, the calculation essentially reduces to the
rank-two system containing the roots $\alpha$ and $\beta$.  For instance,
if $s_\beta$ and $s_\alpha$ commute then we find that 
$\sigma^*_\beta (B_1,B_2) = (\sigma_\beta B_1,
\sigma_\beta B_2)$.  
The explicit rank-two relations are obtained
in a slightly different
context in \cite{H3,VI.1.6}.
Particular cases are found in Examples 2.3 and 2.6.
Details are left to the reader.
\end{proof}

\proclaim{Example 2.3}
  Consider the group $G=Sp(6)$.  We have three
simple roots $\beta_1 = 2t_1$,
$\beta_2 = t_2-t_1$, and $\beta_3 = t_3-t_2$.  We let the fixed simple
root $\alpha$ of the construction be $\alpha=\beta_2$.  
The line
$\bP(\alpha,N)$ intersects three other projective lines in $\B_N$, say
$\bP_1$, $\bP_2$, and $\bP_3$.  We assume that the nilpotent element
is {\it distinguished} in the sense that the lines $\bP_1$, $\bP_2$, and
$\bP_3$ are defined over $k$.  Assume that the lines $\bP_1$, $\bP_2$, and
$\bP_3$ are of types $\beta_1$, $\beta_2$, and $\beta_1$, respectively.
Select a coordinate $x$ on $\bP(\alpha,N)$ 
in such a way that the intersections
of $\bP(\alpha,N)$ with the lines $\bP_1$, $\bP_2$, and $\bP_3$ have
coordinates $x=0,1$, and $\infty$, respectively.  Then on an open set of $S =
\bP(\alpha,N)\times \bP(\alpha,N)$ we may choose the coordinate $x$
for $B_1$ and $x+y$ for $B_2$.  An explicit calculation shows that
the birational map $(B_1,B_2)\mapsto \sigma_\beta^{-1}\sigma^*_\beta(B_1,B_2)$
is given for simple roots $\beta$ by
\begin{equation}\tag {2.4}
\begin{aligned}
(x,y) &\mapsto \left(x, {(t_1+t_2) y x\over  x (t_2-t_1) -  t_1 y}\right),
    \qquad \hbox{if } \beta=\beta_1,\\
      &\mapsto \left(x+y,-y\right),\qquad \hbox{if } \beta=\beta_2,\\
      &\mapsto \left(x,{(t_3-t_1)y\over (t_2-t_1) + y (t_3-t_2)/(1-x)}\right),
      \qquad \hbox{if } \beta=\beta_3.
\end{aligned}
\end{equation}
In this simple case, it can be verified directly from these expressions
and the Coxeter relations of the group
that $(B_1,B_2)\mapsto \sigma^*_w(B_1,B_2)$ defines a Weyl group action.
\endproclaim

Let $w_-$ be the element of longest length in $Sp(6)$. 
Write $\sigma_-$ and $\sigma_-^*$ for $\sigma_{w_-}$ and
$\sigma^*_{w_-}$.
The element $\sigma_-$  acts on $t_i$,
for $i=1,2,3$, by $\sigma_{-}(t_i) = -t_i$.  Let $k_1^-$ be the
field fixed by $\sigma_{-}$ in $k_1$.  We create a twisted form
$S^*$ of $S$, defined over $k^-_1$, by using the twisted action $\sigma_{-}^*$
of the nontrivial element of the 
Galois group of $\Gal(k_1/k_1^-)$.  
Since the maps of  Lemma 2.2 are only birational, the surface
$S^*$ is only defined up to birational equivalence.
By writing $w_-$ as a product of simple reflections,
Equations 2.4 may be used to
find explicit rational functions $X(x,y)$ and $Y(x,y)$ such that
$$\sigma_{-}(x) = X(x,y),\qquad \sigma_{-}(y) = Y(x,y)$$
for every rational point $(x,y)$ on the twisted form $S^*(k_1^-)$.
In fact, the longest element is $w_- = s_3s_2s_1s_2s_3s_2s_1s_2s_1$,
where $s_i = s_{\beta_i}$.
The rational functions $X$ and $Y$ are given in the appendix.

\proclaim{Lemma 2.5} Assume the nondegeneracy condition $t_i^2\ne t_j^2$,
for $i\ne j$. An irreducible component of $Y(x,y) - y=0$ is an elliptic
curve with $j$-invariant
$$32 \left( \sum_{i<j} (\mu_i-\mu_j)^2\right)^3/\prod_{i<j} (\mu_i-\mu_j)^2,$$
where $\mu_i = 1/t_i^2$, $i=1,2,3$.  The other irreducible components
of $Y(x,y)-y = 0$
define rational curves.
\endproclaim

Bernstein and Kazhdan construct an elliptic curve
in $Sp(6)$ with this $j$-invariant by completely different methods.
The elliptic curve $y^2 = (1-x t_1^2 )(1-x t_2^2 )(1-x t_3^2)$ given
in $Sp(6)$ in Section 3 also has this $j$-invariant.

\begin{proof}[Proof (Sketch)]  The surface $S$ is
defined only up to birational equivalence, but the birational class of each
irreducible component 
of the divisor $Y(x,y)-y$ is well-defined.
  Starting with the explicit formula for the
rational function $f(x,y) = Y(x,y) - y$ in the appendix, we write
$x = 1 - y/x_1$ to obtain a rational function $f_1(x_1,y) = f(x,y)$.
One of the irreducible polynomial factors of $f_1(x_1,y)$ is 
of degree four in $x$ and quadratic
in $y$.  Completing the square (by a substitution $y_1 = f_2(x_1) + f_3(x_1) y$,
for suitable polynomials $f_2$ and $f_3$) we rewrite $f_1(x_1,y)$ in the
form
$$y_1^2 - (a_0 x_1^4 + a_1 x_1^3 + a_2 x_1^2 + a_3 x_1 + a_4).$$
The constants $a_i$ are found by direct calculation to be
\begin{align*}
a_0 &= t_2^2 (-t_2 + t_3)^2,\\
a_1 &= 4 t_2 (-t_2 + t_3) (t_2^2 + t_1 t_3),\\
a_2 &= 2 (-t_1^2 t_2^2 + 3 t_2^4 - 3 t_1^2 t_2 t_3 + 3 t_1 t_2^2 t_3
           + 2 t_1^2 t_3^2 - 3 t_1 t_2 t_3^2 - t_2^2 t_3^2),\\
a_3 &= 4 (t_1-t_2)(t_1 t_2^2 + t_2^3 - t_1^2 t_3 + t_1 t_2 t_3 + t_2^2 t_3
       - t_1 t_3^2),\\
a_4 &= (-t_1+t_2)^2 (t_1+t_2+t_3)^2.
\end{align*}
A standard formula for the $j$-invariant of an elliptic curve gives
the $j$-invariant of the lemma.   
The other irreducible components of $Y(x,y) - y=0$ are
conics.
\end{proof}

\proclaim{Example 2.6} The second example we consider is $G=SO(8)$.  We write
the four simple roots as $\beta_4 = t_4-t_3$, $\beta_3 = t_3-t_2$,
$\beta_2 = t_2-t_1$, and $\beta_1 = t_2+t_1$.  We select
$\alpha=\beta_3$ as the distinguished
simple root of the construction.  The line
$\bP(\alpha,N)$ intersects three projective lines as before, and we
arrange things so that the intersections of $\bP(\alpha,N)$ with the
lines of types $\beta_4$, $\beta_2$, and $\beta_1$ have coordinates
$0$, $1$, and $\infty$, respectively. Fix
coordinates
 $x$ and $x+y$ for $B_1$ and $B_2$ as in Example 2.3.
Direct calculation shows that the birational maps $(B_1,B_2)\mapsto
\sigma^{-1}_\beta\sigma^*_\beta(B_1,B_2)$, for $\beta$ simple, are
given by
\begin{align*}
(x,y) &\mapsto \left(x,{(t_4-t_2) y\over (t_3-t_2) + y (t_3-t_4)/x}
  \right),\ \hbox{ if } \beta=\beta_4,\\
      &\mapsto \left(x+y,-y\right),\ \hbox{ if } \beta=\beta_3,\\
      &\mapsto \left(x,{(t_3-t_1) y\over (t_3-t_2) + y (t_1-t_2)/(x-1)}
  \right),\ \hbox{ if } \beta=\beta_2,\\
      &\mapsto \left(x,{(t_3+t_1)y\over (t_3-t_2)}\right),\ \hbox{ if } 
    \beta=\beta_1.
\end{align*}
\endproclaim

We proceed as before, constructing the longest element 
$$w_- = s_3 s_4 s_1 s_3 s_2 s_1 s_3 s_1 s_4 s_3 s_1 s_2$$
(with $s_i = s_{\beta_i}$), the
field $k_1^-$, and the rational function $Y(x,y) = \sigma_{-} (y)$.

\proclaim{Lemma 2.7}  Assume the nondegeneracy condition
$t_i^2\ne t_j^2$, for $i\ne j$.
An irreducible component of $Y(x,y)-y=0$ is an elliptic
curve with $j$-invariant
$${256 \left(6\mu_1\mu_2\mu_3\mu_4-\sum_{i<j<k} (\mu_i+\mu_j+\mu_k)(\mu_i \mu_j \mu_k)
        + \sum_{i<j} \mu_i^2\mu_j^2\right)^3\over
(\mu_1-\mu_2)^2 (\mu_1-\mu_3)^2 (\mu_1-\mu_4)^2 (\mu_2-\mu_3)^2
 (\mu_2-\mu_4)^2 (\mu_3-\mu_4)^2},$$
where $\mu_i = t_i^2$, for $i=1,2,3,4$.
The other irreducible components of $Y(x,y)-y$ are rational.
\endproclaim

The elliptic curve $\epsilon y^2 = (1- x \tau_1^2)(1-x \tau_2^2)(1-x \tau_3^2)
(1-x \tau_4^2)$, with $\epsilon \in k^\times\setminus k^{\times\,2}$,
 has the same $j$-invariant, if we set $\mu_i=\tau_i^2$.  
This elliptic curve
is the curve that should appear in
$Sp(8)$ by the results of Lemma 4.10.  This
might well be expected,
since $SO(8)$ is an endoscopic group of
$Sp(8)$.

\begin{proof}  The same argument as in the proof of the previous lemma 
gives an elliptic curve.  The calculations are even longer in this
case.  As with Example 2.3,
they were obtained by computer and {\it Mathematica}.
The elliptic curve obtained by completing the
square is
$$y_1^2 = a_0 x_1^4 + a_1 x_1^3 + a_2 x_1^2 + a_3 x_1 + a_4,$$
with
\begin{align*}
% OLD TRY:
%a_0 &=	(t_1+t_3)^2 (-t_3+t_4)^2,\\
%a_1 &=  4 (t_1+t_3) (-t_3+t_4)(-t_1 t_2 + t_3^2 - t_1 t_4 + t_2 t_4),\\
%a_2 &= 2 (2 t_1^2 t_2^2 - 3 t_1^2 t_2 t_3 + 3 t_1 t_2^2 t_3 - t_1^2 t_3^2 
%      - 3 t_1 t_2 t_3^2 - 
%     t_2^2 t_3^2 + 3 t_3^4 \\
%      &\quad + 3 t_1^2 t_2 t_4 - 3 t_1 t_2^2 t_4 - 3 t_1^2 t_3 t_4 + 
%     6 t_1 t_2 t_3 t_4 - 3 t_2^2 t_3 t_4 - 3 t_1 t_3^2 t_4 + 3 t_2 t_3^2 t_4 \\
%      &\quad + 
%     2 t_1^2 t_4^2 - 3 t_1 t_2 t_4^2 + 2 t_2^2 t_4^2 + 3 t_1 t_3 t_4^2 - 3 t_2 t_3 t_4^2 - 
%     t_3^2 t_4^2),\\
% 
%a_3 &=  4(t_2-t_3) (-t_1^2 t_2 + t_1 t_2^2 -t_1 t_2 t_3 - t_1 t_3^2
%        + t_2 t_3^2 + t_3^3 \\
%        &\quad - t_1^2 t_4 + t_1 t_2 t_4 - t_2^2 t_4
%        - t_1 t_3 t_4 + t_2 t_3 t_4 + t_3^2 t_4 + t_1 t_4^2 - t_2 t_4^2),\\
% 4*(t2 - t3)*(-(t1^2*t2) + t1*t2^2 - t1*t2*t3 - t1*t3^2 + t2*t3^2 + t3^3 - 
%     t1^2*t4 + t1*t2*t4 - t2^2*t4 - t1*t3*t4 + t2*t3*t4 + t3^2*t4 + t1*t4^2 - 
%     t2*t4^2)
%
%a_4 &= (-t_2+t_3)^2(-t_1+t_2+t_3+t_4)^2.
%
%
%
% NEW TRY:
 a_0 &= (-t_1+t_2)^2 (t_1+t_3)^2,\\
 a_1 &= 4(t_1-t_2)(t_1+t_3)(t_1 t_2-t_1 t_3 + t_2 t_3 - t_4^2),\\
 a_2 &= 4(t_1^2 t_2^2 - 3t_1^2t_2t_3 + 3t_1t_2^2 t_3 + t_1^2 t_3^2-3t_1t_2t_3^2
        + t_2^2t_3^2 + t_1^2t_4^2 - 3t_1 t_2t_4^2 \\
      &\hbox{}\qquad + t_2^2 t_4^2 + 3t_1t_3t_4^2 - 3t_2t_3t_4^2+t_3^2t_4^2),\\
 a_3 &= 8(-t_2+t_3)(t_1t_2t_3 - t_1t_4^2 + t_2t_4^2 - t_3t_4^2),\\
 a_4 &= 4(-t_2+t_3)^2t_4^2.
\end{align*}
A standard formula for the $j$-invariant of an elliptic curve leads to
the given $j$-invariant of the lemma.
\end{proof}

\proclaim{Example 2.8}  Consider the group $G=SO(2n+1)$.  We have the simple
roots $\beta_n = t_n-t_{n-1}$, $\ldots,$ $\beta_2 = t_2-t_1$, and
$\beta_1 = t_1$.  Let the distinguished simple root $\alpha$ be $\beta_1$.
The line $\bP(\alpha,N)$ intersects two other projective lines
$\bP_1$ and  $\bP_2$ in $\B_N$, and both of these lines are of type $\beta_2$.
Assume that $N$ is chosen in such a way that the lines $\bP_1$ and
$\bP_2$ are defined over $k$.  Select a coordinate $x$ on $\bP(\alpha,N)$
in such a way that $x=0$ and $x=\infty$ define the intersection
of $\bP(\alpha,N)$ with $\bP_1$ and $\bP_2$.  Let $x$ and $x+y$ be the
coordinates of $B_1$ and $B_2$.  The birational map $(B_1,B_2)\mapsto
\sigma_\beta^{-1}\sigma_\beta^*(B_1,B_2)$ is given for simple roots
$\beta$ by
\begin{align*}
(x,y)&\mapsto (x+y,-y),\quad\hbox{if } \beta=\beta_1,\\
     &\mapsto \left(x ,{2t_2xy\over 2t_1x-t_2y+t_1y}\right),
              \quad\hbox{if } \beta=\beta_2,\\
     &\mapsto (x,y),\qquad\hbox{if } \beta\ne \beta_1,\beta_2.
\end{align*}
$k[\h]$ is a graded ring.  Let $t_{x}\in k[\h]\subset k_1$ be an
element of degree two that is Weyl group invariant (say
$t_1^2+\cdots+t_n^2$).  Let $k_2/k_1$
be the quadratic field extension $k_2 = k_1[t_{0}]/(t_0^2 - t_{x})$.
The Weyl group action extends to $k_2$ with trivial action on $t_{0}$.
The group of order two with generator $\sigma_0$ acts on $k_2$ by
restricting trivially to $k_1$ and $\sigma_0(t_0)= -t_0$.  We extend
the action of Lemma 2.2 to $S(k_2)$.  Let $k_0\subset k_1$ be the
fixed field of $W$.

The centralizer $C_G(N)$ has two components.  The group $C_G(N)$ acts
on $\B_N$, on $\bP(\alpha,N)$, and diagonally on $S = \bP(\alpha,N)\times
\bP(\alpha,N)$.  This action commutes with the action of Lemma 2.2
and with the conjugation $\sigma_0$.  Pick an involution $\iota$
of $S$ coming from the nonneutral component of $C_G(N)$.
It has the form
$$
(x,y)\mapsto \left({1\over cx},{-y\over c x(x+y)}\right),
$$
for some constant $c\ne0$ depending on the choice of $\iota$.
We choose $\iota$ in such a way that $c=1$.
  Let
$\sigma_0^*$ be the involution $\iota\sigma_0$ of $S(k_2)$.
We have an action of $\Gal(k_2/k_0)$ on $S(k_2)$ by the maps
$\sigma^*_w$ and $\sigma^*_0$.  By Galois descent, we obtain
a variety $S^*$ defined over $k_0$ (or at least the birational
equivalence class of a variety, since the Weyl group acts only
by birational maps).   

There are coordinates $y_1$ and $x_1$ on $S^*$ with the property
that $x_1$ is a $k_0$-variable, $y_1$ is a $k_0(t_0)$-variable,
and the condition for $(x_1,y_1)$ to define a point of $S^*(k_0)$
is 
\begin{equation}\tag{2.9}
y_1 \sigma_0(y_1) = (1- x_1^2 \tau_1^2)\cdots (1-x_1^2 \tau_n^2),
\end{equation}
where $\tau_i = t_i/t_0$.  We do not give a proof of these relations here.
They are proved in a slightly different context in \cite{H5,1.2}.
We conclude that at points of $S^*(k_0)$, the rational function
$\sigma_0(y_1) - y_1$ 
vanishes along the hyperelliptic
curve $y_1^2 = f(x_1)$, where $f(x_1)$ is given by the
right-hand side of Equation 2.9.
\endproclaim

%3.
\section{Relation to Shalika Germs}

In this section we sketch the connection between Shalika germs and
varieties over finite fields.  As we have already mentioned, many of the
proofs are long and are not given here.  The purpose of this section
is merely to orient the reader to a suitable
conceptual framework.  This
framework is provided by the work of Denef,
Igusa, Langlands, Shelstad, and others.

Let $F$ be a $p$-adic field of characteristic zero.  Igusa \cite{I}
established
a general
asymptotic expansion for
families of  integrals over $p$-adic
manifolds.
The asymptotic expansion of the family of integrals $I(\lambda)$,
depending on a parameter $\lambda\in F^\times$,
takes  the general
form of a {\it finite} sum of quasicharacters 
$\theta:F^\times\to \C^\times$ and powers of logarithms:
$$I(\lambda) \sim \sum_\theta \theta(\lambda) (\log|\lambda|)^m
c_{\theta,m}$$
as $\lambda$ tends to zero.
The expansion actually gives an exact formula for $I(\lambda)$
when $\lambda$ is  sufficiently small.
For each quasicharacter
$\theta_0$, Igusa considers
the Mellin transform 
$$Z(\theta_0,s)= 
     \int_{|\lambda|\le \epsilon} I(\lambda) \theta_0(\lambda)|\lambda|^s 
    \,|d\lambda|,\ \ \hbox{for } s\in\C\ \ \hbox{and}\quad  \Re(s)\gg 0,$$
for any small positive constant $\epsilon$.
The Mellin transforms $Z(\theta_0,s)$ are rational functions of 
$q^{-s}$.  As $\theta_0$ varies, the Mellin transforms determine
the original asymptotic expansion.
The functions $Z(\theta_0,s)$ are essentially
examples of {\it Igusa's  
local zeta functions}.
We refer the reader to Denef's survey article \cite{D2}.

Langlands and Shelstad
showed  how the asymptotic expansion of Shalika 
fits naturally into
the framework of Igusa's theory.  In Shalika's germ
expansion there are no logarithmic terms: $c_{\theta,m}=0$,
for $m\ne0$. In their work,
the terms $c_{\theta,0}$ of the
original asymptotic expansion (and so also the Shalika germs)
are given as principal-value integrals
on $p$-adic manifolds $X$ (see \cite{L} and \cite{H3}).

Denef proved that a large class of Igusa zeta functions have
a description in terms of the number of points on varieties over
finite fields \cite{D1}.

If we formulate Denef's argument in terms that mesh with 
the treatment 
of Shalika germs given by Langlands and Shelstad, 
then the strategy
is to replace each of the $p$-adic
manifolds $X$ with a scheme $X_0$ that is proper over
$O_F$, the ring of integers of $F$.  The integration over the
analytic manifold $X$ is then to be replaced
 with integration over the $O_F$-points
of $X_0$.  Let $k$ be the residue field of $F$, and set
$\bar X_0 = X_0\times_{O_F} k$.  There is a canonical map
$\varphi:X_0(O_F) \to \bar X_0(k)$ to the set of points on the
variety over the finite field.  The integral of a measure $|\omega|$
over $X$ may then be written
$$\int_X |\omega| = \sum_{x\in \bar X_0(k)} \int_{\varphi^{-1}(x)} |\omega|.$$
If we knew, for instance, that the integral over the fiber $\varphi^{-1}(x)$
was a constant $c$ independent of $x$, then the integral over $x$
would 
become $c\,\, \#\bar X_0(k)$.  In general, the most we can hope for is that
the integral over the fiber will be independent of $x$,
 for $x$ in a 
Zariski open set $U\subset \bar X_0$. We then
repeat the argument on the complement of $U$ in $\bar X_0$.
 The integral over $X$ is 
then eventually expressed as 
the
sum of $c\,\,\#\bar X_0(k)$ and similar terms coming from a finite collection
of closed subvarieties of $\bar X_0$. 

This argument, when it can be carried out, expresses the Shalika
germs in terms of points on varieties over finite fields.  In the context
of Shalika germs, the varieties $X$ 
and the accompanying data depend on a parameter $t$ in the
Lie algebra, so that we actually obtain families of varieties.
Typically, the constants $c$, independent of $t$, are elementary
functions of $q$, the cardinality of the residue field.
This
paper deals with examples where $\bar X_0$ is a nonconstant
family of elliptic or hyperelliptic curves.
An important unsolved problem is to determine how
generally Denef's argument applies
to Shalika germs and other
basic objects of harmonic analysis on $p$-adic groups.
Are the Fourier transforms of nilpotent orbits and the Shalika germs
always representable by points on varieties over finite fields?

The surface $S$ of the previous section is birationally equivalent
to one of the irreducible components $S'$
of the surface $X$ that
arises in the study of Shalika germs of subregular unipotent elements
in a reductive group $G$.  The Weyl group acts on $S'$, and this permits
us to define a form of $S'$ for every Cartan subgroup $T$ of $G$.
The Shalika germ is expressed as an integral over the twisted form
of $S'$ and of the other irreducible components of the surface $X$.
The fields $k$ and $k_1$ of Section 2 are to be a $p$-adic field $F$
of characteristic zero, and a splitting field $T$ of the Cartan subgroup.

Examples 2.3 and 2.6 of the previous section stem from the elliptic
Cartan subgroup $T$ split by a ramified quadratic extension $E/F$,
obtained by twisting the split Cartan subgroup by the longest element
$w_-$.  In this case, we may specialize the parameters $t_i$ so that
$t_i/t_j\in F$ and $t_i\in E$.
Assume that the parameters $t_i$ all have the
same half-integral valuation, say $|t_i| = q^{-1/2}$, for $i=1,2,3$.
Let $\tau_i$ be the residue of the unit $t_i/\sqrt{\pi}$, where
$\pi$ is a uniformizing parameter.
The rational function $Y(x,y)$ descends to a rational
function on the finite residue field, again denoted
$Y(x,y)$: take $x,y\in k$ and use the fact that
$Y(x,y)$ is homogeneous of degree zero in the coordinates
$t_i$ to replace each $t_i$ by $\tau_i$.
  Consider the expression
$$\sigma_{-}(y) - y = Y(x,y) - y,\qquad\hbox{for } \sigma_{-}\in \Gal(E/F).$$
If $E/F$ is ramified, then this expression vanishes
for $x$ and $y$ in the residue field $k$, because
$\sigma_{-}$ acts trivially on the
residue field. We conclude that 
$0 = Y(x,y)-y$,
for $x,y\in k$.

The calculations of Examples 2.3 and
2.6 descend to the finite field without difficulty and give us an
elliptic curve over the finite field.  
The $j$-invariants are obtained by replacing $t_i$ by
$\tau_i$ in Lemmas 2.5 and 2.7.
One part of
the integral over the surface $X$ 
is then a constant times the number
of points on the elliptic curve.  This gives one term
(the nonrational term) in the formula
for the subregular Shalika germ of orbital integrals. 

%4
\section{Examples}

We establish some notation and conventions for Sections
4-7.  Let $F$ be a $p$-adic field of characteristic
zero and odd residual characteristic, with ring
of integers $O_F$, algebraic closure $\bar F$, and
residue field $k$ of cardinality $q$.  Let $\pi$ be
a uniformizer in $F$. 

The Shalika germs, unless indicated to the contrary,
are to be the germs of stable orbital integrals 
normalized by the usual discriminant factor
(denoted $|\eta(X)|^{1/2}_{\p}$ in \cite{HC} and
$D_{G^*}(\exp(X))$, for $X$ sufficiently small,
 in \cite{LS}).  We write it $D(X)$.
The measures of orbital integrals are normalized as
in \cite{L}.  We take the Shalika germs to be homogeneous
functions on the Lie algebra, with the understanding
that the germ expansion holds only for regular
semisimple elements sufficiently close to the 
identity.  Since the discriminant factor $D(X)$ is
already included in the definition of germs, we omit
the discriminant factor $\Delta_{IV}$ from the
definition of the Langlands-Shelstad transfer factor
$\Delta$.


Let $G$ be a reductive group over $F$. More
specifically, we
assume that $G$ is one of the groups $SU_E(n)$, $Sp(2n)$, or $SO(n)$. We
assume that $G$ is defined by a matrix equation
$${}^tg\,J\,\sigma(g) = J,$$
where the entries of the matrix $J$ satisfy $J_{ik}=0$, for $i+k\ne n+1$,
and $J_{ik}= \pm 1$ along the skew-diagonal. (Replace $n$ by $2n$ if 
$G=Sp(2n)$.)
  We assume that the nonzero
entries of $J$ equal one, for the orthogonal group.  We take $g\in SL(n,F)$
for $SO(n)$, $g\in SL(2n,F)$ for $Sp(2n)$, and 
$g\in SL(n,E)$ for $SU_E(n)$.  We assume that $E/F$ is a fixed
{\it ramified} quadratic extension, and $\sigma(g)$ denotes the matrix
obtained by applying the nontrivial field automorphism of $E/F$ to each
coefficient of $g$.  
We take $\sigma(g) = g$ in the orthogonal and symplectic groups.

Each regular semisimple element of the Lie algebra $\g$ of $G$ is diagonalizable
over $\bF$.
We write 
$t=[t_1,\ldots,t_n]\in \g(\bF)$ for a diagonal element.  Specifically,
we let $[t_1,\ldots,t_n]$ denote the element
\begin{align*}
&\diag(t_1,\ldots,t_n),\quad \hbox{ if } G=SU_E(n),\\
&\diag(t_1,\ldots,t_n,-t_n,\ldots,-t_1),\quad \hbox{ if } G = Sp(2n) 
   \hbox{ or } G = SO(2n),\\
&\diag(t_1,\ldots,t_n,0,-t_n,\ldots,-t_1),\quad \hbox{ if } G = SO(2n+1).
\end{align*}

The elements $t_1,\ldots,t_n$ lie in $\bF$.  We assume that the image
of $[t_1,\ldots,t_n]$ in $\h/W$ (the diagonal subalgebra modulo the Weyl group)
is an $F$-point.  Extend the normalized absolute value on $F$ to $\bF$.

Most of our examples come from a particular set of elliptic semisimple elements
in the Lie algebra of our reductive group $G$, a set that we will call
the half-integral set.  We say that $t=[t_1,\ldots,t_n]$ is {\it half-integral\/}
if $$|t_i| = |\alpha(t)| = q^{-1/2},$$
for all $i$ and every root $\alpha$ of $\h$.  For example, if $G=Sp(2n)$,
this condition reads $|t_i|=|t_i\pm t_j| = q^{-1/2}$,
for all $i$ and $j$.  Given a half-integral element $t$, and a ramified
quadratic extension $E/F$, we define elements $\tau_i$ in 
the algebraic closure $\bk$ of the residue field $k$ of $F$.
Let $\tau_i$ be the residue of 
$t_i/\pi_E$, where $\pi_E$ is a uniformizer of $E$ whose
square lies in $F$.  A different choice of uniformizer changes
all  the elements
$\tau_i$ by the same element of $k^\times$.
We write $\tau_E(t) = [\tau_1,\ldots,\tau_n]$
for the array of elements obtained.
Fix a unit $\epsilon\in k^\times$ that is not a square
(occasionally we identify $\epsilon$ with a unit of $F$). 

Assume that $t$ is half-integral and that a ramified quadratic extension $E/F$
has been fixed.  The following hyperelliptic curves are then defined over $k$.
\begin{align*}
\epsilon y^2 &= ( 1 - x^2 \tau_1^2)(1 - x^2 \tau_2^2) \cdots (1-x^2\tau_n^2),
      \quad (G=SO(2n+1)),\tag{4.1}\\
\epsilon y^2 &= (1-x \tau_1^2)(1-x\tau_2^2)\cdots (1- x\tau_n^2),\quad
      (G=Sp(2n)),\tag{4.2}\\
\epsilon y^2 &= x (1-x \tau_1^2)(1-x \tau_2^2)\cdots (1-x \tau_n^2),\quad
      (G=Sp(2n)),\tag{4.3}\\
\epsilon y^2 &= (1 + x \tau_1)(1 + x \tau_2) \cdots (1 + x\tau_n),\quad
      (G=SU_E(n),\ \  n \hbox{ even}),\tag{4.4}\\
\epsilon y^2 &= (x + \tau_1)(x+ \tau_2)\cdots (x+\tau_n), \quad
      (G=SU_E(n),\ \  n \hbox{ odd}).\tag{4.5}
\end{align*}
We assume that the ramified extension $E/F$ used for
$\tau_E$ is the splitting field of $SU_E(n)$ in the
case of unitary groups.
The half-integrality  of $t$
implies that the polynomials on the
right-hand side of these equations have distinct roots. 
Let $P(x)$ be a polynomial of degree $d$ with distinct roots.
The genus $g$ of the complete nonsingular
curve associated with $y^2 = P(x)$ is the integral
part of $(d-1)/2$.  
By definition, a hyperelliptic curve is complete.
A second coordinate patch, which includes the points
at {\it infinity}, is described by the equation
$y_1^2 = P_1(x_1)$, where $y_1 = y/x^{g+1}$, 
$x_1 = 1/x$, and $P_1$ is the polynomial
$P_1(x) = x^{2g+2} P(1/x)$.

Let $C_i(\tau_E(t))$
denote the number of points in $k$ on the hyperelliptic curve
of Equation $4.i$, for $i=1,2,3,4,5$.  
The number $C_i(\tau_E(t))$ depends on $E$, but not
on the choice of uniformizer $\pi_E$ in $E$
(except for $C_5(\tau_E(t))$: see Lemma 4.15).  
The examples in this section
will show how certain Shalika germs on the half-integral set 
of $G$ are expressed
in terms of the number of points $C_i(\tau_E(t))$ on hyperelliptic curves.

\proclaim{Lemma 4.6}  Half-integral elements are regular elliptic.
\endproclaim

\begin{proof}  If $t$ is half-integral
and not elliptic, then there exists a nonempty subset $I\subset
\{1,\ldots,n\}$ and a choice of
signs $\epsilon_i=\pm 1$, for $i\in I$, such that
$\Gal(\bF/F)$ acts on and even acts transitively on the set
$\{t_i\epsilon_i : i\in I\}$ (see \cite{H5,\S4}).  
Fix a ramified quadratic extension $E/F$ and 
a uniformizing parameter $\pi_E\in E$.
  Write $\{t_i\epsilon_i/\pi_E : i\in I\} =
\{s_1,\ldots,s_\ell\}$.  The polynomial
$$(X-s_1)\cdots (X-s_\ell) = X^\ell + a_1 X^{\ell-1} + \cdots 
+ a_\ell$$
has coefficients in $O_E$.  If $\sigma$ is the nontrivial element of
$\Gal(E/F)$, then $\sigma(a_i) = (-1)^i a_i$. 
Since $E/F$ is ramified, $a_i$ is congruent to $(-1)^i a_i$ modulo
the maximal ideal of $O_E$.  If $\ell$ is odd, then
$a_\ell$ is congruent to zero, and this
contradicts the condition that $s_i$ are units.  If $\ell$ is even, then
the polynomial is congruent to a polynomial in $X^2$, and this means that
for every $i$, there exists a $j$ such that $s_i$ is congruent to $-s_j$,
or that $|t_i\epsilon_i + t_j\epsilon_j|< q^{-1/2}$.  For $G=SO(n)$
or $G=Sp(2n)$, clearly $t_i\epsilon_i + t_j\epsilon_j$ is a root $\alpha(t)$,
and this inequality contradicts the condition of half-integrality.

For $SU_E(n)$ we divide our set into positive and negative pieces:
$$\{s_1,\ldots,s_m\} =\!\{t_i/\pi_E : i\in I,\epsilon_i = +1\},\
  \{s_1',\ldots,s_m'\} =\!\{t_i/\pi_E: i\in I,\epsilon_i = -1\}.$$
The polynomials
$$P(X) = (X-s_1)\cdots (X-s_m)\quad\hbox{and}\ P'(X) = (X-s_1')\cdots (X-s_m')$$
are defined over $E$, and $\sigma(P) = P'$, for $\sigma$ nontrivial in $\Gal(E/F)$.
Thus, for every $i$ there exists a $j$ such that $s_i$ is congruent to $s_j'$,
so that we again reach the contradiction $|\alpha(t)|< q^{-1/2}$ for
some root $\alpha$. \end{proof}

\proclaim{Lemma 4.7}  Fix a ramified quadratic character $\eta$ of $F^\times$.
Let $P(x)$ be a polynomial with
coefficients in $O_F$.  Assume that
$P(x)$ reduces to a polynomial in the residue field whose roots are
distinct.  Assume that $P(0)$ is a unit and a square.
Then 
$$\int_{\bP^1(F)}\eta(P(x))\left| {dx\over x^2}\right| =
   {-C(P)\over q},$$
where 
$C(P)$ is the number of points on the
hyperelliptic curve over the residue field $k$ obtained by
reduction of the equation
$\epsilon y^2 = P(x)$.
\endproclaim

\begin{proof}  As in \cite{H5,2.2}, we may replace the integrand $\eta(P(x))$ by 
$\eta(P(x))-1$.  For each residual point $x_0\in\bP^1(k)$, let 
$U(x_0)$ denote the set of $x\in \bP^1(F) = \bP^1(O_F)$ that reduce to
$x_0$ under the canonical reduction map $\bP^1(O_F)\to \bP^1(k)$.  Then it is
easy to check that
$$\int_{U(x_0)} (\eta(P(x)) - 1) \left|{dx\over x^2}\right|$$
is equal to $0$, $-1/q$, or $-2/q$.  We obtain $0$ when
$\eta(P(x))=1$ on $U(x_0)$, that is, $P(x)$ is a unit and a square.
We obtain $-2/q$ when $\eta(P(x))=-1$ on $U(x_0)$, that is, $P(x)$ is
a unit but not a square.  And we obtain $-1/q$ when $P(x)$ lies in the
maximal ideal, giving a point of ramification on the hyperelliptic
curve over the finite field.  The points at {\it infinity} ($|x|>1$)
 are treated similarly
by using local coordinates $(x_1,y_1)=(1/x,y/x^{g+1})$, where $g$ is the genus
of the hyperelliptic curve.  The result now follows easily.\end{proof}

On the group $SO(2n+1)$, the subregular unipotent classes are
parameterized by the set $F^\times/F^{\times\,2}$.  For $a\in F^\times/
F^{\times\,2}$ and $t=[t_1,\ldots,t_n]$,
we let $\Gamma_a(t)$ be the stable
subregular
Shalika germ on the half-integral set of $SO(2n+1)$ associated
with the class parameterized by $a$.  The stable
Shalika germs are well-defined up to a scalar multiple that depends
on the normalization of measures.

\proclaim{Lemma 4.8}  With appropriate normalizations, the stable
subregular Shalika germs of $SO(2n+1)$ on the half-integral set
are
\begin{align*}
\Gamma_a(t) &= 1,\ \hbox{if $a$ has even valuation,}\\
\Gamma_a(t) &= C_1(\tau_E(t)),\ \hbox{if $a$ has odd valuation.}
\end{align*}
Here 
$E$ is the ramified extension 
$E=  F(\sqrt{a})$, and
$C_1(\tau_E(t))$ is the number of points on the hyperelliptic
curve in Equation 4.1.
\endproclaim

\begin{proof}  By \cite{H5,1.2}, the stable subregular Shalika germs are given by
the following principal-value integrals:
\begin{equation}
\tag{4.9}
\begin{aligned}
\Gamma_a(t) &= \int_{F} \log|(1-x^2 t_1^2)\ldots
(1-x^2 t_n^2)|\left| {dx\over x^2}\right|,\ \hbox{if } a=F^{\times\,2},\\
\Gamma_a(t) &= \int_{\Ima(a)}
        \eta_a((1-u^2 t_1^2)\cdots (1-u^2 t_n^2))\left| {du\over u^2}\right|,
        \ \hbox{if } a\ne F^{\times\,2},
\end{aligned}
\end{equation}
where $\eta_a$ is the quadratic character of $F^\times$
associated with  the extension $F(\sqrt{a})$, and $\Ima(a)$ is the
corresponding imaginary axis: $\sqrt{a}F \subset F(\sqrt{a})$.

If $a$ is even, then it is easy to see that these integrals are
independent of $t$ on the half-integral set.  
Explicitly, we have
\begin{align*}
\Gamma_a(t) &= n\int_F \log|(1-\pi x^2)| \left|{dx\over x^2}\right|=
 {n (q+1)\over q (q-1)}, \quad \hbox{if } a=F^{\times\,2},\\
\Gamma_a(t) &= {-1+(-1)^n\over q}, \quad \hbox{if } a\ne F^{\times\,2}.
\end{align*}
(Compare \cite{H5,2.8}.)  A change in normalizations gives 
$\Gamma_a(t)=1$.

If $a$ has odd valuation, then these integrals are determined on
the half-integral set by Lemma 4.7, under the substitution
$u=x/\pi_E $, where $\pi_E$ is a uniformizer in $E$.  Changing
normalizations, we may eliminate the constant $-1/q$ appearing
in Lemma 4.7.
The result follows. \end{proof}

\bigskip
For the group $\PSp(2n)$, the subregular unipotent classes are also
parameterized by the set $F^\times/F^{\times\,2}$.  For $a\in F^\times/
F^{\times\,2}$, we let $\Gamma_a(t)$ be the stable
subregular Shalika germ on the half-integral set 
attached to the
subregular unipotent element associated with $a$.  It is well-defined
up to a scalar multiple depending on the normalizations of measures.

\proclaim{Lemma 4.10}  Assume the conjectures of Kottwitz and Shelstad
on the transfer of orbital integrals.  With suitable normalizations,
the stable subregular Shalika germs of $\PSp(2n)$ on the half-integral
set are
\begin{align*}
\Gamma_a(t) &= \gamma_a +  C_2(\tau_E(t)),\quad
\hbox{if $a$ has even valuation,}\\
\Gamma_a(t) &= \gamma_a +  C_3(\tau_E(t)),\quad
\hbox{if $a$ has odd valuation.}
\end{align*}
Here $C_2(\tau_E(t))$ and $C_3(\tau_E(t))$ are the numbers of points
on the hyperelliptic curves defined by Equations 4.2 and 4.3,
and the $\gamma_a$ are (known) constants in $\C$
that are independent of $t$ (but dependent on $n$ and $q$).
  The ramified extension
$E$ is given by $E=F(\sqrt{\pi})$.
\endproclaim

\begin{proof}   For the stable germs, there is no harm in working on the
group $Sp(2n)$ instead of $\PSp(2n)$.
As \cite{H5,\S1} explains, the conjectures of Kottwitz and
Shelstad lead to a conjectural duality between groups of type $C_n$
and groups of type $B_n$.  Each stable Shalika germ on $B_n$ should
be a linear combination of stable Shalika germs on $C_n$, and each
stable Shalika germ on $C_n$ should be a linear combination of stable
Shalika germs on $B_n$.  We add superscripts $B$ and $C$ to the
stable
subregular germs on $B_n$ and $C_n$ to distinguish notation for these
two groups.  It is also convenient to begin with
the normalization
of germs given in \cite{H5,\S2}, which we distinguish by adding a bar:
$\bGamma_a^B$, $\bGamma_a^C$, and so forth.  For stable subregular
germs on the groups $B_n$ and $C_n$ of rank $n$, the theory of  
Kottwitz and Shelstad thus predicts the existence of an
invertible $4\times 4$ matrix relation:
\begin{equation}\tag {4.11}
\bGamma^C_a(t) = \sum_{a'} L^{(n)}_{a,a'}
  \bGamma^B_{a'}(t).
\end{equation}
The matrix $L^{(n)}$ is independent of $t$.  Equation 4.11
has been verified
in rank two (see \cite{H5,2.1}), and the matrix $L^{(2)}$ 
is uniquely determined.
We have
$$L^{(2)} = {1\over2}\begin{pmatrix} 1&1&1&1\\
  1&1&-1&-1\\
  1&-1&s_q&-s_q\\
  1&-1&-s_q&s_q\end{pmatrix}.
$$
Here $s_q=1$, if $-1$ is a square in $F$, and $s_q=-1$ otherwise.
The cosets have been put into the order $F^{\times\,2}$,
$\epsilon F^{\times\,2}$,
$\pi F^{\times\,2}$, $\pi\epsilon F^{\times\,2}$.

The stable Shalika germs $\bGamma_a^C$ and $\bGamma_a^B$, 
restricted respectively
to Levi factors $\G_m^{n-2}\times Sp(4)$ and 
$\G_m^{n-2}\times SO(5)$, reduce to the
stable Shalika germs on the rank-two groups $Sp(4)$ and $SO(5)$.
This reduction is compatible with the normalization that we have
given the germs.  Since the restrictions are already linearly
independent, the matrix $L^{(n)}$ is already determined by this
restriction.  Hence we must have $L^{(n)}=L^{(2)}$, for all $n$.  Thus,
Equation 4.11 gives an exact 
(conjectural) formula for the stable
subregular Shalika germs of $C_n$.  For instance,
picking representatives $1$, $\epsilon$, $\pi$, and 
$\pi\epsilon$ for the cosets of $F^\times/F^{\times\,2}$,
we should have
$$\bGamma_1^C = {1\over 2}(\bGamma_1^B + \bGamma_\epsilon^B +
 \bGamma_\pi^B + \bGamma_{\pi\epsilon}^B),$$
where $\bGamma_a^B$ is given by Equation 4.9.
  Lemma 4.8 shows that the term
${1\over 2}(\bGamma_1^B+\bGamma_\epsilon^B)$ contributes a constant
 to the formula of the lemma.  The term
${1\over 2}(\bGamma_1^B-\bGamma_\epsilon^B)$ is also constant on the
half-integral set. 

To prove the lemma, we must show that up to constants $d_0,d_1,d_2,d_3$ 
independent of $t$ we have
$\bGamma_\pi^B + \bGamma_{\pi\epsilon}^B = d_0+ 2d_2 C_2(\tau_E(t))$
and 
$\bGamma_\pi^B - \bGamma_{\pi\epsilon}^B = d_1+ 2d_3 C_3(\tau_E(t))$.
Set $E= F(\sqrt{\pi})$, and
$E'=F(\sqrt{\pi\epsilon})$.  
The normalization constants, which relate $\bGamma^B_a$ to the
normalization of Equation 4.9, and Equation 4.9 to the normalization
of Lemma 4.8, are the same for the cosets $a=\pi F^{\times\,2}$
and $a=\pi\epsilon F^{\times\,2}$.
By Lemma 4.8, it will suffice to show
that 
\begin{equation}\tag{4.12}
C_1(\tau_E(t)) + C_1(\tau_{E'}(t)) =2  C_2(\tau_E(t)),
\end{equation}
and
\begin{equation}\tag{4.13}
C_1(\tau_E(t)) - C_1(\tau_{E'}(t)) = 2 (C_3(\tau_E(t))-1-q).
\end{equation}
Underlying these identities is the fact that the Jacobian of the
curve $y^2 = (1-x^2\tau_1^2)\cdots (1-x^2\tau_n^2)$ is isogenous
to the product of the Jacobians of $y^2 = (1-x \tau_1^2)\cdots (1-x\tau_n^2)$ and $y^2 = x(1-x \tau_1^2)\cdots (1-x\tau_n^2)$.  

We check
the identities directly as follows.  To prove Identity 4.12, note that
each point $(x,y)$ of the curve $\epsilon y^2 = (1-x\tau_1^2)\cdots
(1-x\tau_n^2)$ leads to two points $(x_1,y_1) = (\pm\sqrt{x},y)$ on the
curve $\epsilon y_1^2 = (1- x_1^2\tau_1^2)\cdots (1-x_1^2 \tau_n^2)$,
if $x$ is a square,
or to two points $(x_1,y_1) = (\pm \sqrt{x/\epsilon},y)$ on the
curve $\epsilon y_1^2 = 
(1-x_1^2\tau_1^2\epsilon)\ldots(1-x_1^2 \tau_1^2\epsilon)$
otherwise.
We may assume $\tau_{E'}(t) = \epsilon \tau_E(t)$.
This proves Identity 4.12.

To prove Identity 4.13, we observe that there are two points on
$y_1^2 = (1-x_1^2\tau_1^2)\cdots (1-x_1^2\tau_n^2)$ given by
$(x_1,y_1) = (\pm \sqrt{x},y/x_1)$, or two points
on $\epsilon y_1^2 = (1-x_1^2\tau_1^2\epsilon)\cdots
   (1-x^2_1\tau_n^2\epsilon)$ given by $(x_1,y_1) 
= (\pm\sqrt{x/\epsilon},y/(x_1\epsilon))$,
for each point $(x,y)$ on $y^2=x(1-x\tau^2_1)\cdots (1-x\tau^2_n)$.
A twist $\epsilon y^2 = P(x)$
has $1+q-b$ points if $y^2 = P(x)$ has $1+q+b$ points.  This completes
the proof of Lemma 4.10. \end{proof}

\bigskip
In the adjoint group of $U_E(2k)$ there is a single 
subregular conjugacy class.  Stable germs may be grouped together
according to the unipotent classes in the adjoint group
\cite{H3,VII.5.4}.

\proclaim{Lemma 4.14}  Assume that $E/F$ is a ramified quadratic
extension.  With suitable normalizations, the stable subregular Shalika
germ of $U_E(2k)$ on the half-integral set is equal to
$C_4(\tau_E(t))$, the number of points on the hyperelliptic
curve of Equation 4.4.
\endproclaim

\begin{proof}
We deduce the lemma from
the results of Chapter VII of \cite{H3}.  The argument is nearly
identical to the argument given in \cite{H5,\S1} for the group $SO(2n+1)$.
In the notation of \cite{H3,page 124}, set $u = w/(-T_{-1}w+1)$. 
The structure constants $e(\alpha,\beta)$ appearing in \cite{H3,V.6}
are $e(\alpha_i,\alpha_{i+1})=1$ and $e(\alpha_{i+1},\alpha_i)=-1$.
 Then
an easy calculation using \cite{H3,VII.5.6} shows that $u$ is a Weyl group
invariant coordinate, and that $\sigma_0(u) = -u$, where $\sigma_0$
denotes the nontrivial element of $\Gal(E/F)$.  Let
$[b] = (1 + u T_1)\cdots (1+ u T_k)$, considered as an $\bF$-point
of $U_E(1)$ depending on $u$.  Consider the cocycle 
$a_\sigma$ of $\Gal(\bF/F)$
with values in $\bF$, given by \cite{H3,VII.5.6}. Then
$a_\sigma\,\sigma[b][b]^{-1}$ is equal to one, if the image
of $\sigma$ in $\Gal(E/F)$ is trivial, and is equal to
$$1/((1+u T_k)\cdots (1+u T_1)(1+u T_{-1})\cdots (1+u T_{-k}))$$
otherwise.  The nontrivial character of $H^1(\Gal(\bF/F),U_E(1))$
evaluated on $a_\sigma$ is then
$$\eta((1+u T_k)\cdots (1+u T_{-k})),$$
where $\eta$ is the quadratic character of $F^\times$
attached to $E/F$.  Shifting from the notation $T_i$ of \cite{H3}
back to our notation $t_i$, we find that this expression is equal to
$\eta((1+ u t_1)\cdots (1 + u t_n))$
(because $\{t_1,\ldots,t_n\}= \{T_k,\ldots,T_{-k}\}$).

Now $|dw/w^2| = |du/u^2|$, and the stable germ is
$${1\over 2} |\lambda| \int_{\Ima}
\eta((1+ u t_1)\cdots (1+u t_n)) \left|{du\over u^2}\right |,$$
where $\Ima$ is the set of elements $u\in E$ that satisfy
$\sigma_0(u) = -u$.
On the half-integral set, we take $\lambda=1$ and evaluate
by Lemma 4.7 to obtain the theorem.
\end{proof}

\bigskip
Now consider the group $G=U_E(2k+1)$.  On its adjoint group
there are two subregular conjugacy classes, but their stable
germs differ only by a sign if the appropriate normalizations of measures
are chosen. (Compare \cite{H2,\S5}.)

\proclaim{Lemma 4.15}  Assume that $E/F$ is  a ramified quadratic
extension.  With a suitable normalization of measures, the
stable subregular Shalika germ of $U_E(2k+1)$ on the half-integral
set is equal to $1+q-C_5(\tau_E(t))$, where
$C_5(\tau_E(t))$ is the number of points on the
hyperelliptic curve of Equation 4.5.
\endproclaim

As we have noted, the constant $C_5(\tau_E(t))$
depends on the choice of a uniformizer $\pi_E$.  The lemma
holds for every choice of uniformizer if we adapt the sign of the 
normalization of measures to $\pi_E$. In fact, if $\pi_E' = \pi_E\epsilon$,
then with obvious notation $1+q-C_5(\tau_E(t)) =
-(1+q-C_5(\tau_E'(t)))$. 

\begin{proof}  The argument is similar to the proof of the
previous lemma.  The structure constants are $e(\alpha_i,\alpha_{i+1})=1$
and $e(\alpha_{i+1},\alpha_i)= -1$ as
before.  We define $u$ by $u=w/(-T_0 w+1)$.  It is a Weyl group
invariant coordinate, and $\sigma_0(u) = -u$ as before.
This time we adjust the cocycle $a_\sigma$ by $\sigma[b][b]^{-1}$,
where $[b]^{-1} = (1+ u T_0)\cdots (1+ u T_{-k})$.  We find that
the nontrivial character of $H^1(\Gal(\bF/F),U_E(1))$ evaluated
on $a_\sigma$ is equal to 
$$\eta(x(\gamma))\eta((1+u T_k)\cdots (1+u T_0)\cdots (1+ u T_{-k})) =
\eta(x(\gamma) u (1+u t_1)\cdots (1+u t_n)),$$
where $\eta$ is the character of $F^\times$ attached to $E$
and $x(\gamma)$ is a constant in $F^\times$.  
The formula of \cite{H3,VII.5.5} now shows that the germ is
$${1\over 2}\eta(x(\gamma))\eta(\lambda)|\lambda|\int_{\Ima(a)}
   \eta(u (1+ u t_1)\cdots (1+u t_n))\left| {du\over u^2}\right|.$$
$\eta(x(\gamma))$ is a sign depending on the subregular unipotent class.
On the half-integral set, we take $\lambda=1$, and we use
an adaptation of Lemma 4.7 to show that the germ is a multiple
of the difference of $1+q$ and
the number of points on the hyperelliptic curve
$$\epsilon y^2 = x (1+ x\tau_1)\cdots (1+x \tau_n).$$
In fact, Lemma 4.7 does not hold because $P(0)$ is not a unit, and
$$\int_{U(0)} \eta(P(x))-1 \left| {dx\over x^2}\right | = 1 = 
{q+1\over q} - {1\over q},$$
instead of the constant $-1/q$ occurring
in the proof of the lemma.  We pick up the term $(q+1)/q$, so the
integral is $q^{-1}$ times $1+q-C_5(\tau_E(t))$.
Under the substitution $(x,y)\mapsto (1/x,y/x^{(n+1)/2})$, we
obtain the curve
$\epsilon y^2 = (x+\tau_1)\cdots (x+\tau_n)$
as desired.\end{proof}

\bigskip
The groups $Sp(6)$ and $SO(8)$ are treated in the following
lemma.  We do not give a proof here.  The starting point
of the proofs is contained in Lemmas 2.5 and 2.7.
Recall that the subregular unipotent classes in $\PSp(6)$
are parameterized by $F^\times/F^{\times\,2}$.  In $SO(8)$
there is a single subregular unipotent class.

\proclaim{Lemma 4.16}  Let $E/F$ be a ramified quadratic
extension.  Let $T$ belong to  the stable conjugacy
class of Cartan subgroups
(in $G= Sp(6)$ or $SO(8)$) that splits over $E$ and
is obtained by twisting the split Cartan subgroup
by the longest element $w_-$ of the Weyl group.
\begin{enumerate}[label=(\arabic*)]
\item With appropriate normalizations, on the elements
of the half-integral set of $Sp(6)$ in the Lie algebra
of $T$,
the stable Shalika germ for the subregular unipotent
class indexed by the coset
$F^{\times\,2}$ has the form
$$\gamma_0 + C_6(\tau_E(t)),$$
where $\gamma_0$ is an elementary function of $t$
and $C_6(\tau_E(t))$ is the number of points on the elliptic
curve obtained by replacing $t_i\in E$ by
$\tau_i\in k$ in the curve of Lemma 2.5.
\item With appropriate normalizations,
the stable  subregular
Shalika germ on the half-integral set of $SO(8)$ has the form
$$\gamma_1 + C_7(\tau_E(t)),$$
where $\gamma_1$ is an elementary function of $t$
and $C_7(\tau_E(t))$ is the number of points on the elliptic
curve obtained by replacing $t_i\in E$ by $\tau_i\in k$
in the curve of Lemma 2.7.
\end{enumerate}
\endproclaim

%5
\section{The Unit Element of the Hecke Algebra}


This section will analyze the orbital integrals of the unit element
of the Hecke algebra on the group $SO(2n+1)$.
Consider the stable orbital integral 
$D(\lambda t)\Phi^{st}
(\exp(\lambda t),{\pmb 1}_K)$
 of the unit element of
the spherical Hecke algebra on $SO(2n+1)$
over the stable conjugacy class of $\exp(\lambda\,t)$,
where $t$ belongs to the half-integral set.
  Let
$E/F$ be a ramified quadratic extension, and let
$C_2(\tau_E(t))$ be the number of points on the
hyperelliptic curve of Equation 4.2.

\proclaim{Theorem 5.1}
 When $\lambda\in F^\times$ is sufficiently small,
$D(\lambda t)\Phi^{st}
(\exp(\lambda t),{\pmb 1}_K)$ is
a linear combination of the functions
$$1,\quad |\lambda|,\quad |\lambda| C_2(\tau_E(t)),$$
and functions of higher degrees of homogeneity: $o(|\lambda|)$.
In this linear combination, the coefficient of $|\lambda|C_2(\tau_E(t))$
is nonzero.  In particular, the stable orbital integrals of the
unit element of the Hecke algebra are not elementary
on $SO(2n+1)$, for $n\ge 3$.
\endproclaim

A preliminary lemma must precede the proof.
All subregular unipotent conjugacy
classes in $SO(2n+1)$ lie in the same stable conjugacy
class.  We fix an invariant form on the
stable subregular conjugacy class in $SO(2n+1)$.  To be
specific, let $e_{ij}$ denote the $(2n+1)\times(2n+1)$ matrix
whose ${ij}$-coefficient is one, and whose other coefficients
are zero. For every root $\gamma$, we
fix a root vector $X_\gamma$
in the Lie algebra of
$SO(2n+1)$ of the form $X_\gamma=e_{ij} - e_{i'j'}$, where
$i>i'$.  Let $\alpha_1$ denote the short simple root of $SO(2n+1)$.
We endow a Zariski open set of the stable subregular
unipotent class with coordinates by the equation $y^{-1} x y$, where 
\begin{equation}\tag{5.2}
x = \prod_{\gamma\ne\alpha_1\atop \gamma>0} \exp(
x(\gamma) X_\gamma),\ \ 
y = \prod_{\gamma\ne\alpha_1\atop\gamma>0}\exp(
y(\gamma) X_{-\gamma}),\quad\hbox{and } x(\gamma),\ y(\gamma)\in F.
\end{equation}
  (These
products require a fixed linear order on the positive roots.)

An invariant measure on the stable subregular class is then
$|\omega|$, where $\omega$ is the differential form
$$\bigwedge_{\gamma\ne\alpha_1} (dx(\gamma)\wedge dy(\gamma)).$$
This restricts to
an invariant measure $\mu_a$, for $a\in F^\times/
F^{\times\,2}$, on each subregular unipotent class.  
(Recall that the subregular unipotent classes in $SO(2n+1)$
are parameterized by $a\in F^\times/F^{\times\,2}$.)

\proclaim{Lemma 5.3}  The integrals $\mu_a({\pmb 1}_K)$ and
$\mu_{a'}({\pmb 1}_K)$ are equal for the two 
cosets $a,a'$ of {\it odd} valuation.
\endproclaim

\begin{proof}  If $\theta:F^\times\to\C^\times$ is a quadratic
character, let $\mu^\theta$ be the signed measure on the
stable subregular conjugacy class given by
$$\mu^\theta = \sum_{a\in F^\times/F^{\times\,2}}
  \theta(a) \mu_a.$$
To establish the lemma, it is enough to show that
$\mu^\theta({\pmb 1}_K) = \mu^{\theta'}({\pmb 1}_K)$, if
$\theta$ and $\theta'$ are the two ramified quadratic
characters of $F^\times$. This we check by direct
computation.

If $x$
is a subregular unipotent element
of the form (5.2), then its conjugacy class
has parameter $$a=x(\alpha_1+\alpha_2)^2 + 2 x(\alpha_2) x(2\alpha_1
+\alpha_2)\ \hbox{modulo}\ 
F^{\times\,2},$$ where $\alpha_2$ is the long simple root adjacent
to the short simple root $\alpha_1$. To see this, we recall how
the parameterization is defined.  The subregular unipotent class
$\calO_a$
with parameter $a$ has the characteristic property that every
irreducible component of
$\B_u$, the variety of Borel subgroups containing a given unipotent element $u\in \calO_a$,
is defined over $F(\sqrt{a})$.
More concretely $a$ is the discriminant (modulo squares) of the
quadratic polynomial in $z$ determined by the condition
$$
\exp(z X_{-\alpha_1}) x \exp(-z X_{-\alpha_1}) \in N_{2},$$
where we let $N_{i}$ denote the unipotent radical of the
upper-block triangular
parabolic subgroup associated with the short root $\alpha_i$.
A short calculation shows that the quadratic polynomial is
$$z^2 x(2\alpha_1 + \alpha_2) +2 z x(\alpha_1+\alpha_2) - 2 x(\alpha_2)=0,$$
whose discriminant gives $x(\alpha_1 +\alpha_2)^2 + 2 x(\alpha_2)
x(2\alpha_1 + \alpha_2)$ as desired.  

According to the method of \cite{H4,3.9}, there is a constant $c$ independent
of $\theta$, such that $\mu^{\theta,x} = c\mu^\theta$, where 
$\mu^{\theta,x}$ is the measure in Ranga Rao's normalization:
$$\mu^{\theta,x}(f) = \int_K\int_{N_1} f(k^{-1} x k) 
\theta(x(\alpha_1+\alpha_2)^2 + 2 x(\alpha_2) x(2\alpha_1+\alpha_2))
\left|\prod_{\gamma\ne\alpha_1} dx(\gamma)\right| dk.$$
Here $dk$ is the Haar measure of mass one on the group
$K=SO(2n+1,O_F)$.  

We must check that $\mu^{\theta,x}({\pmb 1}_K) = \mu^{\theta',x}({\pmb 1}_K)$
for the two ramified characters $\theta,\theta'$ of order two on $F^\times$.
For the unit of the Hecke algebra, the integral over $K$ is one,
and the integral over each $|dx(\gamma)|$ is also one, for
$\gamma\ne \alpha_1,\alpha_2,\alpha_1+\alpha_2$, and $2\alpha_1+\alpha_2$.
We are left to show that the integral
$$\int_{|x|\le 1,\,|y|\le 1,\,|z|\le 1} \theta(x^2 + 2 y z) |dx\,dy\,dz|$$
is independent of the ramified quadratic character $\theta$.  This is
an easy exercise that is left to the reader.
Thus, our claim is established that $\mu_a({\pmb 1}_K)=\mu_{a'}({\pmb 1}_K)$
when $a$ and $a'$ are the two cosets of odd valuation.\end{proof}

\begin{proof}[Proof (Theorem 5.1)]  The Shalika germ expansion of the stable
orbital integral of the
unit element of the Hecke algebra has the form
$$D(\lambda t)\Phi^{st}(\exp(\lambda t),{\pmb 1}_K) =
       \sum_{\calO} \Gamma_{\calO} (\lambda t)\mu_{\calO}({\pmb 1}_K),$$
where $\calO$ ranges over the unipotent classes of $G$ and $t$ is half-integral
in the Lie algebra of $G$.  
By \cite{HC}, the $r$-regular Shalika germs satisfy
$\Gamma_\calO(\lambda^2 t) = |\lambda|^{2r} \Gamma_\calO(t)$,
so the theorem calls for an analysis of the regular and
subregular terms $(r=0,1)$.
  By Shelstad \cite{Sh}, the
stable regular Shalika germ is a constant, which with an appropriate
normalization is simply the constant 1.  Now consider the subregular
terms.  By the results of \cite{H3,VI.2}, we find that
$\Gamma_\calO(\lambda t) = |\lambda | \Gamma_\calO(t)$ for the
stable subregular germs of $SO(2n+1)$.
By Lemma 4.8, each stable germ $\Gamma_\calO(t)$ is a linear combination
of the functions $1$, $C_1(\tau_E(t))$, and $C_1(\tau_{E'}(t))$,
where $E$ and $E'$ are distinct ramified quadratic extensions of $F$.
The stable germs associated with the cosets 
$a,a'$ of even valuation are $\Gamma_a(\lambda t) = \Gamma_{a'}
(\lambda t) = |\lambda|$.
The cosets $a,a'$ of odd valuation give
\begin{align*}
&|\lambda| \Gamma_{\calO_a}(t)\mu_a({\pmb 1}_K) 
   + |\lambda|\Gamma_{\calO_{a'}}(t)
\mu_{a'}({\pmb 1}_K)\\
=&|\lambda|\mu_a({\pmb 1}_K) (x_a C_1(\tau_E(t)) + x_{a'} C_1(\tau_{E'}(t))).
\end{align*}
The nonzero 
constants $x_a$ and $x_{a'}$ account for the different normalizations of
measures.  The normalization of $\Gamma_{\calO_a}(t)$ 
comes from the measure
$\mu_a$, and this is not the same as the normalization of Lemma 4.8.

The constants $x_a$ and $x_{a'}$ are equal.  To see this, we
must track down the constants of normalization that we have ignored
in going from $\Gamma_{\calO_a}$ to $C_1(\tau_E(t))$.  In the proof
of \cite{H5,1.2}, the constant ${1\over 2}\int |d\xi/\xi|$ was
discarded, where $\int |d\xi/\xi|$ is the integral over the
elements in $F(\sqrt{a})$ of norm one.  This integral is $2/q$
for every ramified extension. The constant $-1/q$ in the
statement of Lemma 4.7 was discarded in the proof of Lemma 4.8.
There is also a factor of $\sqrt{q}$ in the proof of Lemma 4.8 
arising from a
change of coordinates: $u=x/\pi_E$, and $\sqrt{q}|du/u^2|=|dx/x^2|$.
No other constants were dropped.

  With $x_a=x_{a'}$, Equation 4.12 shows that
the term
coming from cosets of odd valuation is
$$2|\lambda| C_2(\tau_E(t)) x_a \mu_a({\pmb 1_K}).$$
Finally, we note that the constant $x_a\mu_a({\pmb 1}_K)$ is nonzero.
In fact, the factor $\mu_a({\pmb 1}_K)$ is positive, being the
positive measure of a characteristic function on the orbit $\calO_a$.
 The conclusion follows.  \end{proof}

%6
\section{Fourier Transforms and Characters}

Results of Arthur, Harish-Chandra, and Kazhdan allow us to draw some
conclusions about characters and Fourier transforms of invariant
measures supported on nilpotent
orbits from our work on stable Shalika germs.

Harish-Chandra has shown that the Fourier transform $\hat\mu_\calO$ of
the invariant measure on a nilpotent conjugacy class $\calO$ is
represented by a locally integrable function
that  is locally  constant
on the regular semisimple set \cite{HC}.  
Let $T_X$ be the Cartan subgroup corresponding to a given
regular elliptic element $X$, and let $A$ be the split
component of $G$.  Let $\vvol(T_X/A)$ be the volume
of $T_X/A$ computed with respect to the measure 
used to define the algebraic measure on $G/T$ used
to compute Shalika germs in \cite{L}.

On the regular elliptic set, these Fourier
transforms are related to the
unstable Shalika germs $\Gamma^{un}_\calO(X)$ 
and discriminant $D(X)$ by the identity \cite{HC}
\begin{equation}\tag{6.1}
\sum_\calO D(X)\hat\mu_\calO(X)\Gamma^{un}_\calO(Y)\vvol(T_Y/A) = 
   \sum_\calO D(Y)\hat\mu_\calO(Y) \Gamma^{un}_\calO(X)\vvol(T_X/A),
\end{equation}
$(X,Y \ \hbox{regular elliptic})$,
where the sum runs over nilpotent conjugacy classes.  (It is harmless
to shift the indexing set back and forth from nilpotent classes to
unipotent classes.)  We view this equation as giving for each $Y$ a
linear relation between Fourier transforms
 and
the Shalika germs
 on the elliptic set.

A {\it cuspidal\/} linear combination of Shalika germs
$\sum x_\calO \Gamma^{un}_\calO(X)$ is defined to be a linear combination that
vanishes on all nonelliptic elements.  Kazhdan has shown that the
space of cuspidal linear combinations of Shalika germs coincides with
the span of Fourier transforms of nilpotent orbits \cite{K}. For every
regular elliptic $Y$, the right-hand side of Equation 6.1 is a cuspidal
linear combination of Shalika germs, and as
$Y$ varies, we span the
space of cuspidal linear combinations of Shalika germs.
In other words, Harish-Chandra's identity (6.1) gives a transition
matrix from one spanning set of functions to another.  As a result,
if cuspidal linear combinations of Shalika germs are not elementary,
then neither are certain Fourier transforms.

Results showing that Fourier transforms of nilpotent orbits are
not 
elementary carry over to characters of admissible representations.
Harish-Chandra proved that every character $\chi_\pi$ of an
irreducible
admissible representation of a  reductive $p$-adic group has an
expansion parameterized by nilpotent orbits $\calO$
\begin{equation}\tag{6.2}
\chi_\pi(\exp(X)) = \sum_\calO c_\calO(\pi) \hat\mu_\calO(X),
\end{equation}
for some collection of constant $c_\calO(\pi)$
whenever the regular
semisimple element $X$ is sufficiently small.  If a nilpotent orbit
$\hat\mu_{\O'}(X)$ is known not to be elementary, then the local
expression for
$\chi_\pi(\exp(X))$ is not elementary either
except in the degenerate
cases when the particular coefficients $c_\calO(\pi)$ cancel the
nonrational behavior from Equation 6.2.  
  Although cancellation may occur
for particular representations, it cannot happen in general because
Equation 6.2 may be inverted.  

\proclaim{Lemma 6.3}   Assume the center of $G(F)$ is
compact. Each Fourier
transform of an invariant nilpotent measure is, on the regular elliptic set,
a finite linear combination of irreducible admissible characters:
$$\hat\mu_\calO(X) = \sum_\pi \hat c_\calO(\pi)\chi_\pi(\exp(X)),$$
for $X$ sufficiently small.
\endproclaim

\begin{proof}  Kazhdan \cite{K,\S2} proves that there is a cuspidal function
$f$ whose orbital integrals on the regular elliptic set equal 
$\hat\mu_\calO(X)$.
Arthur \cite{A,5.1} has used the local trace formula to express the
orbital integrals of a cuspidal function as a linear combination
of elliptic characters (again on the regular elliptic set).  These
two results combine to give the lemma.\end{proof}

\bigskip
Thus, the nonrationality of cuspidal linear combinations of Shalika
germs implies that characters admit no general elementary description.
A cuspidal linear combinations of subregular Shalika germs
is identically zero -- except when the rank of the group is small.
As a result, other unipotent classes must be studied to prove general
nonrationality results for characters.  Nevertheless, the subregular
Shalika germs give nonrationality results for the characters on some
groups of small rank, and it is worthwhile to consider these examples.

Our first example will be the group $G=SO(5)$.
Let $C_3(\tau_E(t))$ denote the number of points on the elliptic
curve of Equation 4.3, with $t=[t_1,t_2]$ in the half-integral
set.

\proclaim{Theorem 6.4}  Fix any ramified quadratic extension $E/F$.
The space of Fourier transforms of subregular
nilpotent classes in $SO(5)$ restricted to the elliptic set is
two-dimensional.  A basis of this space on the half-integral set
is given by the two functions
$$A_T,\qquad A_T C_3(\tau_E(t)),$$
for some collection of nonzero constants $A_T$ depending only
on the stable conjugacy class of Cartan subgroups 
$T$ containing (a conjugate of) $t$.
\endproclaim

\proclaim{Corollary}  The characters of admissible representations of
$SO(5)$ (and $GSp(4)$, $Sp(4)$, and so forth) are not elementary
in general.
\endproclaim

\begin{proof}  The local character expansion on the half-integral set
will in general contain the term $C_3(\tau_E(t))$ of Theorem 6.4.\end{proof}

\smallskip
If we select a different ramified extension $E'$, we obtain a 
different function $C_3(\tau_{E'}(t))$, but the vector space spanned
remains unchanged.

The discriminant factor $D(X)$ is constant on the half-integral
set, so that it does not matter whether we include it or
not in the normalizations of the Fourier transforms. 

 Murnaghan \cite{M1} has identified a family of supercuspidal
representations on $GSp(4)$ for which the the 
local character expansions
are elementary.  The subregular term of this family lies in the
one-dimensional subspace
$(A_T)$ of the two-dimensional space  of the theorem.

\begin{proof}[Proof (Theorem 6.4)]  We write the four Shalika germs of subregular
unipotent elements as $\bGamma^{un}_1$, $\bGamma^{un}_\epsilon$, $\bGamma^{un}_\pi$,
and $\bGamma^{un}_{\pi\epsilon}$, using the representatives $1$, $\epsilon$,
$\pi$, and $\pi\epsilon$ of cosets $F^\times/F^{\times\,2}$.  The
bar indicates that we are using the normalization of \cite{H5}
(compare Lemma 4.10).
The superscript $un$ reminds us that we are now using ordinary
(unstable) Shalika germs, and no longer the stable germs used throughout
Sections 4 and 5.  We write $\bGamma^{T,\kappa}_a$ for the $\kappa$-combination
of Shalika germs on the restriction to a particular Cartan subgroup $T$,
for any character $\kappa$ of the cohomology group
$H^1(\Gal(\bF/F),T)$.

Suppose that we have a cuspidal linear combination of Shalika germs
$$\sum_a x_a \bGamma^{un}_a.$$
Each $\kappa$-combination of Shalika germs is a linear combination
of unstable Shalika germs.  Thus, cuspidality implies that
\begin{equation}\tag {6.5}
\sum x_a \bGamma^{T,\kappa}_a = 0,
\end{equation}
for every nonelliptic Cartan subgroup $T$ and every $\kappa$.
Since each unstable
Shalika germ is a linear combination of the $\kappa$-germs, we see
that Condition 6.5 for all $(T,\kappa)$ implies cuspidality.

The stable germs of $SO(5)$ are given in Equation 4.9.  The split
torus $T$ gives no constraints on cuspidality.  Suppose first
that
$T$ is a product $T_1\times T_2$, with parameter $t_i$ in the
Lie algebra of $T_i$, for $i=1,2$.
In this case $T$ has an
endoscopic group $SO(3)\times SO(3)$ (and this is the only 
nontrivial endoscopic
group of $SO(5)$), so we may consider $\kappa$-orbital integrals.
Let $\Delta^{T,\kappa}$ be the transfer factor of Langlands
and Shelstad, which we write as a function of $t_1$ and $t_2$.
The transfer to the endoscopic group (proved in \cite{H1,5.25}
and \cite{H1,page 243,Cor3})
 forces each
germ $\Delta^{T,\kappa}\bGamma^{T,\kappa}_a$
 to be a linear combination of
$\bGamma_a(t_1,0)$ and $\bGamma_a(t_2,0)$, where $\bGamma_a(t_1,0)$
is equal up to a constant to the integral of Equation 4.9, specialized to
$t_2=0$, and similarly for $\bGamma_a(t_2,0)$.
The constants relating the normalizations $\bGamma_a$
and $\Gamma_a$ are given in \cite{H5,\S2}.
  Looking at the
two extreme cases -- first when $T_1$ is split, then when
$T_2$ is split -- we find that 
$\Delta^{T,\kappa}\bGamma_a^{T,\kappa} =
\bGamma_a(t_1,0) + \bGamma_a(t_2,0)$.  As \cite{H5,2.3} points
out, the integral $\bGamma_a(t_i,0)$ is independent of $a$.
This is, in fact, the purpose of shifting to the normalization
$\bGamma_a$.
Thus, the given linear combination $\sum_a x_a\bGamma_a^{T,\kappa}$
 becomes
$$(x_1 + x_\epsilon + x_\pi + x_{\pi\epsilon}) 
  (\bGamma_a(t_1,0) + \bGamma_a(t_2,0))/\Delta^{T,\kappa}(t_1,t_2).$$
The vanishing condition of cuspidality forces $x_1+x_\epsilon +
 x_\pi + x_{\pi\epsilon} = 0$.  But when this sum is zero, the
$\kappa$-germ vanishes on all Cartan subgroups (associated with
the endoscopic group) -- not just the nonelliptic
ones.  It follows that the cuspidal  linear combinations of Shalika germs
are essentially stable:
on each Cartan subgroup, the unstable germ is a multiple (determined
by the number $h_T$ of conjugacy classes in the stable class) of the
stable germ.  Thus, we may drop the superscripts $un$ and work directly
with stable germs.

The constraint $\sum_a x_a=0$ comes from one of the parabolic subgroups.
The constraint for the other proper parabolic subgroup comes from the
duality of \cite{H5,2.1}.  This condition is $x_1=0$.  The cuspidal linear
combinations are made up of the two-dimensional space
$$(x_\epsilon \bGamma_\epsilon + x_\pi\bGamma_\pi +
  x_{\pi\epsilon} \bGamma_{\pi\epsilon})/h_T,\ \
\hbox{with } x_\epsilon+x_\pi+x_{\pi\epsilon}=0.$$
By the explicit results of \cite{H5,\S2} and Equation 6.1,
this gives an explicit basis for the Fourier transforms of
subregular orbits on elliptic elements.
By \cite{H5,2.8}, the germ $\bGamma_\epsilon$ is identically zero on the
half-integral set, giving us arbitrary linear combinations of
$\bGamma_\pi$ and $\bGamma_{\pi\epsilon}$.  The theorem now follows
from Equations 4.12 and 4.13.  The curve of Equation
4.2 has genus zero in rank two.  It follows that the
term $C_2(\tau_E(t))$ appearing in Equation 4.12 is independent
of $t$.  This gives the desired basis.
The constants $A_T$ are $A_T = \vvol(T/A)/h_T$, where
$h_T\in\{1,2\}$ is the number of conjugacy classes of
Cartan subgroups in the stable conjugacy class of $T$.
\end{proof}

\bigskip
We give one more example.  Consider the group $G=U_E(3)$ split over
a ramified quadratic extension $E/F$.  There are two subregular
unipotent conjugacy classes, parameterized by $F^\times$ modulo
the norms of $E$.  Let $\eta$ be the ramified quadratic
character of $F^\times$ associated with $E/F$.

\proclaim{Theorem 6.6}  The Fourier transforms of subregular
nilpotent measures on $U_E(3)$
span a two dimensional space on elliptic elements.
A basis of this space
on the half-integral set is formed by the functions
$$A_T \,\eta((t_1-t_2)(t_2-t_3))\ \hbox{ and }
  B_T\,(1+q- C_5(\tau_E(t))),$$
where $C_5(\tau_E(t))$ is the number of points on the elliptic
curve $\epsilon y^2 = (x+\tau_1)(x+\tau_2)(x+\tau_3)$ of
Equation 4.5.   The constants $A_T$ and $B_T$ depend only on the conjugacy
class of Cartan subgroup containing the given half-integral
element.  The constant $A_T$ is nonzero if and only if $T$ is
stably conjugate to a Cartan subgroup in
$$H=\left\{\begin{pmatrix} *&&*\\&*&\\*&&*\end{pmatrix}\right\}.$$
The constant $B_T$, defined for every elliptic Cartan subgroup $T$
meeting the half-integral set, is nonzero for every such $T$.
If $T$ and $T'$ are stably conjugate but not conjugate, then
$A_T + A_{T'} = 0$ and $B_T = B_{T'}$.
\endproclaim

\proclaim{Corollary}  The characters of representations on $U_E(3)$ are
not in general elementary.
\endproclaim

\begin{proof}  This follows easily from the results of \cite{H2}.  We sketch
the argument.  There
is only one nonelliptic Cartan subgroup up to conjugacy, and all
subregular orbital integrals vanish on it. Thus, every subregular
germ is cuspidal, and the subregular cuspidal space is two-dimensional.

There are either one or two conjugacy classes of Cartan subgroups
in the stable conjugacy class of $T$.  If there is only one,
then there is no difference between the  unstable germ 
and the stable germ. We obtain the lemma
in this case from Lemma 4.15 with $A_T=0$.

If there are two conjugacy classes, then $H$ is the endoscopic group.
If $t'$ is stably
conjugate (but not conjugate) to $t$, 
we set $\Gamma_a^{T,st}(t) = \Gamma_a^{un}(t) + \Gamma_a^{un}(t')$
and $\Gamma_a^{T,\kappa}(t) = 
 \Gamma_a^{un}(t) - \Gamma_a^{un}(t')$, for $t\in Lie(H)$.
The germ is expressed in terms
of the $\kappa$-germs as $\Gamma^{un}_a(t) = {1\over 2} (\Gamma_a^{T,st}(t) +
\Gamma_a^{T,\kappa}(t))$.
  Multiplying
$\Gamma_a^{T,\kappa}$ by the transfer factor $\eta((t_1-t_2)(t_2-t_3))$
of \cite{H2,2.1} and transferring 
the $\kappa$-germ to the endoscopic group, we find that
$\eta((t_1-t_2)(t_2-t_3))\Gamma_a^{T,\kappa}(t)$ is equal to
the stable germ on $H$, which is a constant $A_T$ on the half-integral set
\cite{Shk,2.2.2}.
Clearly $\Gamma_a^{T,\kappa}(t') = -\Gamma_a^{T,\kappa}(t)$, if
$t$ and $t'$ are stably conjugate but not conjugate.  Thus,
$A_T + A_{T'} = 0$.  Also $\Gamma_a^{T,\kappa} = \Gamma_{a'}^{T,\kappa}$
with the normalizations of \cite{H2,\S5}. 

The stable germ $\Gamma_a^{T,st}$ is given by Lemma 4.15.  With the
normalizations of \cite{H2,\S5}, the two subregular
stable germs differ by the sign
$\eta(x(\gamma))$ (appearing in the proof of Lemma 4.15). The sign
is positive on one of the subregular classes and negative on the other.
Hence $\Gamma_a^{T,st} + \Gamma_{a'}^{T,st} = 0$. 
Finally $$\Gamma^{un}_a(t) + \Gamma^{un}_{a'}(t) 
    = \Gamma_a^{T,\kappa}(t)=
A_T\eta((t_1-t_2)(t_2-t_3))$$
and $\Gamma^{un}_a(t)-\Gamma^{un}_{a'}(t) = \Gamma_a^{T,st}(t)
   =B_T(1+q- C_5(\tau_E(t)))$
as desired. \end{proof}

%7
\section{Proofs and Conclusion}

The proofs of Theorems 1.1, 1.2, and 1.3 are now nearly complete.  Lemma 6.3,
Theorem 6.4, and Theorem 6.6 prove that there is no general elementary
formula for the Fourier transform of nilpotent orbits, and that there
is no general elementary formula for characters.  Sections 4 and 5
treat Shalika germs and the unit element of the Hecke algebra.

To prove that the
Langlands principle of functoriality is not elementary,
we rely on the main example of \cite{H5}.  The embedding of $L$-groups
$${}^L\!SO(5) = Sp(4,\C)\to GL(4,\C) = {}^LGL(4)$$ 
leads to conjectures relating twisted stable
characters on $GL(4,F)$ 
and stable characters on $SO(5,F)$ (see \cite{KS1}).
Expressed dually in terms of orbital integrals, the twisted orbital
integrals on $GL(4)$ should be related to stable orbital integrals
on $SO(5)$.  Concrete calculations on the half-integral set of $GL(4)$
give points on the elliptic curve
$$ y_1^2 = 1+ a x_1^2 + b x_1^4, \ \ \hbox{ for } a,b\in k,$$
over the residue field $k$ of $F$.  On $SO(5)$, we find the elliptic
curve
$$ y_2^2 = 1 - 2 a x_2^2 + (a^2 - 4 b) x_2^4, \ \ \hbox{ for } a,b\in k.$$
For details see \cite{H5,2.8}.
These curves have different $j$-invariants.  The theory of endoscopy
then predicts that these two elliptic curves have the same number
of points.  This follows from the pair of dual isogenies between
the curves:
\begin{align*}
\phi^*x_2 = x_1/y_1,\ \qquad&\phi^*y_2 = (1-b x_1^4)/y_1^2,\\
\psi^*x_1 = 2x_2/y_2,\ \qquad&\psi^* y_1 = (1-(a^2-4b) x_2^4)/y_2^2.
\end{align*}
The theory of endoscopy and functoriality, formulated
as correspondences between varieties over finite fields,
will be
a vast generalization of this pair of dual isogenies.  We conclude
the principle of functoriality is not generally elementary.
The proof of Theorem 1.1 is now complete.

Turn to the proofs of
Theorems 1.2 and 1.3.  The stable invariant theory on the quasisplit groups
$SU_E(n)$ for $n\ge 3$, $Sp(2n)$ for $n\ge 2$ (assuming transfer), 
and $SO(2n+1)$, for $n\ge 2$,  is not
elementary by the results of Section 4.  
The inner forms of $SO(2n+1)$
and of $SU_E(2k)$ do not have elementary theories because of 
\cite{H3,VII.4.1}, 
which asserts that the stable
subregular Shalika germs are affected
only by a sign in passing to the inner form.  The groups
 $Sp(6)$ and $SO(8)$ are treated in Lemma 4.16.

For the quasisplit form of the group $SO(2n)$, 
for $n\ge 5$, we make use of the 
standard endoscopic group $H$ that is a quasisplit
form of $SO(8)\times SO(2n-8)$, with $SO(8)$ split.
(see \cite{H5,VII.1.4}).
  The following
lemma will complete the proof of Theorems 1.2 and 1.3.

\proclaim{Lemma 7.1}  Assume the transfer of Shalika germs from
the quasisplit forms $G_{adj}$ of 
$SO(2n)$ to the endoscopic group $H$ given above.  Then the
Shalika germs of $\kappa$-orbital integrals of $SO(2n)$
on the half-integral set are not in general elementary.
\endproclaim

\begin{proof}
There is a single subregular conjugacy class in $G_{adj}$, for $n\ge 3$.
The transfer of germs implies that the 
$\kappa$-combinations of subregular germ, multiplied
by the Langlands-Shelstad transfer factor \cite{LS},  is a linear combination
of stable germs on the endoscopic groups.  
Since $H$ is a product, the stable subregular germs on $H$
are linear combinations of the stable subregular
germs on the two factors $SO(8)$ and $SO(2n-8)$.  To
determine the particular linear combination that arises
in the transfer, we look at the Cartan subgroups in
the Levi factors $\G_m^{n-4}\times SO(8)$ and
$SO(2n-8)\times \G_m^4$.  For such Cartan subgroups,
the $\kappa$-combination is already stable, the Langlands-
Shelstad transfer factor is one (since the discriminant
has already been removed), and the transfer to the
endoscopic group degenerates to ordinary Levi descent.
From this, we see that the transfer gives a {\it nontrivial\/}
linear combination of the stable subregular germs
on $SO(8)$ and $SO(2n-8)$.  In particular, the stable
subregular germ of $SO(8)$ contributes a nonzero term
to the $\kappa$-orbital integrals on $G$.  The other
term coming from $SO(2n-8)$ cannot cancel this, because
they are functions of independent sets of parameters
in the Lie algebra.
By Lemma 4.16,  
the Shalika germ is not elementary on the
half-integral set of $SO(8)$.  Thus, it
cannot be elementary on $SO(2n)$ either.\end{proof}

\bigskip
\centerline{\hfil \leaders\hrule\hskip1in \hfil}
\bigskip

Objects such as the Fourier transform of nilpotent orbits,
Shalika germs, orbital integrals (especially the orbital integrals
of the unit element of the Hecke algebra) have been studied
extensively, and many interrelationships between these basic objects
have been
obtained.  Yet despite persistent efforts, explicit formulas
have been produced for these objects only in special cases such
as $GL(n)$,
and we are led to suspect that the general theory is
profoundly different from these special cases.

Experts have long harbored vague feelings
that harmonic analysis on reductive $p$-adic groups
is not elementary.  This paper quantifies some of the complexities
long felt to exist and
explains why previous efforts have encountered such difficulty:
certain methods treat invariant harmonic analysis 
as a rational theory, and
have only been successful for {\it rational groups}.
Any approach to harmonic analysis on $p$-adic groups that is
not equipped to deal with nonrational geometry is ill-fated.

The principle of functoriality, as envisioned by Langlands and
elaborated
by many, predicts relationships between the basic objects
of invariant harmonic analysis of pairs of reductive groups
whose $L$-groups
are related.  To the extent that these
basic objects are described by varieties over finite fields,
 we may reformulate the principle of functoriality geometrically
 and seek to prove
it by geometrical methods. 

This paper has
given a series of examples of nonrational varieties arising in
harmonic analysis.
These results are tentative in the sense that the fundamental
problems in this nascent theory 
remain largely unformulated and yet unsolved.  
I conclude with three problems that I see as essential to the
future progress of harmonic analysis on $p$-adic groups.
First, to establish the proper foundations for this theory, some
general structure theorems are required.  For example, give general 
arguments that
show that the basic objects of
invariant harmonic analysis are described by the number of points
on varieties over finite fields.
Second,  identify the family of varieties attached to a given group.
The examples of the paper involve hyperelliptic curves.  What
other nonrational varieties arise -- if any?
Third, produce the morphisms or correspondences of varieties predicted
by the principle of functoriality.  The isogeny of elliptic
curves described in this section should be an isolated example within
a much larger theory yet to be developed.

\section{Appendix}

In this appendix, we list some rational functions that are used
in Example 2.3.  These rational functions 
give an elliptic curve.
  We follow the notation of Example 2.3 for the group
$Sp(6)$.   A symbolic computer calculation shows that on $S^*(k^-_1)$
we have $\sigma_{-}(x) = X(x,y)$ and $\sigma_{-}(y) = Y(x,y)$,
where
$$X(x,y) - x = {(t_2 + t_3) y b_1\over b_2 b_3}, \qquad
  Y(x,y) - y = {y b_1 b_4\over b_2 b_5 b_6 },$$
and
\begin{align*}
b_1 &=  -2 t_1^2 x^2 + 4 t_1 t_2 x^2 - 2 t_2^2 x^2 + 4 t_1^2 x^3 - 8 t_1 t_2 x^3 + 
   4 t_2^2 x^3 - 2 t_1^2 x^4 + 4 t_1 t_2 x^4 \\
   &\quad
   - 2 t_2^2 x^4 - 3 t_1^2 x y + 
   4 t_1 t_2 x y - t_2^2 x y + t_1 t_3 x y - t_2 t_3 x y + 6 t_1^2 x^2 y \\
   &\quad
   - 12 t_1 t_2 x^2 y + 6 t_2^2 x^2 y - 3 t_1^2 x^3 y 
    + 8 t_1 t_2 x^3 y - 
   5 t_2^2 x^3 y - t_1 t_3 x^3 y + t_2 t_3 x^3 y \\
   &\quad - t_1^2 y^2 
    + t_1 t_2 y^2 + t_1 t_3 y^2 + 2 t_1^2 x y^2 
     - 6 t_1 t_2 x y^2 + 2 t_2^2 x y^2 - t_1^2 x^2 y^2 \\
   &\quad
   + 5 t_1 t_2 x^2 y^2 - 4 t_2^2 x^2 y^2 
    - t_1 t_3 x^2 y^2  + 2 t_2 t_3 x^2 y^2 - 
   t_1 t_2 y^3 \\
   &\quad
      + t_1 t_2 x y^3 - t_2^2 x y^3 + t_2 t_3 x y^3,\\
%
b_2 &=  -t_1^2 x + 2 t_1 t_2 x - t_2^2 x + t_1^2 x^2 - 2 t_1 t_2 x^2 + t_2^2 x^2 - 
   t_1^2 y + t_1 t_2 y \\
    &\quad + t_1^2 x y - 2 t_1 t_2 x y + t_2^2 x y + t_1 t_3 x y - 
   t_2 t_3 x y - t_1 t_2 y^2 - t_2 t_3 y^2,\\
%
b_3 &=  -t_1 x + t_2 x + 
       t_1 x^2 - t_2 x^2 - t_1 y + t_1 x y - t_2 x y + t_3 x y + t_3 y^2,\\
%
b_4 &=  -t_1^2 x + 2 t_1 t_2 x - t_2^2 x + t_1^2 x^2 - 2 t_1 t_2 x^2 + t_2^2 x^2 - 
   t_1^2 y + t_1 t_2 y + t_1^2 x y \\
   &\quad - 2 t_1 t_2 x y + t_2^2 x y + t_1 t_3 x y - 
   t_2 t_3 x y + t_1 t_3 y^2 + t_3^2 y^2,\\
%
b_5 &=  -t_1 x + t_2 x + t_1 x^2 - t_2 x^2 - t_1 y + t_1 x y - t_2 x y + t_3 x y,\\
%
b_6 &=  -t_1 x + t_2 x + t_1 x^2 - t_2 x^2 - t_1 y + 
             t_1 x y - t_2 x y + t_3 x y + t_3 y^2.
\end{align*}
It is clear that $b_i=0$ defines a rational curve for $i\ne 1$.  The
polynomial $b_1$ becomes, under the substitution $x = 1- y/x_1$, the
product of $y^2/x_1^4$ and the polynomial
\begin{align*}
b_1' &= 
%
 -2 t_1^2 x_1^2 + 4 t_1 t_2 x_1^2 - 2 t_2^2 x_1^2 -
          4 t_1 t_2 x_1^3 + 4 t_2^2 x_1^3 + 
   2 t_1 t_3 x_1^3 - 2 t_2 t_3 x_1^3 - 2 t_2^2 x_1^4 \\
  &\quad + 2 t_2 t_3 x_1^4 
         + 4 t_1^2 x_1 y - 
   8 t_1 t_2 x_1 y + 4 t_2^2 x_1 y - 3 t_1^2 x_1^2 y + 12 t_1 t_2 x_1^2 y - 
   9 t_2^2 x_1^2 y \\
   &\quad 
   - 3 t_1 t_3 x_1^2 y + 3 t_2 t_3 x_1^2 y 
   - 4 t_1 t_2 x_1^3 y 
   + 6 t_2^2 x_1^3 y + 2 t_1 t_3 x_1^3 y - 4 t_2 t_3 x_1^3 y - t_2^2 x_1^4 y 
          \\
   &\quad
   + t_2 t_3 x_1^4 y 
   - 2 t_1^2 y^2 + 4 t_1 t_2 y^2 
    - 2 t_2^2 y^2 + 3 t_1^2 x_1 y^2 
   - 8 t_1 t_2 x_1 y^2 + 5 t_2^2 x_1 y^2 \\
   &\quad + t_1 t_3 x_1 y^2 
   - t_2 t_3 x_1 y^2 - t_1^2 x_1^2 y^2 
    + 5 t_1 t_2 x_1^2 y^2 - 4 t_2^2 x_1^2 y^2 - t_1 t_3 x_1^2 y^2 \\
    &\quad
    + 2 t_2 t_3 x_1^2 y^2 
    - t_1 t_2 x_1^3 y^2 
    + t_2^2 x_1^3 y^2 - t_2 t_3 x_1^3 y^2.
\end{align*}
This polynomial is quadratic in $y$.  By completing the square it may
be brought into the form $y_1^2 - f(x_1)$.  The resulting elliptic
curve is given in the proof of Lemma 2.5.  

\newpage
%\Refs
%\widestnumber\key{BDKV}


\ref\key A \by J. Arthur 
\paper On Elliptic Tempered Characters 
\jour Acta Mathematica
\vol 171
\yr 1993
\pages 73--138
\endref

\ref\key {BDKV}
\by  J. Bernstein, P. Deligne, D. Kazhdan and M.-F. \!Vign\'eras
\book
 Repr\'esentations des groupes r\'eductifs sur un corps local
\publ Hermann
\publaddr Paris 
\yr 1984
\endref

\ref\key{CH}
\by L. Corwin, and R. Howe  
\paper Computing characters of tamely
ramified $p$-adic division algebras 
\jour Pacific. J. Math. 
\vol  73 
\yr 1977
\pages 461--477
\endref

\ref\key{CS} \by L. Corwin, and P. Sally 
\paper Discrete Series
Characters for Division Algebras and $GL_n$
\paperinfo in preparation
\endref

\ref\key{D1} \by  J. Denef
\paper On the Degree of Igusa's Local Zeta Function
   \jour Amer. J. Math. 
    \vol 109  \yr1987\pages 991--1008
\endref

\ref\key{D2} \by J. Denef 
\paper Report on Igusa's Local Zeta Function
  \jour S\'eminaire Bourbaki \yr 1990--1991 
   \paperinfo $n^\circ$ 741
\endref

\ref\key{H1} \by  T. Hales
\paper Shalika Germs on $GSp(4)$ \jour Ast\'erisque \vol 171--172
    \yr 1989  \pages 195--256
\endref

\ref\key{H2}\by  T. Hales\paper
Orbital integrals on $U(3)$
\inbook The
Zeta Function of Picard Modular Surfaces
\eds R. Langlands and
D. Ramakrishnan \publ CRM \yr1992\endref

\ref\key{H3} \by T. Hales\book  The Subregular Germ of Orbital Integrals
   \publ Memoirs AMS \vol 476 \yr1992\endref

\ref\key{H4} \by T. Hales \paper Unipotent orbits and unipotent representations on 
$SL(n)$
\yr 1993
\pages 1347--1383
\vol 115:6
\jour Amer. J. Math.\endref

\ref\key{H5} \by T. Hales \paper The Twisted Endoscopy of $GL(4)$ and $GL(5)$ :
Transfer of Shalika germs
\jour Duke Math. J.
\toappear
\yr 1994
\endref

\ref\key{HC}  \by Harish-Chandra
\paper  Admissible invariant distributions on reductive
  $p$-adic groups 
\jour \phantom{The} Queen's Papers in Pure and Applied Math. 
\vol 48 \yr 1978
  \pages 281--347\endref

\ref\key{Ho}\by  R. Howe\paper  The Fourier Transform and Germs of Characters
(Case of $GL_n$ over a $p$-adic Field)\jour Math. Ann. \vol
208\yr 1974\pages 305-322\endref

\ref\key{I}\by J.-I. Igusa\book Lectures on forms of higher degree
\publ Tata Institute of Fundamental Research\publaddr  Bombay\yr  1978
\endref


\ref\key{KLB}\by D. Kazhdan, and G. Lusztig\paper Fixed Point Varieties
on Affine Flag Manifolds\jour Israel Journal of Math.\vol 62:2
\yr 1988\pages 129--168
\paperinfo Appendix by J. Bernstein and D. Kazhdan,
{\it An example of a non-rational variety $\hat\B_N$ for
$G=Sp(6)$}
\endref

\ref\key{K}\by  D. Kazhdan\paper Cuspidal Geometry of $p$-adic Groups\jour
 J.  d'Analyse Math. \vol 47\yr 1986\pages 1-36.
\endref

\ref\key{KS1} \by  R. Kottwitz, and D. Shelstad
  \paper Twisted Endoscopy I: Definitions, Norm Mappings and Transfer
   Factors\paperinfo preprint
\endref

\ref\key{KS2}
\by R. Kottwitz, and D. Shelstad
  \paper Twisted Endoscopy II: Basic Global Theory
\paperinfo  preprint
\endref

\ref\key{L}\by  R. Langlands\paper  Orbital Integrals on Forms
of $SL(3)$, I \jour  Amer. J. Math\yr 1983\pages 465--506\endref


\ref\key{LS} \by  R. Langlands, and D.  Shelstad
\paper On the Definition of
Transfer Factors\jour  Math. Ann. \vol 278\yr 1987\pages  219--271\endref

\ref\key{M1} \by F. Murnaghan\paper
Asymptotic behaviour of supercuspidal
characters of $p$-adic $GSp(4)$\jour Comp. Math. \vol 80\yr  1991
\pages  15--54\endref

\ref\key{M2} \by F. Murnaghan\paper Local character expansions and Shalika
germs for $GL(n)$\paperinfo  preprint\endref

\ref\key{Shk} \by J. Shalika\paper  A theorem on semi-simple $\p$-adic groups
\jour Annals of Math. \vol 95\yr  1972\pages  226--242\endref

\ref\key{Sh} \by  D. Shelstad \paper A formula for regular unipotent germs
\jour Ast\'erisque \vol 171--172\yr 1989\pages  275--277\endref

\ref\key{W1}\by  J.-L. Waldspurger\paper  Sur les int\'egrales orbitales tordues
pour les groupes lin\'eairies: un lemme fondamental
\jour Canad. J. Math\vol 43:4\yr 1991\pages 852-896
\endref

\ref\key {W2}\by  J.-L. Waldspurger\paper  Quelques r\'esultats de finitude
concernant les distributions invariantes sur les alg\`ebres de Lie
$p$-adiques\paperinfo preprint\endref

\ref\key{W3}\by  J.-L. Waldspurger\paper Homog\'en\'eit\'e de certaines
distributions sur les groupes $p$-adiques\paperinfo preprint\endref

\endRefs

\end{document}


% End of File
